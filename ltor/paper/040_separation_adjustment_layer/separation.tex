% ----------------------------------------------------------
% SEPARATION OF CONCERNS
% ----------------------------------------------------------

\section{Separation of Concerns: Planning versus Control}
\label{sec:separation}

Effective production management under \LTOs{} requires a clear separation between decision
problems that operate on fundamentally different time scales, information sets, and reversibility
constraints. Conflating these problems into a single forecasting objective forces fragile assumptions
to propagate across the system. Electric Barometer instead enforces a separation of concerns between
\emph{planning} decisions and \emph{control} decisions, each governed by distinct principles.

\subsection{Supply planning and long-horizon decisions}
\label{subsec:supply-planning}

Supply planning addresses long-horizon, largely irreversible decisions such as procurement of
specialized ingredients, packaging, and distribution capacity. These decisions are made prior to
\LTO{} launch, often weeks in advance, and must account for lead times, minimum order quantities,
and contractual constraints. In this context, explicit \LTO{} demand modeling is both appropriate
and necessary. Scenario analysis, uplift assumptions, and conservative buffers are legitimate tools
for managing the risk of under-supply.

Crucially, the objective of supply planning is not intraday service optimization but feasibility.
Ordering too little may preclude execution entirely, while ordering too much primarily affects
inventory carrying cost and waste. As such, planning models may reasonably encode promotional
beliefs, marketing expectations, and worst-case contingencies. These beliefs are inherently
speculative, but they are acceptable at this stage because decisions cannot be deferred.

\subsection{Intraday production control}
\label{subsec:intraday-control}

Intraday production control operates under a different set of constraints. Decisions are made at
high frequency, are partially reversible, and are continuously informed by realized demand signals.
The objective is to balance service protection against waste while responding to unfolding
conditions in near real time.

In this regime, coupling production decisions directly to speculative \LTO{} demand forecasts is
hazardous. Errors in promotional lift estimation can translate immediately into stockouts or excess
production, and corrective action may arrive too late to prevent service degradation. Moreover,
the availability of real-time sales data reduces the marginal value of prior demand beliefs, shifting
the emphasis toward responsiveness and robustness.

Electric Barometer is explicitly designed for this control loop. It does not attempt to plan inventory
or pre-commit production quantities. Instead, it governs how forecasts are \emph{used} in the face
of uncertainty, determining which forecast signals are trusted and how aggressively they are acted
upon.

\subsection{Why decoupling matters during \LTOs{}}
\label{subsec:decoupling-importance}

During \LTO{} periods, the divergence between planning and control objectives becomes most
pronounced. Supply planning must commit in advance under uncertainty, while intraday control
must adapt quickly as uncertainty resolves. When these layers are coupled through a single demand
forecast, errors and overconfidence propagate across the entire decision chain.

By decoupling the two, Electric Barometer allows each layer to fail safely. Supply planning forecasts
may overestimate or underestimate promotional lift without destabilizing intraday execution.
Conversely, intraday production control can remain conservative and adaptive even if upstream
assumptions prove incorrect. This decoupling preserves operational stability and simplifies
postmortem analysis by isolating the source of failure.

\subsection{Interface between planning and control}
\label{subsec:planning-control-interface}

The interface between planning and control is deliberately narrow. Rather than passing expected
\LTO{} demand levels into production control, the planning layer provides context and constraints:
ingredient availability, maximum feasible throughput, and known limitations. The control layer
then operates within these bounds, guided by realized demand and readiness signals.

In Electric Barometer, \LTO{} activation enters the control loop only as a contextual indicator that
alters forecast trust and risk posture. This signal is consumed by the \RAL{}, which modulates
readiness in response to planned instability without embedding specific demand beliefs. The result
is a control system that is aware of promotional context yet insulated from speculative lift
assumptions.

\subsection{Implications for system design}
\label{subsec:separation-implications}

This separation-of-concerns architecture has direct implications for system design and governance.
Forecast accuracy metrics, promotional modeling, and scenario planning remain valuable tools, but
they are confined to decision layers where their assumptions are appropriate. Intraday production
decisions, by contrast, are governed by readiness, uncertainty, and asymmetric loss.

By enforcing this separation, Electric Barometer aligns each decision layer with its true operational
objective. In the sections that follow, we formalize how \LTO{} context is incorporated into the
control layer through readiness adjustment and asymmetry-aware evaluation, completing the
architecture for robust production management under planned regime shifts.