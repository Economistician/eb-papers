% ----------------------------------------------------------
% CASE PATTERNS AND FAILURE MODES
% ----------------------------------------------------------

\section{Case Patterns and Failure Modes}
\label{sec:patterns}

The abstract principles described in prior sections manifest in recognizable operational patterns
during \LTO{} deployments. Examining these patterns clarifies both the failure modes of traditional
forecast-centric systems and the stabilizing effects of a readiness-centric approach. This section
summarizes common behaviors observed in practice and situates Electric Barometer’s design choices
relative to these outcomes.

\subsection{Pattern 1: Belief-driven overcommitment}
\label{subsec:belief-overcommitment}

A frequent failure mode during \LTO{} launches is belief-driven overcommitment. In this pattern,
production plans are scaled aggressively based on a single promotional lift estimate, often derived
from historical analogs or marketing expectations. When realized demand underperforms these
assumptions, stores are left with excess production, elevated waste, and reduced flexibility for
later dayparts.

This failure mode arises from coupling production decisions directly to speculative demand beliefs.
Forecast accuracy metrics may indicate success \emph{ex ante}, yet operational outcomes deteriorate
when assumptions fail. Electric Barometer avoids this pattern by refusing to treat uplift estimates as
authoritative inputs to intraday control, instead favoring readiness-adjusted conservatism.

\subsection{Pattern 2: Accuracy chasing under instability}
\label{subsec:accuracy-chasing}

Another common pattern is reactive accuracy chasing. When early \LTO{} performance diverges from
expectations, forecasting systems are rapidly retuned or manually overridden in an attempt to
restore numerical accuracy. These interventions often lag realized demand and amplify volatility,
resulting in oscillatory production behavior.

Electric Barometer mitigates this failure mode by shifting the objective away from point accuracy.
Because readiness degradation is expected during \LTO{} regimes, forecast misses are not treated as
exceptions requiring immediate correction. Instead, asymmetric loss and readiness posture govern
selection, allowing production behavior to remain stable while uncertainty resolves naturally.

\subsection{Pattern 3: Late-day shortfall and irreversibility}
\label{subsec:late-day-shortfall}

A particularly costly failure mode occurs when conservative production early in the day is not
revisited under rising demand, leading to late-day shortfalls. In traditional systems, reluctance to
adjust forecasts upward without strong confidence can delay corrective action until it is too late to
recover service.

Under Electric Barometer, this pattern is addressed through the intraday control loop. Readiness
adjustment biases early decisions toward protection against shortfall, while continuous ingestion of
realized demand allows selection to shift as empirical evidence accumulates. This combination
reduces the likelihood of irreversible late-day service failures.

\subsection{Pattern 4: Postmortem misattribution}
\label{subsec:postmortem-misattribution}

Postmortems following \LTO{} underperformance frequently attribute poor outcomes to forecasting
inaccuracy alone. This narrow framing obscures whether decisions were reasonable given the risk
environment and often leads to unnecessary model complexity or ad hoc overrides in subsequent
promotions.

Electric Barometer reframes postmortems around decision quality. By making readiness posture and
loss asymmetry explicit, it becomes possible to assess whether outcomes reflected appropriate risk
management rather than predictive failure. This distinction supports more constructive learning and
avoids repeated cycles of reactive model tuning.

\subsection{Summary of contrasting behaviors}
\label{subsec:pattern-summary}

Table~\ref{tab:pattern-summary} summarizes the contrast between traditional forecast-centric
approaches and the readiness-centric patterns encouraged by Electric Barometer during \LTO{}
periods.

% ----------------------------------------------------------
% TABLE: Forecast-Centric vs Readiness-Centric Patterns
% ----------------------------------------------------------

\begin{table}[htbp]
\centering
\caption{Contrasting forecast-centric and readiness-centric operational patterns during \LTO{} periods}
\label{tab:pattern-summary}
\begin{tabular}{@{}p{4cm}p{5cm}p{5cm}@{}}
\toprule
\textbf{Aspect} & \textbf{Forecast-Centric Pattern} & \textbf{Readiness-Centric Pattern} \\
\midrule
Decision driver & Promotional lift belief & Readiness and loss asymmetry \\
Response to miss & Retune or override model & Maintain posture; adapt via signals \\
Early-day bias & Cautious until confident & Conservative against shortfall \\
Late-day recovery & Often delayed & Incremental and responsive \\
Postmortem focus & Accuracy failure & Decision alignment \\
\bottomrule
\end{tabular}
\end{table}

These patterns illustrate how Electric Barometer alters system behavior not by improving
predictive accuracy during \LTOs{}, but by changing how uncertainty is governed. The following
section discusses limitations and non-goals, clarifying the scope within which these design choices
are intended to operate.