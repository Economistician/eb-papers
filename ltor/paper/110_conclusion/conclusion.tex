% ----------------------------------------------------------
% CONCLUSION
% ----------------------------------------------------------

\section{Conclusion}
\label{sec:conclusion}

Limited-Time Offers (\LTOs{}) expose a fundamental tension in operational forecasting: the periods in
which demand is most important to manage are often the periods in which it is least predictable.
Traditional forecast-centric approaches respond to this tension by attempting to restore predictive
accuracy through increasingly complex demand models. This paper has argued that such efforts are
misaligned with the realities of intraday production management under planned instability.

Electric Barometer adopts a different posture. Rather than treating \LTOs{} as anomalies to be
modeled away, it treats them as regime shifts that warrant explicit readiness degradation and
asymmetry-aware decision-making. By separating supply planning from intraday control, adjusting
forecast trust through the \RAL{}, and selecting forecasts based on operational loss rather than
numerical accuracy, the framework enables production systems to remain stable, explainable, and
robust when forecasts are known to be unreliable.

A central implication of this approach is that being ``wrong'' about demand is not inherently a
failure. What matters is whether decisions are aligned with the asymmetric risks present at the time
they are made. Electric Barometer formalizes this principle by encoding risk posture, loss asymmetry,
and readiness as first-class governance constructs rather than implicit modeling assumptions.

The result is a production management architecture that can increase preparedness during hyped
\LTO{} launches without requiring accurate prediction of hype itself. More broadly, the readiness-
centric perspective advanced here suggests a shift in how operational forecasting systems should be
evaluated and governed: away from accuracy as an end in itself, and toward decision alignment under
uncertainty as the defining measure of success.

As part of the Electric Barometer Series, this paper situates \LTO{} management within a broader
framework for forecast readiness. Together with cost-weighted and tolerance-based readiness
primitives, it contributes to a unified view of operational forecasting in which uncertainty is not an
exception to be eliminated, but a condition to be governed.