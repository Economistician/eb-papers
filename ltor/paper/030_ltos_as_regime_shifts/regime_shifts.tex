% ----------------------------------------------------------
% LTOS AS REGIME SHIFTS
% ----------------------------------------------------------

\section{\LTOs{} as Regime Shifts}
\label{sec:regime-shifts}

From an operational forecasting perspective, \LTOs{} are best understood not as incremental demand
features but as temporary regime shifts in the demand-generating process. Unlike routine variation
driven by seasonality or trend, \LTOs{} introduce coordinated, time-bounded changes to guest
behavior, menu composition, and operational execution. These changes alter not only the level of
demand, but also its variance, structure, and sensitivity to error.

A regime shift, in this context, is characterized by a breakdown in the assumptions under which
historical forecast calibration remains valid. Error distributions observed during steady-state
periods no longer reliably describe future outcomes once an \LTO{} is active. As a result, both
point forecasts and their associated uncertainty estimates become misaligned with realized
behavior.

\subsection{Characteristics of \LTO{}-driven regimes}
\label{subsec:lto-characteristics}

Several properties distinguish \LTO{} regimes from steady-state demand environments:

\begin{enumerate}[leftmargin=*, itemsep=2pt]
  \item \textbf{Abrupt onset.} \LTO{} launches create discrete discontinuities rather than gradual
  transitions, invalidating local smoothness assumptions commonly exploited by forecasting
  models.
  
  \item \textbf{Nonstationary variance.} Demand volatility typically increases during launch and
  early sustain phases, with variance evolving over the course of the promotion rather than
  remaining constant.
  
  \item \textbf{Asymmetric error consequences.} The operational cost of under-forecasting often
  rises sharply during promotional peaks, while over-forecasting costs may remain bounded.
  
  \item \textbf{Substitution and cannibalization.} Incremental sales attributable to an \LTO{} are
  frequently offset by declines in adjacent menu items, obscuring true demand signals at the
  item level.
  
  \item \textbf{Heterogeneous execution.} Store-level compliance, staffing, and supply availability
  vary widely, producing uneven realization of promotional intent.
\end{enumerate}

These properties jointly imply that \LTO{} periods are not well modeled as simple covariate shifts
or additive demand adjustments. Instead, they represent structural breaks in the relationship
between historical data and near-term outcomes.

\subsection{Why historical calibration fails under \LTOs{}}
\label{subsec:calibration-failure}

Most forecasting systems rely, implicitly or explicitly, on the stability of residual behavior. Model
selection, parameter tuning, and uncertainty estimation are all calibrated using historical
forecast--actual pairs assumed to be representative of future conditions. When an \LTO{} is
introduced, this assumption fails in predictable ways.

Residual distributions widen, skew, and often become multimodal as launch dynamics interact with
local execution constraints. Forecasts that were well-calibrated under steady-state conditions may
remain numerically reasonable yet become operationally fragile. Importantly, this degradation is
not a signal of poor modeling practice; it is a consequence of applying steady-state assumptions
to a non-steady-state regime.

Attempting to ``fix'' this mismatch by injecting promotional lift estimates into the forecasting model
does not resolve the underlying instability. It merely replaces one set of assumptions with another,
often less observable and harder to audit.

\subsection{Planned instability as a first-class concept}
\label{subsec:planned-instability}

A defining feature of \LTOs{} is that their destabilizing effect is intentional. Unlike weather shocks,
supply disruptions, or system outages, \LTO{}-driven regime shifts are planned in advance and
explicitly designed to perturb demand. This makes them neither anomalies nor noise, but
first-class operational events.

Electric Barometer treats this planned instability as an explicit contextual signal rather than an
implicit modeling challenge. The activation of an \LTO{} conveys reliable information about the
forecasting environment: variance is expected to increase, historical calibration is expected to
weaken, and the asymmetry of error costs is expected to intensify. These facts hold regardless of
whether the realized promotional lift is large, moderate, or negligible.

\subsection{Implications for production management}
\label{subsec:regime-implications}

Recognizing \LTOs{} as regime shifts has direct implications for production management. During
such periods, the objective of forecasting systems should shift from precision toward robustness.
The relevant decision is not which forecast best predicts demand, but which forecast choice leads
to the most resilient operational outcome given elevated uncertainty and skewed risk.

This perspective motivates the readiness-centric approach developed in subsequent sections.
Rather than embedding \LTO{} effects directly into demand predictions, Electric Barometer
responds to regime shifts by adjusting forecast trust through the \RAL{} and shaping decision
selection through asymmetry-aware readiness primitives. In doing so, it preserves operational
stability without requiring accurate prediction of promotional outcomes.