% ----------------------------------------------------------
% CONCLUSION
% ----------------------------------------------------------
\section{Conclusion}

The No-Shortfall Level provides a focused and interpretable measure of how often a
forecast meets operational requirements within individual intervals. Its binary
structure captures the frequency of demand coverage and highlights temporal
patterns of shortfalls that aggregate error metrics may obscure. The \NSL\ metric
is broadly applicable in settings where interval-level alignment between
forecasted and realized demand is essential to maintaining performance.

While \NSL\ does not reflect the magnitude of errors or the costs associated with
surpluses, it offers clear advantages for diagnosing reliability and identifying
patterns of operational strain. When applied in appropriate contexts and used
alongside complementary metrics, \NSL\ supports a meaningful and actionable
evaluation of forecasting performance. Importantly, \NSL\ functions as an
evaluative diagnostic of reliability rather than a prescriptive execution or
control rule, supporting readiness assessment without dictating operational
decisions.
