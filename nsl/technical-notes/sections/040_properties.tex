% ----------------------------------------------------------
% PROPERTIES
% ----------------------------------------------------------
\section{Properties and Behavior}

The \NSL\ metric exhibits several properties that are relevant for interpretation
and application in operational forecasting contexts.

\subsection{Range and Interpretability}

The \NSL\ metric lies within $[0,1]$ and admits a direct interpretation as the
proportion of intervals in which realized demand was met. Thresholds for acceptable
performance are application-specific, but the bounded scale supports intuitive
interpretation across domains and use cases.

\subsection{Sensitivity to Forecast Bias}

Directional bias affects \NSL\ asymmetrically. A downward-biased forecast is likely
to exhibit a lower \NSL, as a greater number of intervals will experience shortfalls.
Conversely, an upward-biased forecast may yield higher \NSL\ values, reflecting more
frequent coverage. As a result, \NSL\ is useful for detecting the operational
consequences of bias even when average error metrics appear balanced.

\subsection{Sensitivity to Temporal Clustering}

Because each interval contributes equally to the metric, \NSL\ reflects not only
the number of shortfalls but also their temporal distribution. Forecasts that
concentrate shortfalls in peak or high-impact periods may exhibit similar \NSL\
values to forecasts with shortfalls spread uniformly across time, even though the
operational implications differ. \NSL\ makes these patterns visible without relying
on aggregate error measures.

\subsection{Invariance to Demand Scaling}

Multiplying all $y_{it}$ and $\yhat_{it}$ values by a positive constant does not
affect the comparison $\yhat_{it} \ge y_{it}$. The \NSL\ metric is therefore invariant
to demand scaling and can be applied consistently across items, locations, or
systems with substantially different demand volumes.

\subsection{Binary Event Structure}

The \NSL\ metric evaluates performance at the level of binary events: each interval
either meets demand or does not. This structure simplifies interpretation and
prevents the metric from being dominated by extreme values. However, it also implies
that \NSL\ does not capture the severity of shortfalls. Forecasts with different
error magnitudes may therefore exhibit identical \NSL\ values if they fail in the
same number of intervals.

\subsection{Sampling Variability}

As an empirical proportion, \NSL\ is subject to sampling variability. For a sample
of $|T|$ intervals, the variance of the estimator is given by $p(1-p)/|T|$, where
$p$ denotes the true probability that demand is met in a given interval. Confidence
intervals may be constructed using standard binomial methods. This consideration is
most relevant when comparing \NSL\ values across models or items with differing
numbers of evaluation intervals.