% ----------------------------------------------------------
% OPERATIONAL MOTIVATION
% ----------------------------------------------------------
\section{Operational Motivation}

In many applications, meeting demand within each interval is more important than
minimizing average deviation over time. Operational systems such as production
lines, service environments, and short-horizon scheduling processes depend on
consistent alignment between forecasted and realized demand. Shortfalls, even when
small, may immediately affect throughput, capacity utilization, or service levels.

The \NSL\ metric quantifies how often a forecast provides adequate coverage, making
it a natural indicator of interval-level reliability. High $\NSL$ values suggest
that shortfalls are infrequent and that the forecast generally supports
uninterrupted execution. Low $\NSL$ values may signal recurring mismatches between
forecasted and required capacity.

The metric is particularly informative when shortfalls cluster in critical
periods. Because $\NSL$ evaluates each interval independently, it highlights
patterns in which shortfalls are concentrated in peak or high-impact windows, even
when overall average accuracy appears reasonable.

Finally, $\NSL$ aligns with common operational preferences. In many settings, a
slight surplus is preferable to a shortfall, particularly when the cost of unmet
demand is high. Because $\NSL$ does not penalize surplus magnitude, it supports
interpretations rooted in operational sufficiency rather than strict error
minimization.