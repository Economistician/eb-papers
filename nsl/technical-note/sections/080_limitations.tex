% ----------------------------------------------------------
% LIMITATIONS
% ----------------------------------------------------------
\section{Limitations and Appropriate Use}

Although the \NSL\ metric offers an interpretable measure of interval-level
reliability, several limitations should guide its application.

\subsection{Lack of Severity Information}

The \NSL\ metric does not indicate how large shortfalls are when they occur.
Forecasts that occasionally miss demand by large amounts may exhibit the same
\NSL\ value as forecasts that frequently miss by small amounts.

\subsection{Insensitivity to Surplus Magnitude}

Intervals in which forecasted demand exceeds realized demand contribute equally to
\NSL, regardless of surplus magnitude. In operational settings where excess
capacity is costly, \NSL\ should therefore be supplemented with additional metrics
that reflect surplus severity.

\subsection{Dependence on Interval Specification}

The choice of interval length affects the interpretation of \NSL. Shorter
intervals may reveal shortfalls that are masked under aggregation, while longer
intervals can obscure timing-specific failures.

\subsection{Interpretation in Low-Demand Settings}

When demand is sparse or intermittent, \NSL\ values may be dominated by
zero-demand intervals. In such cases, additional context or complementary metrics
may be required to interpret results meaningfully.