% ----------------------------------------------------------
% USE CASES
% ----------------------------------------------------------
\section{Use Cases Across Operational Domains}

The \NSL\ metric applies broadly to settings in which operational outcomes depend
on meeting demand within discrete intervals.

\subsection{Production and Manufacturing}

Shortfalls in component, material, or throughput forecasts may interrupt
production flow or create downstream bottlenecks. The \NSL\ metric quantifies how
reliably forecasts support uninterrupted operation by measuring the frequency
with which required demand levels are met.

\subsection{Inventory and Replenishment}

Retail and distribution systems depend on satisfying demand at fine temporal
resolutions. The \NSL\ metric serves as an interval-level analogue to availability
measures and helps identify stockout tendencies or timing misalignments that may
not be evident from aggregate error statistics. In inventory and replenishment
contexts, \NSL\ is conceptually related to service level measures commonly used in
production and inventory management \citep{silver1998inventory}.

\subsection{Service and Staffing Operations}

Call centers, hospitality environments, and other service systems rely on
forecast-aligned staffing to meet fluctuating workloads. The \NSL\ metric
highlights how often staffing levels are sufficient to meet realized demand within
each interval, providing a direct indicator of service reliability.

\subsection{Logistics and Transportation}

Forecasting parcel volume, passenger flow, or freight capacity requires ensuring
capacity sufficiency at each decision interval. The \NSL\ metric provides a direct
measure of how consistently such requirements are met, independent of shipment or
passenger volume scale.