% ----------------------------------------------------------
% CONCLUSION
% ----------------------------------------------------------
\section{Conclusion}

Cost-Weighted Service Loss (\CWSL) provides a principled, operationally aligned
approach to evaluating forecast performance in environments where the
consequences of under- and overforecasting differ meaningfully. By applying
explicit penalty weights to shortfall and overbuild magnitudes and normalizing
by realized demand, \CWSL{} quantifies the effective fraction of throughput
“lost’’ due to misalignment between forecasted and actual values. This
interpretation makes the metric especially useful in high-frequency,
readiness-oriented contexts where service reliability and cost exposure must be
managed jointly.

\CWSL{} complements traditional symmetric accuracy metrics by emphasizing the
\emph{cost of error} rather than its purely numerical magnitude. As a result, it
provides insight into operational risk that may remain hidden when evaluation
relies solely on RMSE, MAE, or related measures. When paired with
frequency-based diagnostics, distributional analysis, and readiness-focused
metrics, \CWSL{} helps create a richer and more actionable understanding of
forecasting system performance.

As organizations increasingly rely on automated forecasting for production,
staffing, logistics, replenishment, and energy management, metrics such as
\CWSL{} enable more transparent decision-making and more resilient operational
planning. Future extensions may include probabilistic formulations, integration
with scenario-based cost models, or dynamic adjustment of penalty weights to
reflect changing conditions or risk preferences. Together, these developments
further extend the connection between statistical forecasting and the operational
realities it must ultimately support.