% ----------------------------------------------------------
% LIMITATIONS
% ----------------------------------------------------------
\section{Limitations and Appropriate Use}

While Cost-Weighted Service Loss (\CWSL) provides an operationally aligned,
asymmetric evaluation of forecast error, several limitations should be
considered when applying the metric or comparing models across settings.

\subsection{Dependence on Cost Weights}

Because \CWSL{} relies on user-specified penalty weights \((\cui, \coi)\),
interpretability depends on selecting values that accurately reflect the true
operational trade-offs. If cost weights are misaligned with system priorities,
the resulting \CWSL{} values may misrepresent risk or readiness.

Accordingly:
\begin{itemize}
    \item cost weights should be grounded in domain knowledge or empirical study,
    \item sensitivity analysis is recommended when costs are uncertain,
    \item models should be compared only under a consistent cost structure.
\end{itemize}

\subsection{Sensitivity to Outliers and Large Deviations}

Because \CWSL{} aggregates cost-weighted error magnitudes, a small number of
large deviations can dominate the metric. Operational systems with heavy-tailed
or volatile demand patterns should interpret large \CWSL{} values carefully and
may benefit from complementary diagnostics that characterize error distribution.

\subsection{No Direct Information on Error Frequency}

\CWSL{} captures the \emph{cost-weighted magnitude} of forecast error but does
not distinguish how often errors occur. Two forecasts may produce identical
\CWSL{} values despite very different shortfall frequencies. For this reason,
\CWSL{} is often interpreted alongside frequency-based measures (e.g.,
No–Shortfall Level).

\subsection{Lack of Temporal Differentiation}

Unless temporal importance is encoded directly into the cost weights, \CWSL{}
treats all intervals equally. In settings where certain periods (e.g., peaks,
events, promotions) are disproportionately important, additional weighting or
separate evaluation may be required.

\subsection{Scale Sensitivity Across Heterogeneous Entities}

Although normalization by realized demand improves comparability, entities with
extremely small volumes may exhibit unstable or highly variable \CWSL{} values.
Entity-level smoothing, segmentation, or minimum-volume thresholds may be needed
for robust comparison.

\subsection{Appropriate Role Within a Broader Evaluation Framework}

\CWSL{} is not a substitute for symmetric accuracy metrics or distributional
diagnostics. Rather, it is a complementary measure designed to capture the
operational cost of directional error. When used together with traditional error
metrics and readiness-focused measures, \CWSL{} supports a more complete
assessment of forecasting system performance.