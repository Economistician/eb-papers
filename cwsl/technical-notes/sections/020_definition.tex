% ----------------------------------------------------------
% DEFINITION AND MATHEMATICAL FORMULATION
% ----------------------------------------------------------
\section{Definition and Mathematical Formulation}

Cost-Weighted Service Loss (\CWSL) quantifies the cost-adjusted impact of
forecast error as a fraction of total realized demand. The metric incorporates
asymmetric penalties for underbuild and overbuild, allowing users to encode
operational priorities directly into the evaluation process.

Let \(i\) index entities (e.g., products, locations) and let \(t \in T\) index
evaluation intervals. For each entity–interval pair, define shortfall and
overbuild magnitudes as
\[
\sit = \pospart{\yit - \yhatit},
\qquad
\oit = \pospart{\yhatit - \yit},
\]
where \(\yit \ge 0\) denotes realized demand and \(\yhatit \ge 0\) the
corresponding forecast.

Let \(\cui > 0\) and \(\coi > 0\) denote the asymmetric penalty weights for
shortfall and overbuild, respectively. The use of asymmetric cost weighting is
consistent with longstanding results showing that forecast errors can impose
directionally distinct economic consequences \citep{granger1969}. The
system-level Cost-Weighted Service Loss is defined as
\[
\CWSL
=
\frac{
\sumit
\left(
    \cui \,\sit
    +
    \coi \,\oit
\right)
}{
\sumit \yit
}.
\]

\subsection*{Interpretation}

\CWSL{} measures the fraction of effective throughput ``lost’’ under the chosen
cost structure. A value of zero indicates perfect alignment between forecast and
realized demand (or zero penalty weights). Larger values indicate increasing
operational inefficiency driven by either the magnitude or the frequency of
errors, scaled by their cost relevance.

% ----------------------------------------------------------
% FIGURE: ASYMMETRIC PENALTY CURVE (Modular)
% ----------------------------------------------------------
\input{figures/cwsl_penalty_curve}