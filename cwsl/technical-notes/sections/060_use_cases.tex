% ----------------------------------------------------------
% USE CASES
% ----------------------------------------------------------
\section{Use Cases Across Operational Domains}

Cost-Weighted Service Loss (\CWSL) is particularly useful in environments where
underforecasting and overforecasting have materially different operational
consequences. Because the metric applies explicit penalty weights to shortfall
and overbuild magnitudes, it allows practitioners to align model evaluation with
the true cost structure of their system. This section highlights several domains
in which asymmetric error penalties are essential for readiness-oriented
forecasting.

\subsection{Production and Manufacturing}

In make-to-order and high-utilization manufacturing environments, underbuilding
can cause throughput loss, work-in-process buildup, or schedule instability.
Overbuilding, while inefficient, is often less disruptive. \CWSL{} provides a
cost-weighted assessment that prioritizes the avoidance of shortfalls, helping
identify models that maintain stable production flow.

\subsection{Retail and Replenishment}

Retail and omnichannel fulfillment systems experience asymmetric impacts from
forecast error: underforecasting drives stockouts and lost sales, while
overforecasting increases inventory carrying cost or spoilage risk. By adjusting
\((\cu, \co)\) to match these economics, \CWSL{} yields a readiness-oriented
measure that reflects service protection and inventory efficiency simultaneously.

\subsection{Staffing and Service Operations}

In labor-intensive service systems—such as restaurants, call centers, field
service operations, and healthcare—underforecasting demand leads to long queues,
service degradation, and lost revenue. Overstaffing is typically less harmful.
\CWSL{} captures this asymmetry and supports selection of forecasting methods
that minimize service failures while balancing labor efficiency.

\subsection{Logistics and Transportation}

Routing, fleet management, and distribution workflows are sensitive to demand
shortfalls, which create bottlenecks, delays, and capacity imbalances. Excess
forecasting generally results in modest underutilization. \CWSL{} makes these
trade-offs explicit, enabling operational teams to compare models based on their
expected cost exposure.

\subsection{Energy and Utilities}

Load forecasting in energy distribution, generation scheduling, and grid
operations exhibits pronounced asymmetry: underforecasting load may require
expensive balancing or reserve activation, while overforecasting typically leads
to curtailment or minor efficiency loss. \CWSL{} provides a natural evaluation
framework that reflects this operational reality.

\subsection{Cross-Entity or Portfolio Evaluation}

Because \CWSL{} normalizes by realized demand, it scales naturally across
products, stores, dayparts, network nodes, or customer segments. This makes the
metric useful for identifying pockets of asymmetric risk within a broader
operational portfolio and for diagnosing entities where forecast misalignment
has outsized impact.