% ----------------------------------------------------------
% PROPERTIES
% ----------------------------------------------------------
\section{Properties and Behavior}

The Cost-Weighted Service Loss (\CWSL) metric exhibits several mathematical and
operational properties that make it well suited for environments where
forecast errors have asymmetric consequences. This section summarizes the most
important characteristics governing how \CWSL{} behaves under different patterns
of error.

\subsection{Non-Negativity}

Since shortfall and overbuild magnitudes are nonnegative and the penalty weights
\(\cui, \coi > 0\), the cost-weighted numerator is always nonnegative.
Therefore,
\[
\CWSL \ge 0.
\]

\subsection{Demand Normalization}

Dividing by total realized demand produces a scale-free metric that is comparable
across entities of different sizes. The resulting value can be interpreted as the
fraction of effective throughput ``lost’’ due to misalignment between forecast
and demand.

\subsection{Asymmetric Sensitivity}

When \(\cui > \coi\), shortfalls contribute more heavily to the loss than
overbuilds. This encodes directional operational priorities: environments where
underbuilding is more disruptive will exhibit higher \CWSL{} values for the same
numerical error profile.

\subsection{Dependence on Both Frequency and Magnitude}

\CWSL{} increases with both the number of errors and the size of those errors
once cost-weighted. A forecast that underbuilds frequently but by small amounts
may produce the same \CWSL{} as a forecast that occasionally underbuilds by
larger amounts. This dual dependence enables flexible and context-aware
evaluation.

\subsection{Entity-Level Customization}

Different entities (products, locations, dayparts, equipment classes) may incur
different operational costs for underbuilding or overbuilding. The parameters
\(\cui\) and \(\coi\) allow \CWSL{} to flexibly encode these differences without
changing the metric’s structural interpretation.

\subsection{Monotonicity in Cost Weights}

For fixed forecasts and realizations, increasing either \(\cui\) or \(\coi\)
always increases the value of \CWSL{}. This monotonicity ensures that increasing
the penalty attached to an error type never reduces its measured impact.

% ----------------------------------------------------------
% FIGURE: MODEL COMPARISON (Modular)
% ----------------------------------------------------------
\input{figures/cwsl_model_comparison}

\subsection{Illustrative Comparison of Error Profiles}

Figure~\ref{fig:cwsl_model_comparison} shows two forecasts with similar
symmetric error but different \CWSL{} outcomes. Model~A consistently underbuilds,
resulting in higher accumulated shortfall cost. Model~B exhibits balanced errors,
resulting in substantially lower cost exposure despite comparable RMSE or MAE.
This demonstrates that \CWSL{} distinguishes operationally risky forecasts even
when symmetric metrics do not.