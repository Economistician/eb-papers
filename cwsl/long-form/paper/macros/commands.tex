% ----------------------------------------------------------
% commands.tex -- Custom macros for the CWSL long-form paper
% ----------------------------------------------------------

% ---------- Sets / Indexing ----------
\newcommand{\Iset}{\ensuremath{\mathcal{I}}}     % set of items
\newcommand{\Tset}{\ensuremath{\mathcal{T}}}     % set of intervals

% ---------- Core Variables ----------
\newcommand{\yit}{\ensuremath{y_{it}}}           % actual demand
\newcommand{\yhatit}{\ensuremath{\hat{y}_{it}}}  % forecast
\newcommand{\sit}{\ensuremath{s_{it}}}           % shortfall
\newcommand{\oit}{\ensuremath{o_{it}}}           % overbuild

% ---------- Cost Parameters ----------
\newcommand{\cu}{\ensuremath{c_{u,i}}}           % shortfall penalty
\newcommand{\co}{\ensuremath{c_{o,i}}}           % overbuild penalty
\newcommand{\Ri}{\ensuremath{R_i}}               % penalty ratio

% ---------- Main Metrics ----------
\newcommand{\cwsl}{\ensuremath{\mathrm{CWSL}}}   % Cost-Weighted Service Loss
\newcommand{\nsl}{\ensuremath{\mathrm{NSL}}}     % No-Shortfall Level
\newcommand{\ud}{\ensuremath{\mathrm{UD}}}       % Underbuild Depth
\newcommand{\wMAPE}{\ensuremath{\mathrm{wMAPE}}} % Weighted MAPE
\newcommand{\frs}{\ensuremath{\mathrm{FRS}}}     % Forecast Readiness Score

% ---------- Formatting Helpers ----------
\newcommand{\E}{\ensuremath{\mathbb{E}}}         % expectation
\newcommand{\Var}{\ensuremath{\mathrm{Var}}}     % variance
\newcommand{\cov}{\ensuremath{\mathrm{cov}}}     % covariance