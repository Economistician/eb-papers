% ----------------------------------------------------------
% CONTRIBUTIONS
% ----------------------------------------------------------
\section{Contributions}

This paper makes three primary contributions to the forecasting and operations
management literature:

\begin{enumerate}[itemsep=8pt]
    \item \textbf{A new forecast evaluation metric grounded in asymmetric operational
    cost.} We introduce \textit{Cost-Weighted Service Loss (CWSL)}, a demand-normalized
    metric that incorporates explicit penalties for shortfalls and overbuilds. CWSL
    captures the directional consequences of forecast error at the interval level,
    addressing a gap in existing evaluation methods that assume symmetric cost.

    \item \textbf{A diagnostic framework for operational readiness assessment.} Beyond the primary metric, we develop four complementary measures—No-Shortfall Level (NSL), Hit Rate within Tolerance (HR@\(\tau\)), Underbuild Depth (UD), and the Forecast Readiness Score (FRS). Together, these metrics provide a structured approach for identifying failure modes, assessing tolerance accuracy, and quantifying the severity of operational shortfalls, and they are defined in a way that supports aggregation across items, categories, stores, and other operational hierarchies.

    \item \textbf{Evidence of cross-domain applicability and operational relevance.}
    Through illustrative examples and industry-specific discussion, we demonstrate how
    CWSL applies to a wide range of high-frequency operational settings, including QSR,
    retail replenishment, manufacturing, logistics, workforce planning, and energy load
    forecasting. This highlights the generalizability of CWSL as a unifying framework for
    evaluating forecast performance under asymmetric error structures.
\end{enumerate}

These contributions position CWSL as both a methodological advancement and a practical
tool for organizations seeking to align forecasting evaluation with operational
priorities.