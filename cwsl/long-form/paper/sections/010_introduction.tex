% ----------------------------------------------------------
% INTRODUCTION
% ----------------------------------------------------------
\section{Introduction}

Forecasts play a central role in operational decision-making in environments where
timing, readiness, and service reliability matter. Quick-service restaurants (QSR),
retail stores, production lines, and high-frequency inventory systems all rely on
forecasts to ensure that product, labor, or capacity is available exactly when needed.
In such environments, forecast errors translate directly into operational outcomes.

Most organizations evaluate forecast performance using symmetric accuracy metrics such
as mean absolute error (MAE), root mean squared error (RMSE), and mean absolute
percentage error (MAPE). These measures treat equal-magnitude errors identically
regardless of direction, implicitly assuming that the operational cost of
under-forecasting is no different from the cost of over-forecasting. However, this
assumption rarely holds in practice. Under-forecasting often produces service failures,
unmet demand, and elevated operational stress, while over-forecasting typically leads
to excess production or inventory that carries far lower cost.

This mismatch between statistical evaluation and operational impact can lead
organizations to select models that appear accurate on paper yet consistently produce
readiness failures. Forecasts may achieve low MAPE while still causing shortfalls during
peak periods—precisely when operational reliability is most important. As organizations
increasingly rely on short-horizon forecasts for high-frequency decisions, symmetric
evaluation metrics become inadequate for assessing real operational performance.

To address this gap, this paper introduces the \textit{Cost-Weighted Service Loss
(CWSL)} metric, which evaluates forecast performance by explicitly incorporating
asymmetric penalties for shortfalls and overbuilds and normalizing results by total
demand. CWSL is complemented by supporting measures that provide insight into
reliability, tolerance accuracy, and the severity of shortfalls. Together, these metrics
offer a practical and interpretable framework for assessing forecast quality in
environments where direction, magnitude, and operational consequence all matter.