% ----------------------------------------------------------
% CONCLUSION
% ----------------------------------------------------------
\section{Conclusion}

High-frequency operational environments require evaluation methods that reflect not only
the magnitude of forecast error, but the asymmetric operational consequences associated
with it. Traditional symmetric accuracy metrics such as MAE, RMSE, and MAPE assume that
over-forecasting and under-forecasting impose equivalent cost. In practice, this
assumption is rarely valid. Under-forecasting disrupts service delivery, reduces
throughput, triggers recovery delays, and can degrade both customer experience and
operational reliability. Over-forecasting, by contrast, typically results in modest and
predictable excess. As a result, organizations that rely on symmetric metrics may
misjudge model performance and overlook readiness-related failures that materially
affect operations.

This paper introduced \textit{Cost-Weighted Service Loss (CWSL)}, an evaluation metric
that embeds operational cost asymmetry directly into its formulation. By combining
directional penalty weights with demand normalization and interval-level granularity,
CWSL provides a more operationally meaningful assessment of forecast performance. The
supporting diagnostic metrics—NSL, HR@\(\tau\), UD, and the composite Forecast Readiness
Score (FRS)—offer a multi-dimensional view of reliability, tolerance accuracy, and
shortfall severity. Together, these measures establish a unified framework for evaluating
forecast quality in environments where readiness and service reliability are primary
objectives.

The illustrative example demonstrated that symmetric metrics, particularly wMAPE, can
substantially understate the operational impact of shortfall-heavy forecasts. CWSL, by
contrast, captures both the frequency and severity of asymmetric error, providing a more
accurate reflection of real-world performance. Applications across multiple industries—
including quick-service restaurants, retail, manufacturing, logistics, supply chain
planning, workforce management, and energy forecasting—underscore the broad relevance of
this metric in domains where shortages impose disproportionately high cost.

More broadly, this research highlights the importance of aligning analytical evaluation
techniques with the environments they are intended to support in practice. As forecasting
systems become increasingly integrated into real-time and near-real-time decision
processes, evaluation metrics must reflect the economic and operational consequences of
error, not merely its statistical properties. CWSL provides a practical and extensible
foundation for achieving this alignment, enabling organizations to select, monitor, and
refine forecasting models that better support reliability, service performance, and
operational resilience.