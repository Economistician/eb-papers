% ----------------------------------------------------------
% ILLUSTRATIVE EXAMPLE
% ----------------------------------------------------------
\section{Illustrative Example}

To demonstrate how \cwsl{} and its supporting diagnostic metrics behave in practice, we
construct an example involving two items with differing operational characteristics.
Item \(A\) represents a high-volume, high-impact product (e.g., a core entrée), for which
shortfalls are especially costly due to recovery time and throughput effects. Item \(B\)
represents a lower-volume, lower-impact item where recovery is quicker and the
operational consequences of misses are less severe.

To reflect these differences, shortfall penalties are set to \(\cu = 6\) for Item \(A\)
and \(\cu = 3\) for Item \(B\), while overbuild penalties are \(\co = 2\) and \(\co = 1\),
respectively. These values mirror typical operational asymmetry in environments where
shortages impose substantially higher cost than excess. Forecasts are evaluated over
twelve consecutive intervals, representative of a peak service window with meaningful
variability.

Table~\ref{tab:example-expanded} summarizes actual demand, forecasts, shortfalls,
overbuilds, and resulting penalties for both items across all intervals.

\begin{table}[h!]
\centering
\small
\begin{tabular}{@{}cccccccccc@{}}
\toprule
 & \multicolumn{4}{c}{Item A (Core)} & & \multicolumn{4}{c}{Item B (Secondary)} \\
\cmidrule(lr){2-5} \cmidrule(lr){7-10}
Interval &
$y$ & $\hat{y}$ & $s$ & $o$ &
 &
$y$ & $\hat{y}$ & $s$ & $o$ \\
\midrule
1  & 20 & 22 & 0 & 2 & & 8 & 9 & 0 & 1 \\
2  & 28 & 25 & 3 & 0 & & 7 & 9 & 0 & 2 \\
3  & 32 & 29 & 3 & 0 & & 10 & 8 & 2 & 0 \\
4  & 35 & 33 & 2 & 0 & & 11 & 11 & 0 & 0 \\
5  & 40 & 36 & 4 & 0 & & 12 & 10 & 2 & 0 \\
6  & 42 & 45 & 0 & 3 & & 9 & 11 & 0 & 2 \\
7  & 38 & 34 & 4 & 0 & & 8 & 7 & 1 & 0 \\
8  & 30 & 32 & 0 & 2 & & 7 & 8 & 0 & 1 \\
9  & 26 & 22 & 4 & 0 & & 6 & 5 & 1 & 0 \\
10 & 24 & 27 & 0 & 3 & & 5 & 7 & 0 & 2 \\
11 & 22 & 23 & 0 & 1 & & 8 & 9 & 0 & 1 \\
12 & 18 & 21 & 0 & 3 & & 9 & 8 & 1 & 0 \\
\bottomrule
\end{tabular}
\caption{Illustrative example showing actual demand ($y$), forecast ($\hat{y}$),
shortfall ($s$), and overbuild ($o$) for two items over twelve intervals.}
\label{tab:example-expanded}
\end{table}

\subsection{Cost-Weighted Penalties}

The cost-weighted penalty for item \(i\) in interval \(t\) is defined as
\[
\text{Penalty}_{it} = \cu s_{it} + \co o_{it},
\]
which assigns substantially higher cost to shortfalls than to overbuilds. For Item~\(A\),
each unit of shortfall and overbuild incurs penalties of 6 and 2, respectively; for
Item~\(B\), the corresponding penalties are 3 and 1.

Summing across items and intervals yields a total penalty of
\[
\sum_{i \in \{A,B\}} \sum_{t=1}^{12}
(c_{u,i} s_{it} + c_{o,i} o_{it}) = 186.
\]

Total realized demand across the horizon is
\[
\sum_{i \in \{A,B\}} \sum_{t=1}^{12} y_{it} = 340,
\]
and the Cost-Weighted Service Loss is therefore
\[
\cwsl = \frac{186}{340} = 0.547.
\]

A value of \(\cwsl = 0.547\) indicates that, when weighted by asymmetric penalties,
forecast errors generated an operational burden equivalent to approximately 54.7\% of
total demand. This reveals how a modest number of high-penalty shortfall intervals can
dominate overall performance, even when symmetric error appears moderate.

\subsection{Supporting Metrics}

We compute additional diagnostic metrics to provide visibility into error structure
beyond what \cwsl{} alone captures.

\begin{itemize}[itemsep=4pt]

    \item \textbf{No-Shortfall Level (NSL).}
    Across the 24 item–interval pairs, forecasts meet or exceed actual demand in 10
    cases:
    \[
    \nsl = \frac{10}{24} = 0.417.
    \]
    Fewer than half of the evaluation intervals achieved full service.

    \item \textbf{Hit Rate within Tolerance (HR@\(\tau = 2\)).}
    The forecast lies within \(\pm2\) units of actual demand in 15 of 24 cases:
    \[
    \mathrm{HR}@2 = \frac{15}{24} = 0.625.
    \]
    Although tolerance accuracy appears respectable, it conceals the severity of key
    shortfall intervals.

    \item \textbf{Underbuild Depth (UD).}
    Conditional on a shortfall occurring, the average magnitude of the miss is
    \[
    \ud = 2.82 \text{ units}.
    \]
    Shortfalls, when they occur, are operationally meaningful rather than minor.

    \item \textbf{Weighted MAPE (wMAPE).}
    Total absolute error across both items is 78 units, yielding
    \[
    \wMAPE = \frac{78}{340} = 0.229.
    \]
    Symmetric error appears modest, but fails to reflect asymmetric operational risk.

\end{itemize}

\subsection{Comparison Against Symmetric Metrics}

This example highlights a sharp divergence between symmetric and asymmetric evaluation.
Although \wMAPE{} suggests moderate average error (\(0.229\)), the Cost-Weighted Service
Loss is more than twice as large (\(0.547\)), indicating significantly higher
operational burden. The discrepancy arises because \wMAPE{} assigns equal weight to
over-forecasting and under-forecasting, allowing relatively benign overbuild intervals
to offset severe shortfalls.

NSL indicates that full demand was met in only 41.7\% of item–intervals despite the
moderate symmetric error. UD confirms that when shortfalls occur, their magnitude is
nontrivial, amplifying operational risk.

\cwsl{} resolves these inconsistencies by integrating shortfall frequency, shortfall
depth, and asymmetric penalty weights into a single demand-normalized measure. It
therefore exposes readiness failures that symmetric metrics systematically obscure.

\subsection{Interpretation}

The illustrative example underscores several key properties of \cwsl{} in a realistic
operational setting:

\begin{enumerate}[itemsep=4pt]
    \item \textbf{Shortfalls disproportionately drive operational cost.}
    High-penalty shortfall intervals dominate the total cost-weighted penalty even when
    symmetric error is moderate.

    \item \textbf{Symmetric metrics obscure readiness failures.}
    Measures such as \wMAPE{} understate operational risk because overbuilds offset
    shortfalls in the average.

    \item \textbf{CWSL integrates all relevant dimensions of service readiness.}
    By combining shortfall frequency, shortfall depth, and penalty asymmetry into a
    single metric, \cwsl{} provides an evaluation aligned with actual operational
    impact.
\end{enumerate}

These insights reinforce the need for cost-weighted, directionally aware evaluation
methods in high-frequency operational environments where symmetric accuracy measures
offer an incomplete and sometimes misleading view of forecast performance.