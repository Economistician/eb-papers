% ----------------------------------------------------------
% FUTURE RESEARCH DIRECTIONS
% ----------------------------------------------------------
\section{Future Research Directions}

The Cost-Weighted Service Loss (CWSL) framework offers several avenues for further
research at the intersection of forecasting, operations management, and applied
analytics. Because CWSL unifies asymmetric cost, interval-level performance, and
demand normalization within a single metric, it provides a flexible foundation for
advancing both methodological work and application-driven inquiry.

\subsection{Integration into Model Training and Optimization}

While CWSL is introduced here as an evaluation metric, an important extension is its
incorporation into model-training objectives. Embedding asymmetric penalties directly
into gradient-based forecasting models, quantile regression methods, boosted trees, or
neural architectures could align model estimation more closely with operational cost.
Future work may compare models optimized on CWSL-inspired loss functions against those
trained on symmetric or probabilistic objectives to quantify performance trade-offs.

\subsection{Automated and Data-Driven Penalty Calibration}

Penalty ratios currently rely on domain expertise or scenario-based tuning. Research
opportunities exist in developing data-driven methods for calibrating penalty weights
using historical operational outcomes, cost modeling, or reinforcement learning. Such
approaches could help organizations dynamically infer penalty structures that reflect
service-level targets, congestion risk, or marginal cost curves. This remains an open
problem with both methodological and applied relevance.

\subsection{Context-Aware and Time-Varying Penalty Structures}

Operational asymmetry is rarely constant. Daypart effects, system bottlenecks, seasonal
shift patterns, and location-specific characteristics all influence the relative cost of
shortfalls and overbuilds. Extending CWSL to incorporate time-varying or
context-conditioned penalty weights may improve evaluation fidelity, particularly in
multi-item, multi-location systems. Estimating these penalties from historical behavior
represents a promising direction for future empirical work.

\subsection{Forecast Monitoring, Drift Detection, and Control}

CWSL provides a natural basis for performance monitoring in production systems.
Researchers may explore how CWSL behaves under distributional shift, concept drift, or
data-quality degradation, as well as how it can be incorporated into statistical
process-control charts or anomaly-detection frameworks. Threshold-based intervention
rules or early-warning indicators derived from CWSL could support more proactive
forecast governance.

\subsection{Simulation, Scenario Planning, and Decision Support}

Another research direction lies in integrating CWSL into simulation and scenario-planning
environments. Discrete-event simulation, agent-based modeling, or stochastic operations
models could incorporate CWSL-based performance metrics to quantify the operational
impact of alternative forecast strategies, staffing rules, inventory policies, or
production schedules. Such tools would enhance both managerial insight and methodological
understanding of asymmetric error in complex systems.

\subsection{Connections to Optimization and Control Models}

CWSL may also serve as a bridge between forecasting and downstream optimization
problems. Future work could investigate how CWSL-informed forecasts alter the structure
or optimality of scheduling, routing, labor allocation, or inventory-control models.
Embedding asymmetric cost into multi-stage decision models could yield new formulations
that explicitly account for the differing impacts of shortage and surplus.

\subsection{Empirical Validation and Cross-Domain Benchmarking}

Finally, additional empirical research is needed to benchmark CWSL across industries,
temporal resolutions, and model classes. Large-scale comparative studies—similar in
spirit to the M-competitions—could evaluate how CWSL aligns with or diverges from
traditional metrics and readiness-based KPIs. Such benchmarking would help establish
normative performance ranges, best practices for penalty selection, and guidelines for
deploying CWSL in operational workflows.

\medskip

Overall, CWSL introduces a flexible and operationally aligned basis for evaluating
forecast performance under asymmetric cost. Extending this foundation through model
training, automated calibration, monitoring, simulation, and empirical validation
represents a promising research agenda at the intersection of forecasting science and
operations management.