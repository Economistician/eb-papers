% ----------------------------------------------------------
% ABSTRACT
% ----------------------------------------------------------
\begin{abstract}
Forecast performance in high-frequency operational environments is typically evaluated
using symmetric error metrics such as mean absolute error (MAE), root mean squared error
(RMSE), and mean absolute percentage error (MAPE). These metrics implicitly assume that
under-forecasting and over-forecasting impose equivalent cost, despite substantial
evidence that shortages create significantly higher operational burden than excess.
Across domains such as quick-service restaurants (QSR), retail, manufacturing,
transportation, and workforce management, shortfalls can lead to unmet demand, degraded
service performance, and recovery delays, whereas overbuilds generally result in modest
waste or temporary overcapacity.

This paper introduces \textit{Cost-Weighted Service Loss (CWSL)}, a demand-normalized evaluation metric that incorporates explicit asymmetric penalties for shortfalls and overbuilds and supports aggregation across multiple items and intervals. CWSL provides a more operationally aligned assessment of forecast quality by
capturing both the magnitude and direction of error at the interval level. We further
propose a set of diagnostic measures—No-Shortfall Level (NSL), Hit Rate within Tolerance
(HR@\(\tau\)), Underbuild Depth (UD), and the composite Forecast Readiness Score (FRS)—that
enable practitioners and analysts to evaluate service reliability, tolerance accuracy,
and shortfall severity.

Through an illustrative example and cross-industry applications, we show that CWSL
reveals readiness-related performance patterns that symmetric metrics systematically
obscure. The framework offers a practical and generalizable basis for evaluating
forecasts in environments where operational reliability and asymmetric error cost are
central concerns.
\end{abstract}

% ----------------------------------------------------------
% MANAGERIAL INSIGHTS (IJAA REQUIRED)
% ----------------------------------------------------------
\begin{framed}
\noindent \textbf{Managerial Insights} \\[4pt]
Organizations that rely on short-horizon forecasts—such as restaurants, retailers,
manufacturers, logistics networks, and workforce operations—often experience service
failures not because forecasts are statistically inaccurate, but because errors occur in
the wrong direction. Traditional evaluation metrics treat being “5 units short” the same
as being “5 units long,” even though shortfalls typically lead to unmet demand, slower
service, operational recovery time, or lost revenue, while overages usually result only
in modest waste or temporary excess capacity.

The \textit{Cost-Weighted Service Loss (CWSL)} metric makes these operational realities
visible by weighting shortfalls more heavily than overbuilds and normalizing performance
by total demand. This enables managers to distinguish between forecasts that are
numerically accurate but operationally risky, and those that reliably support readiness.

The supporting metrics—No-Shortfall Level (NSL), Hit Rate within Tolerance (HR@\(\tau\)),
Underbuild Depth (UD), and the Forecast Readiness Score (FRS)—allow managers to pinpoint
the sources of readiness failures and prioritize corrective actions. These measures help
leaders identify when problems stem from frequent misses, severe shortages, inadequate
tolerance accuracy, or misaligned penalty structures.

Taken together, CWSL and its diagnostic framework give managers a practical toolkit for:
\begin{itemize}[itemsep=4pt]
    \item evaluating forecast performance in terms of service reliability rather than
          statistical error alone,
    \item selecting forecasting models that minimize operational risk,
    \item calibrating penalty structures to reflect true cost asymmetry across items,
    \item improving readiness through more targeted production, staffing, and
          replenishment decisions.
\end{itemize}
\end{framed}