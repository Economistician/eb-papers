% ----------------------------------------------------------
% ILLUSTRATIVE EXAMPLE
% ----------------------------------------------------------
\section{Illustrative Example}

To illustrate the behavior of Cost-Weighted Service Loss (\CWSL), consider a
single entity evaluated over twelve intervals. Consistent with many operational
settings where underforecasting is more harmful than overforecasting, we adopt
an asymmetric penalty structure with
\[
\cu = 2, 
\qquad 
\co = 1.
\]

Table~\ref{tab:cwsl_example} reports realized demand \(y_t\), forecast values
\(\hat{y}_t\), raw errors \(e_t = \hat{y}_t - y_t\), and the corresponding
cost-weighted contributions for each interval.

\begin{table}[h!]
\centering
\begin{tabular}{c c c c c}
\toprule
Interval $t$ & $y_t$ & $\hat{y}_t$ & Error $e_t$ & CWSL Contribution \\
\midrule
1  & 20 & 22 &  \phantom{-}2 & $c^o \cdot 2 = 2$ \\
2  & 28 & 25 & -3            & $c^u \cdot 3 = 6$ \\
3  & 32 & 29 & -3            & $c^u \cdot 3 = 6$ \\
4  & 35 & 36 &  \phantom{-}1 & $c^o \cdot 1 = 1$ \\
5  & 40 & 37 & -3            & $c^u \cdot 3 = 6$ \\
6  & 42 & 45 &  \phantom{-}3 & $c^o \cdot 3 = 3$ \\
7  & 38 & 34 & -4            & $c^u \cdot 4 = 8$ \\
8  & 30 & 31 &  \phantom{-}1 & $c^o \cdot 1 = 1$ \\
9  & 26 & 22 & -4            & $c^u \cdot 4 = 8$ \\
10 & 24 & 27 &  \phantom{-}3 & $c^o \cdot 3 = 3$ \\
11 & 22 & 23 &  \phantom{-}1 & $c^o \cdot 1 = 1$ \\
12 & 18 & 19 &  \phantom{-}1 & $c^o \cdot 1 = 1$ \\
\bottomrule
\end{tabular}
\caption{Cost-weighted contributions for a 12-interval example under a 
penalty structure with $c^u = 2$ and $c^o = 1$.}
\label{tab:cwsl_example}
\end{table}

Summing across intervals, the total cost-weighted deviation is
\[
\text{Numerator} 
= 2 + 6 + 6 + 1 + 6 + 3 + 8 + 1 + 8 + 3 + 1 + 1 
= 46.
\]

The denominator is total realized demand:
\[
\text{Denominator}
= \sum_{t=1}^{12} y_t
= 20 + 28 + 32 + 35 + 40 + 42 + 38 + 30 + 26 + 24 + 22 + 18
= 357.
\]

Thus, the Cost-Weighted Service Loss for this example is
\[
\CWSL
= \frac{46}{357}
\approx 0.129.
\]

\subsection*{Interpretation}

A \CWSL{} value of approximately \(0.129\) indicates that, under the assumed cost
structure, the forecast produced cost-weighted deviations equivalent to losing
roughly 13\% of realized throughput. Despite several intervals with small
overbuilds, the repeated underbuild events—magnified by the higher shortfall
penalty—dominate the loss. This example illustrates how \CWSL{} highlights
directional operational risk even when symmetric error metrics (e.g., RMSE or
MAE) may appear moderate.