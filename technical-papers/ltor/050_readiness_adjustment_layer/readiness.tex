% ----------------------------------------------------------
% READINESS ADJUSTMENT LAYER
% ----------------------------------------------------------

\section{The Readiness Adjustment Layer}
\label{sec:ral}

The Readiness Adjustment Layer (\RAL{}) formalizes how contextual instability alters the degree to
which forecasts should be trusted for operational decision-making. Rather than modifying demand
predictions directly, \RAL{} operates upstream of evaluation and selection, adjusting the effective
readiness of forecasts in response to known risk conditions. In the context of \LTOs{}, \RAL{}
provides the mechanism by which planned regime shifts influence intraday production behavior
without introducing speculative demand assumptions.

% ----------------------------------------------------------
% FLOW DIAGRAM: LTO Readiness-Centric Production Control
% ----------------------------------------------------------

\begin{figure}[htbp]
\centering

\resizebox{\linewidth}{!}{%
\begin{tikzpicture}[
    node distance=1.2cm and 1.6cm,
    font=\scriptsize,
    box/.style={
        draw,
        rectangle,
        rounded corners,
        align=center,
        text width=3.1cm,
        minimum height=1.0cm,
        inner sep=4pt
    },
    arrow/.style={->, thick}
]

% ----------------------------------------------------------
% NODES
% ----------------------------------------------------------

\node[box] (forecast) {Raw Forecasts\\(Any Model)};
\node[box, below=of forecast] (context) {\LTO{} Context\\(\ltoOn{}, \ltoPhase{})};

\node[box, right=of forecast] (ral) {Readiness Adjustment Layer\\(\RAL{})};
\node[box, right=of ral] (evaluation) {Asymmetry-Aware\\Evaluation\\(\CWSL{}, \NSL{})};
\node[box, right=of evaluation] (selection) {Forecast Selection};

\node[box, below=of selection] (production) {Intraday Production\\Decisions};

% ----------------------------------------------------------
% ARROWS
% ----------------------------------------------------------

\draw[arrow] (forecast) -- (ral);
\draw[arrow] (context) -- (ral);

\draw[arrow] (ral) -- (evaluation);
\draw[arrow] (evaluation) -- (selection);
\draw[arrow] (selection) -- (production);

% ----------------------------------------------------------
% ANNOTATIONS
% ----------------------------------------------------------

\node[above=2pt of ral] {\footnotesize Readiness Modulation};
\node[above=2pt of evaluation] {\footnotesize Loss-Based Scoring};

\end{tikzpicture}%
}

\caption{Readiness-centric production control under \LTO{} conditions in the Electric Barometer framework.
\LTO{} activation enters the system as contextual information that degrades forecast readiness via the
Readiness Adjustment Layer (\RAL{}), rather than as a demand uplift. Forecasts are evaluated under
asymmetry-aware loss and selected to minimize expected operational harm before being translated
into intraday production actions.}
\label{fig:lto-readiness-flow}
\end{figure}

\subsection{Purpose and positioning of the \RAL{}}
\label{subsec:ral-purpose}

Within the Electric Barometer architecture, \RAL{} sits between raw forecast generation and
readiness-aware evaluation. Its role is not to correct forecasts, improve accuracy, or estimate
demand uplift. Instead, it answers a narrower operational question:

\begin{quote}
\emph{Given current operating conditions, how much confidence should be placed in any forecast,
regardless of how it was produced?}
\end{quote}

This distinction is critical during \LTO{} periods. Forecasts may remain numerically reasonable yet
become operationally fragile due to increased variance, structural breaks, or execution
heterogeneity. \RAL{} captures this fragility explicitly, allowing downstream decision logic to
respond conservatively when trust is degraded.

\subsection{\LTOs{} as exogenous readiness shocks}
\label{subsec:lto-readiness-shocks}

As formalized in Section~\ref{sec:regime-shifts}, Electric Barometer treats \LTO{} activation as an
exogenous readiness shock. The presence of an active \LTO{} conveys reliable information about the
forecasting environment independent of any realized demand signal: uncertainty is elevated,
historical calibration is weakened, and the cost of error is likely to be asymmetric.

Formally, \LTO{} context enters the system through a binary or categorical indicator (e.g.,
\ltoOn{} or \ltoPhase{}), which is consumed by the \RAL{} rather than by the forecasting model
itself. This design ensures that the system responds to the \emph{existence} of planned instability
without embedding assumptions about its magnitude. Whether the promotion ultimately drives
substantial lift or negligible change, the readiness posture during the launch window reflects the
known increase in risk.

\subsection{Readiness modulation rather than demand adjustment}
\label{subsec:readiness-modulation}

A central design principle of \RAL{} is that readiness modulation is preferable to demand
adjustment under uncertainty. Adjusting demand forecasts upward to reflect expected \LTO{} lift
requires committing to a specific belief about future behavior; when that belief is wrong, the error
propagates directly into production decisions.

By contrast, readiness modulation alters how forecasts are evaluated and acted upon. Reduced
readiness during \LTO{} periods may manifest as increased tolerance for forecast dispersion,
heightened sensitivity to downside risk, or more conservative selection among candidate forecasts.
These responses increase preparedness without asserting that demand will increase by any
particular amount.

\subsection{Interaction with asymmetry-aware evaluation}
\label{subsec:ral-asymmetry}

The impact of \RAL{} is realized through its interaction with asymmetry-aware readiness primitives.
When readiness is reduced, downstream metrics such as \CWSL{} and \NSL{} exert greater influence
on forecast selection and decision support. In effect, \RAL{} amplifies the operational consequences
of error during high-risk periods without altering the underlying forecasts.

During hyped \LTO{} windows, this interaction biases production behavior toward shortfall
avoidance and service protection, particularly in peak dayparts. Importantly, this bias emerges from
governed loss asymmetry rather than from hard-coded demand uplift factors, preserving
explainability and auditability.

\subsection{Temporal dynamics of readiness}
\label{subsec:ral-temporal}

Readiness adjustment need not be static over the life of an \LTO{}. Launch, sustain, and wind-down
phases often exhibit distinct risk profiles. Early launch periods may warrant aggressive readiness
degradation due to uncertainty and execution variance, while later phases may allow partial
recovery as empirical demand signals accumulate.

The \RAL{} accommodates these dynamics by allowing readiness modulation to depend on phase-level
context (\ltoPhase{}) or elapsed time since activation. Such modulation remains rule-based and
transparent, avoiding adaptive behavior that could obscure governance or complicate postmortem
analysis.

\subsection{Role of \RAL{} in production management}
\label{subsec:ral-role}

By construction, \RAL{} does not attempt to predict demand, allocate inventory, or schedule
production. Its sole function is to ensure that forecast usage reflects the current risk environment.
In doing so, it enables Electric Barometer to operate effectively during periods when forecasts are
known \emph{ex ante} to be least reliable.

In the following section, we show how readiness-adjusted forecasts are evaluated and selected using
asymmetry-aware loss primitives, completing the mechanism by which \LTO{} context influences
intraday production decisions without relying on promotional demand models.