% ----------------------------------------------------------
% OVERVIEW
% ----------------------------------------------------------

\section{Overview}
\label{sec:overview}

Limited-Time Offers (LTOs) are intentionally disruptive. In quick-service restaurant (QSR)
operations, they are deployed to shift guest behavior, stimulate incremental traffic, and refresh
menu engagement. The operational consequence of this disruption is predictable: LTO periods
systematically degrade forecast reliability. Launch effects, substitution dynamics, execution
variance across stores, and supply constraints together create a regime in which classical notions of
forecast ``accuracy'' become unstable and operationally misleading.

Rather than framing this degradation as a modeling deficiency to be corrected, this paper treats
LTO handling in production management as a readiness and governance problem. The relevant
question is not whether promotional lift or incremental transactions can be predicted precisely, but
whether the operational system can remain stable, explainable, and economically disciplined when
such predictions fail. Electric Barometer adopts this perspective by treating LTOs as exogenous
readiness shocks: events that reliably increase uncertainty and error asymmetry, regardless of the
magnitude of realized lift.

A central design principle is separation of concerns. Long-horizon planning problems—such as
procurement of specialized ingredients (e.g., buns, proteins, packaging)—may require explicit LTO
demand estimates and scenario-based buffers. Intraday production control, however, is a distinct
decision loop: it is fast, partially reversible, and continuously informed by realized demand.
Electric Barometer therefore does not depend on promotional lift models to govern production
behavior. Instead, it modulates forecast trust through the Readiness Adjustment Layer (\RAL{}) and
shapes decision-making through asymmetry-aware readiness primitives such as cost-weighted and
shortfall-sensitive evaluation.

\subsection{Contributions}
\label{subsec:contributions}

This paper makes the following contributions within the Electric Barometer Series:
\begin{enumerate}[leftmargin=*, itemsep=2pt]
  \item \textbf{Reframing of LTO forecasting.} We formalize LTO periods as planned regime shifts in
  which predictive accuracy is structurally degraded and should not be treated as the primary
  operational objective.

  \item \textbf{Separation of planning and control.} We propose a clean division between
  supply-planning forecasts (used for ordering and long-lead decisions) and intraday
  production-control decisions (governed by readiness, uncertainty, and asymmetric loss).

  \item \textbf{Readiness-centric LTO integration.} We define how LTO activation enters the
  production management pipeline as an exogenous context signal that reduces readiness via
  the Readiness Adjustment Layer, rather than as a demand uplift coefficient.

  \item \textbf{Asymmetry-driven production behavior.} We show how governed asymmetric evaluation
  (e.g., via \CWSL{} and shortfall-sensitive primitives) can increase preparedness during hyped LTO
  windows without requiring accurate hype prediction.
\end{enumerate}

\subsection{Scope and non-goals}
\label{subsec:scope-nongoals}

This paper clarifies what Electric Barometer does \emph{not} attempt to do:
\begin{enumerate}[leftmargin=*, itemsep=2pt]
  \item We do not propose a universal model for predicting LTO lift, hype, or marketing response.
  \item We do not treat LTO demand estimates as authoritative inputs to intraday production control.
  \item We do not optimize forecasts to appear accurate during regime change at the expense of
  operational stability.
\end{enumerate}

Instead, we present an architecture and vocabulary for operating safely under known instability,
where decision quality is defined by operational consequence rather than symmetric error.

\subsection{Organization of the paper}
\label{subsec:organization}

Section~\ref{sec:background-problem} summarizes why \LTOs{} destabilize classical evaluation and
production heuristics. Section~\ref{sec:regime-shifts} formalizes \LTOs{} as regime shifts and
outlines their operational failure modes. Section~\ref{sec:separation} details the
separation-of-concerns architecture between planning and control. Section~\ref{sec:ral} introduces
the role of the Readiness Adjustment Layer under promotional stress. Section~\ref{sec:asymmetry}
presents asymmetry-aware readiness primitives and their implications for production selection.
Section~\ref{sec:intraday-loop} describes an implementable intraday control loop.
Sections~\ref{sec:governance}--\ref{sec:patterns} discuss governance, reporting, and common
deployment patterns. Section~\ref{sec:limitations} documents limitations and non-goals, and
Section~\ref{sec:conclusion} concludes.