% ----------------------------------------------------------
% INTRADAY CONTROL LOOP
% ----------------------------------------------------------

\section{Intraday Control Loop}
\label{sec:intraday-loop}

Intraday production management is a closed-loop control problem. Decisions are made repeatedly
over short horizons, informed by realized demand signals and constrained by operational capacity.
Under \LTOs{}, this loop must operate in an environment of elevated uncertainty, where forecasts are
known to be less reliable but service risk is higher. Electric Barometer operationalizes readiness
adjustment and asymmetry-aware selection within a disciplined intraday control loop designed to
respond safely to unfolding conditions.

\subsection{Control loop structure}
\label{subsec:control-structure}

At each intraday decision point, the control loop proceeds through a fixed sequence:
\begin{enumerate}[leftmargin=*, itemsep=2pt]
  \item \textbf{Signal ingestion.} Observe realized sales velocity, queue conditions, and any
  relevant operational indicators available at time $t$.
  \item \textbf{Context evaluation.} Determine active contextual signals, including \LTO{}
  activation (\ltoOn{}) and phase (\ltoPhase{}), along with any binding capacity or supply
  constraints.
  \item \textbf{Readiness adjustment.} Apply the \RAL{} to modulate forecast trust based on the
  current context, yielding a readiness posture appropriate to the risk environment.
  \item \textbf{Forecast evaluation and selection.} Evaluate candidate forecasts under
  readiness-adjusted asymmetric loss (e.g., \CWSL{}), and select the forecast that minimizes
  expected operational loss.
  \item \textbf{Production action.} Translate the selected forecast into a production target
  \qprod{}, subject to feasibility and capacity constraints.
\end{enumerate}

This structure ensures that contextual risk influences decisions before production commitments are
made, while preserving a clear separation between observation, evaluation, and action.

\subsection{Role of realized demand}
\label{subsec:realized-demand}

Realized demand plays a central role in the intraday loop. As sales accumulate, uncertainty about
actual demand resolves naturally, reducing reliance on prior beliefs. Electric Barometer leverages
this resolution by allowing realized signals to dominate decision-making as the day progresses,
particularly during sustained \LTO{} periods.

Crucially, realized demand is not used to retroactively justify earlier assumptions about promotional
lift. Instead, it serves as a corrective signal that interacts with readiness and loss asymmetry to
update production behavior incrementally. This avoids abrupt shifts and reduces the risk of
overreaction to early noise.

\subsection{Partial reversibility and pacing}
\label{subsec:reversibility}

Intraday production decisions are partially reversible. Over-production can sometimes be mitigated
through holding buffers, reallocation across channels, or controlled waste, while under-production
during peak windows is often irreversible. Electric Barometer accounts for this asymmetry by
pacing production adjustments over time rather than committing to large, belief-driven jumps.

During \LTO{} launch phases, readiness-adjusted selection favors conservative ramps that protect
against shortfall. As empirical demand signals stabilize, production targets can be tightened
gradually, reflecting improved confidence without abandoning caution.

\subsection{Failure modes avoided}
\label{subsec:failure-modes}

The intraday control loop is explicitly designed to avoid common failure modes observed during
promotional periods:
\begin{enumerate}[leftmargin=*, itemsep=2pt]
  \item \textbf{Belief lock-in.} Production does not hinge on a single promotional lift estimate that
  must be defended throughout the day.
  \item \textbf{Overreaction to noise.} Early demand spikes do not immediately trigger aggressive
  scaling absent readiness-aware evaluation.
  \item \textbf{Delayed correction.} Conservative bias during high-risk windows reduces the
  likelihood of late-day stockouts that cannot be recovered.
\end{enumerate}

By structuring decisions around readiness and loss rather than point accuracy, Electric Barometer
maintains stability even when forecasts miss materially.

\subsection{Operational interpretability}
\label{subsec:intraday-interpretability}

An important property of the intraday control loop is interpretability. Each production decision can
be explained in terms of observable signals, contextual readiness, and explicit cost asymmetry.
Operators are not asked to trust opaque promotional models; instead, they are guided by a system
that behaves predictably under known risk conditions.

This interpretability supports operational adoption and governance, enabling teams to understand
why production was increased or constrained during \LTO{} periods. In the next section, we discuss
how these decisions are governed, monitored, and audited to ensure consistent behavior across
promotions and time.