% ----------------------------------------------------------
% LIMITATIONS AND NON-GOALS
% ----------------------------------------------------------

\section{Limitations and Non-Goals}
\label{sec:limitations}

Electric Barometer’s treatment of \LTOs{} is intentionally constrained. The framework prioritizes
operational robustness, interpretability, and governance over predictive expressiveness. This
section delineates the boundaries of the proposed approach and consolidates its principal
limitations and non-goals, in order to prevent misinterpretation or inappropriate extension.

\subsection{Not a promotional demand forecasting framework}
\label{subsec:not-demand-forecasting}

Electric Barometer does not attempt to forecast \LTO{} demand uplift, hype intensity, or marketing
effectiveness. It does not ingest promotional spend, creative attributes, or campaign metadata as
drivers of intraday production decisions. While such modeling may be appropriate for strategic
planning, procurement, or menu design, it lies outside the scope of the readiness-centric control
problem addressed here.

This exclusion is deliberate. Incorporating speculative promotional predictors into the intraday
control loop would couple production decisions to assumptions that are difficult to validate in real
time and prone to failure under heterogeneous execution.

\subsection{Not an optimization of forecast accuracy}
\label{subsec:not-accuracy-optimization}

The framework does not seek to maximize forecast accuracy during \LTO{} periods. It explicitly
accepts that forecast accuracy is structurally degraded under planned regime shifts, and treats this
degradation as a design premise rather than a deficiency to be corrected.

Accordingly, Electric Barometer evaluates forecasts in terms of readiness and operational loss,
rather than symmetric error magnitude. Forecasts that are numerically less accurate may therefore
be preferred if they yield safer outcomes under asymmetric risk. Accuracy metrics remain useful as
diagnostic tools, but they do not govern production behavior.

\subsection{Dependence on governed assumptions}
\label{subsec:assumption-dependence}

Readiness adjustment and asymmetric loss evaluation depend on explicit, governed assumptions
about risk posture and operational tolerance. These assumptions must be specified, reviewed, and
maintained deliberately. Poorly chosen or implicit assumptions can result in behavior that is either
overly conservative or insufficiently protective.

This dependence reflects an unavoidable property of operational systems: policy choices must be
encoded somewhere. Electric Barometer’s contribution is to surface these choices as inspectable and
auditable artifacts, rather than embedding them implicitly in model structure or training objectives.

\subsection{Limits of reversibility}
\label{subsec:reversibility-limits}

The intraday control loop assumes partial reversibility of production decisions. While this assumption
holds in many QSR contexts, it may not apply uniformly across all products, preparation methods, or
operational constraints. Items with extremely short shelf lives or rigid preparation requirements may
necessitate simplified control policies or more conservative readiness postures.

Electric Barometer does not eliminate irreversibility; it operates within it. Where reversibility is
limited, readiness degradation and loss asymmetry should be calibrated accordingly.

\subsection{Summary of non-goals}
\label{subsec:non-goals-summary}

For clarity, the principal non-goals of this work are summarized as follows:
\begin{enumerate}[leftmargin=*, itemsep=2pt]
  \item We do not provide a method for estimating \LTO{} demand uplift or promotional response.
  \item We do not propose training forecasting models to optimize readiness or loss metrics.
  \item We do not collapse supply planning and intraday production control into a single predictive
  system.
  \item We do not claim that readiness-centric control improves forecast accuracy.
\end{enumerate}

These constraints define the intended scope of Electric Barometer’s \LTO{} handling. Within this
scope, the framework provides a disciplined and auditable approach to managing production risk
during planned instability. The following section concludes by summarizing the broader implications
of this approach for operational forecasting and production management.