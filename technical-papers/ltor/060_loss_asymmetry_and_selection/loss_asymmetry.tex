% ----------------------------------------------------------
% LOSS ASYMMETRY AND FORECAST SELECTION
% ----------------------------------------------------------

\section{Loss Asymmetry and Forecast Selection}
\label{sec:asymmetry}

Forecast selection in operational environments should reflect not only expected deviation from
realized demand, but the asymmetric consequences of being wrong. During \LTOs{}, these asymmetries
intensify: the cost of under-producing during peak promotional windows often exceeds the cost of
over-producing by a wide margin. Electric Barometer encodes this reality explicitly through
asymmetry-aware evaluation and selection, ensuring that production behavior aligns with operational
risk rather than numerical accuracy.

\subsection{Asymmetric loss as an operational primitive}
\label{subsec:asymmetric-loss}

Traditional accuracy metrics implicitly assume symmetry: over- and under-forecasting are penalized
equally. This assumption is rarely valid in production management, and it is especially brittle
during \LTO{} periods. Stockouts during promotional peaks can result in immediate revenue loss,
queue buildup, brand damage, and guest dissatisfaction, while moderate over-production may be
absorbed through holding buffers, reallocation, or controlled waste. Asymmetric loss has long been
recognized as essential for decision-aligned learning and evaluation \citep{elkan2001}.

Electric Barometer addresses this imbalance using asymmetric loss primitives, most notably
\CWSL{}. By weighting under-forecasting and over-forecasting differently via cost parameters
\cu{} and \co{}, \CWSL{} translates forecast error into an estimate of expected operational loss
rather than abstract deviation. This framing ensures that forecast evaluation is grounded in
decision consequence.

\subsection{Readiness-adjusted loss evaluation}
\label{subsec:readiness-adjusted-loss}

The influence of asymmetric loss is modulated by readiness. When the \RAL{} indicates degraded
readiness—such as during active \LTO{} launch phases—loss asymmetry becomes more decisive in
forecast selection. In effect, readiness adjustment amplifies the penalty associated with operationally
dangerous errors without altering the forecasts themselves.

Formally, readiness-adjusted evaluation can be viewed as applying asymmetric loss under a
context-dependent risk posture. Forecasts that appear similar under symmetric accuracy metrics
may diverge sharply once evaluated under \CWSL{} in a low-readiness environment. This divergence
guides Electric Barometer toward forecasts that better protect service and throughput during
high-risk periods.

\subsection{Selection over aggregation}
\label{subsec:selection-over-aggregation}

A key design choice in Electric Barometer is the emphasis on selection rather than aggregation.
Instead of averaging forecasts or blending models to improve point accuracy, the system evaluates
candidate forecasts under readiness-adjusted loss and selects the option that minimizes expected
operational harm.

This approach is particularly important during \LTO{} regimes, where averaging can dilute
protective bias and obscure downside risk. Selection preserves interpretability—each decision can
be traced to a specific forecast and evaluation context—and avoids overconfidence induced by
ensemble smoothing.

\subsection{Implications for production behavior}
\label{subsec:production-implications}

Asymmetry-aware selection produces concrete and intuitive effects on production behavior during
\LTO{} periods. Forecasts that err conservatively on the side of over-preparation are favored during
high-risk windows, increasing buffer availability and reducing the likelihood of service failure.
As uncertainty resolves and readiness recovers, selection pressure relaxes, allowing production to
tighten without abrupt regime changes.

Importantly, these effects arise without explicit demand uplift modeling. Production quantities
increase not because the system believes demand will be higher, but because the cost of being short
is higher. This distinction preserves robustness when promotional outcomes diverge from
expectations.

\subsection{Explainability and governance}
\label{subsec:asymmetry-governance}

Encoding asymmetry at the evaluation and selection stage improves explainability and governance.
Decisions can be justified in terms of explicit cost tradeoffs and readiness context rather than opaque
model internals. Postmortem analysis can distinguish between forecast misspecification and
appropriate risk posture, clarifying whether outcomes reflect modeling gaps or intentional
conservatism.

In the next section, we integrate readiness adjustment and asymmetric selection into an intraday
control loop, demonstrating how Electric Barometer operationalizes these principles in real-time
production management under \LTO{} conditions.