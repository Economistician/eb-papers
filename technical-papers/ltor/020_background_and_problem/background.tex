% ----------------------------------------------------------
% BACKGROUND AND PROBLEM STATEMENT
% ----------------------------------------------------------

\section{Background and Problem Statement}
\label{sec:background-problem}

Forecasting systems in QSR environments are traditionally evaluated and tuned under an implicit
assumption of regime stability. Historical demand patterns, seasonal structure, and recent trends
are treated as informative priors for near-term behavior, and forecast quality is primarily assessed
through symmetric accuracy metrics. Within this paradigm, deviations between forecast and
realized demand are interpreted as modeling failures, to be corrected through additional features,
segmentation, or algorithmic complexity.

Limited-Time Offers (\LTOs{}) violate these assumptions by design. They introduce deliberate,
time-bounded perturbations to the demand-generating process, often accompanied by marketing
activity, menu novelty, and operational constraints. The resulting demand behavior exhibits launch
effects, rapid nonstationarity, substitution across menu items, and heterogeneous execution across
stores and dayparts. Observationally, these effects manifest as abrupt shifts in level, variance, and
error asymmetry relative to steady-state demand.

\subsection{The classical response: demand uplift modeling}
\label{subsec:classical-response}

The dominant analytical response to \LTOs{} has been to treat them as a forecasting problem. In this
view, the task is to estimate incremental demand attributable to the promotion by incorporating
additional explanatory variables such as marketing spend, promotional calendars, product novelty
indicators, or historical analogs. Forecasts are then adjusted upward to reflect expected lift, and
production plans are derived directly from these adjusted predictions.

While this approach is conceptually appealing, it embeds several fragile assumptions. It presumes
that promotional lift is both estimable and sufficiently stable to be operationalized at fine temporal
and spatial granularity. In practice, realized lift depends on execution quality, local conditions,
substitution effects, and supply availability, many of which are unobserved or only partially
observable at decision time. As a result, demand estimates are treated as authoritative inputs to
production control despite being derived from highly uncertain premises.

These assumptions are especially problematic in intraday production contexts, where decisions are
made rapidly, partially reversible, and continuously updated as demand unfolds. In such settings,
errors in promotional uplift estimation propagate directly into waste, stockouts, and service
degradation.

\subsection{Accuracy as a misaligned objective}
\label{subsec:accuracy-misalignment}

A deeper issue lies in the objective function itself. Classical forecasting workflows equate forecast
quality with numerical accuracy, implicitly assuming that smaller errors correspond to better
operational outcomes. As noted by \citet{gneiting2011}, forecast evaluation is inseparable from the
decision context in which forecasts are used. Under \LTO{} conditions, this equivalence breaks down.
Forecast errors become asymmetric in consequence: under-forecasting during peak promotional
windows may result in immediate stockouts and lost transactions, while comparable over-forecasting
may incur manageable waste or reallocation costs.

From an operational perspective, the relevant question is therefore not whether a forecast minimizes
expected error magnitude, but whether it minimizes expected operational loss given the current risk
environment. During \LTO{} periods, forecast accuracy is structurally degraded, while the cost of
error is simultaneously amplified and skewed. Treating these regimes as standard forecasting
problems forces systems to optimize toward metrics that are misaligned with the true decision
stakes.

\subsection{Problem statement}
\label{subsec:problem-statement}

The core problem addressed in this paper is not how to predict \LTO{} demand more accurately, but
how to govern production decisions when forecast reliability is known \emph{ex ante} to be reduced.
Specifically, we seek an approach that:
\begin{enumerate}[leftmargin=*, itemsep=2pt]
  \item acknowledges \LTOs{} as planned sources of instability rather than anomalies to be smoothed
  away,
  \item decouples intraday production control from speculative promotional lift estimates,
  \item incorporates asymmetric operational consequences directly into forecast evaluation and
  selection, and
  \item remains explainable, auditable, and robust under forecast misspecification.
\end{enumerate}

Electric Barometer addresses this problem by reframing \LTO{} handling as a readiness and risk
governance challenge. Rather than attempting to restore predictive accuracy through increasingly
complex demand models, it modulates forecast trust and decision behavior in response to known
regime shifts. The following sections formalize this perspective and describe how \LTOs{} are
integrated into a readiness-centric production management architecture.