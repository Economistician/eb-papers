\begin{abstract}
Limited-Time Offers (LTOs) are designed to perturb demand, yet they routinely degrade
forecast reliability in operational systems by introducing regime shifts, launch effects,
substitution dynamics, and execution variability. In such environments, the dominant
failure mode is not a lack of forecasting sophistication but the operational coupling of
production decisions to brittle assumptions about promotional lift. This paper argues
that LTO handling should be treated as a readiness and risk-governance problem rather
than a demand-modeling problem. We present a readiness-centric approach grounded in
the Electric Barometer philosophy: (i) maintain strict separation between supply planning
forecasts and intraday production-control decisions; (ii) treat LTOs as exogenous readiness
shocks that reduce forecast trust and increase variance; and (iii) shape operational behavior
through governed asymmetry in loss (e.g., shortfall-avoidance and cost-weighted penalties)
rather than speculative uplift coefficients. We outline an implementable architecture for
LTO-aware production management that remains robust under forecast misspecification,
supports postmortem auditability, and improves decision alignment during planned
instability. The result is a framework that can increase preparedness during hyped launches
without requiring accurate prediction of hype itself.
\end{abstract}