% ----------------------------------------------------------
% GOVERNANCE AND REPORTING
% ----------------------------------------------------------

\section{Governance and Reporting}
\label{sec:governance}

Because \LTO{} periods deliberately alter risk posture and decision behavior, their integration into
production management must be governed explicitly. Without clear governance, readiness adjustment
and asymmetric selection risk being misinterpreted as ad hoc conservatism or model instability.
Electric Barometer therefore treats all readiness- and loss-related assumptions as first-class,
auditable artifacts rather than implicit system behavior.

\subsection{Governed contextual signals}
\label{subsec:governed-context}

All contextual signals that influence readiness must be declared, versioned, and inspectable. In the
case of \LTOs{}, this includes activation indicators (\ltoOn{}), phase labels (\ltoPhase{}), and any
rules governing their temporal extent. These signals define when and how readiness is degraded and
must be managed through the same governance processes as other operational parameters.

By externalizing \LTO{} context, Electric Barometer ensures that promotional effects are not
silently embedded in model behavior. This separation enables consistent application across stores,
dayparts, and promotions, and simplifies review when readiness assumptions change.

\subsection{Readiness policies and lifecycle}
\label{subsec:readiness-lifecycle}

Readiness adjustment policies should have an explicit lifecycle. The degree of readiness degradation
associated with \LTO{} launch, sustain, and wind-down phases must be specified in advance and
revisited periodically. Changes to these policies redefine operational risk posture and therefore
require review and approval. This emphasis on explicit assumptions and robustness over point
optimality aligns with established principles of sensitivity analysis and model governance
\citep{saltelli2008}.

Importantly, readiness policies are not tuned opportunistically in response to short-term outcomes.
They are calibrated based on historical behavior, operational tolerance, and governance objectives,
and then held fixed during evaluation and execution. This separation preserves interpretability and
prevents post hoc rationalization of performance.

\subsection{Loss asymmetry governance}
\label{subsec:loss-governance}

Asymmetric loss parameters (\cu{}, \co{}) encode explicit tradeoffs between service protection and
waste. During \LTO{} periods, these parameters often differ from steady-state values, reflecting the
heightened cost of shortfall. Because these weights directly influence production behavior, they must
be governed with the same rigor as readiness policies.

Governance artifacts should record the chosen cost ratios, their justification, and the contexts in
which they apply. This documentation enables postmortem analysis to distinguish between deliberate
risk posture and unintended consequences, and ensures that changes in loss asymmetry are deliberate
rather than emergent.

\subsection{Reporting and diagnostics}
\label{subsec:reporting-diagnostics}

Effective governance requires transparent reporting. For each \LTO{}, reporting should include:
\begin{enumerate}[leftmargin=*, itemsep=2pt]
  \item the period of \LTO{} activation and associated readiness posture,
  \item the loss asymmetry parameters applied during the promotion,
  \item summary diagnostics of forecast selection behavior under readiness adjustment, and
  \item observed service and waste outcomes relative to steady-state baselines.
\end{enumerate}

These diagnostics allow stakeholders to assess whether readiness adjustment and asymmetric
selection behaved as intended, independent of whether forecasts were numerically accurate.

\subsection{Postmortems and continuous improvement}
\label{subsec:postmortems}

Postmortem analysis under Electric Barometer focuses on decision quality rather than forecast error
alone. When outcomes deviate from expectations during \LTO{} periods, the primary questions are:
\begin{itemize}[leftmargin=*, itemsep=2pt]
  \item Were readiness policies appropriate given the risk environment?
  \item Were loss asymmetries aligned with operational priorities?
  \item Did the intraday control loop respond coherently to realized demand signals?
\end{itemize}

This framing avoids the reflexive conclusion that forecasting models must be improved whenever
outcomes are unfavorable. Instead, it supports targeted refinement of readiness and governance
assumptions, reinforcing the principle that operational robustness, not numerical accuracy, is the
ultimate objective.

In the next section, we examine common deployment patterns and failure modes observed when
managing \LTO{} periods, illustrating how the Electric Barometer approach differs from traditional
forecast-centric systems.