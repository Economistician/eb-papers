\section{Future Research Directions}

The Forecast Readiness Framework opens several avenues for future research at the
intersection of forecasting, operations, and decision support. One natural extension
concerns the specification and governance of asymmetric penalty structures used within
the \CWSL\ component. While the framework treats penalty ratios as externally specified
and subject to calibration and sensitivity analysis, future work may explore structured,
data-informed approaches for eliciting operational cost asymmetry from observed
decisions, service outcomes, or recovery behavior, while preserving auditability and
interpretability.

A second direction involves extending readiness-based evaluation to probabilistic
forecasting settings. The current framework emphasizes realized demand and point
forecasts, reflecting a focus on execution outcomes rather than uncertainty
representation. Incorporating predictive distributions, coverage guarantees, or tail-risk
diagnostics into readiness assessment may further enhance deployment decisions in
contexts where probabilistic forecasts are available and operational policies depend on
risk tolerance.

Future research may also examine how readiness diagnostics interact with model
development and selection workflows. Although the \FRF\ is not intended as an
optimization objective, its components may inform training constraints, post-processing
adjustments, or ensemble strategies that balance accuracy, reliability, and stability.
Understanding how readiness-aware evaluation influences model behavior without
collapsing governance and optimization remains an important area for investigation.

Finally, the framework may be extended to more complex operational settings involving
dynamic feedback, multi-stage decisions, or interactions across entities and locations.
Studying how readiness diagnostics aggregate, propagate, or stabilize in such coupled
systems could broaden the applicability of the \FRF\ to networked and large-scale
operational environments.

Together, these directions highlight the potential for readiness-based evaluation to inform
both the theory and practice of forecasting in asymmetric operational contexts, providing a
foundation for continued methodological development and applied research.
