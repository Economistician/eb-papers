\section{Conclusion}

Forecasts used in operational decision-making are ultimately judged not by numerical accuracy
alone, but by their ability to support reliable execution under asymmetric cost. In many applied
settings, traditional symmetric accuracy metrics fail to capture the directional, temporal, and
economic structure of forecast error that determines operational success or failure. As a result,
forecasts that appear accurate by conventional measures may still impose unacceptable service
risk when deployed.

This paper introduced the \emph{Forecast Readiness Framework} (\FRF) as a structured approach
for evaluating forecast performance in such environments. By decomposing readiness into
complementary dimensions—service reliability, failure severity, tolerance stability, and economic
consequence—the framework provides a diagnostic lens that extends beyond aggregate accuracy.
Within this structure, Cost-Weighted Service Loss (\CWSL) serves as the economic axis, translating
directional forecast error into cost-aligned impact, while supporting diagnostics reveal how error
patterns affect execution.

The framework further incorporates a composite readiness signal, the \FRS, to support
deployment-level decisions, monitoring, and governance. Rather than collapsing evaluation into a
single loss function, the \FRF\ preserves interpretability while enabling principled trade-offs between
efficiency and reliability. Through an illustrative example, the paper demonstrated how forecasts
with comparable symmetric accuracy can exhibit sharply different readiness profiles, underscoring
the limitations of conventional evaluation practices.

By reframing forecast evaluation around readiness for deployment under explicit and governed
assumptions, the \FRF\ provides a practical and generalizable basis for assessing forecasting
systems in asymmetric operational contexts. Sensitivity analysis, calibration procedures, and
policy-based constraints ensure that readiness assessments are transparent, auditable, and stable
across models and time periods. More broadly, the framework encourages a shift in forecast
evaluation from measuring how close predictions are to realized demand toward understanding
how forecast error shapes operational outcomes in practice.
