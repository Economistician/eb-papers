\section{Illustrative Example}

To illustrate the diagnostic value of the \FRF, we consider a stylized example in which two
forecasts exhibit similar symmetric accuracy but differ substantially in their readiness for
operational deployment. The example is intentionally simple and is not intended as an empirical
evaluation; rather, it demonstrates how different error structures produce divergent operational
outcomes that are obscured by aggregate accuracy metrics.

Consider a single item evaluated over a short decision horizon consisting of multiple intervals.
Using the notation introduced in Section~2, realized demand is fixed across both scenarios.
Two competing forecasts, denoted Forecast~A and Forecast~B, are constructed such that they
achieve comparable performance according to a symmetric accuracy metric (e.g., MAE).
Under symmetric evaluation, the two forecasts would therefore be regarded as equivalent.

Despite this equivalence, the structure of forecast error differs markedly. Forecast~A exhibits
small but frequent deviations that are predominantly within the operational tolerance band.
Occasional shortfalls occur, but they are shallow and distributed across low-impact intervals.
Forecast~B, by contrast, matches demand closely in most intervals but produces a small number
of pronounced under-forecasting events during high-impact periods. Although these deep
shortfalls are infrequent, they impose disproportionate operational burden when they occur.

When evaluated under the \FRF, the two forecasts diverge clearly. Forecast~A achieves a higher
\NSL, indicating that it avoids shortfalls more consistently. Conditional on shortfalls occurring,
its \UD\ is lower, reflecting limited failure severity. Its deviations also fall within tolerance
more frequently, yielding a higher \HRtau. As a result, the cost-weighted impact of error captured
by \CWSL\ is modest, even when asymmetric penalties are applied. Taken together, these
properties produce a higher \FRS, indicating stronger readiness for deployment.

Forecast~B exhibits the opposite pattern. While its symmetric accuracy is comparable, its \NSL\
is lower due to the presence of critical shortfall events. Its \UD\ is substantially higher, reflecting
the depth of those failures, and its deviations more frequently exceed tolerance bounds. Under
asymmetric penalties, these structural differences translate into a significantly higher \CWSL.
The resulting \FRS\ reflects diminished readiness, despite otherwise favorable average accuracy.

Table~\ref{tab:frf_example} summarizes the contrasting readiness profiles of the two forecasts.
Although symmetric accuracy metrics suggest similar performance, the FRF diagnostics reveal
substantial divergence across reliability, severity, tolerance stability, and economic consequence.

\begin{table}[t]
\centering
\caption{Illustrative comparison of two forecasts with similar symmetric accuracy but different
readiness profiles under the Forecast Readiness Framework (FRF).}
\label{tab:frf_example}
\begin{tabular}{lcc}
\toprule
\textbf{Metric} & \textbf{Forecast A} & \textbf{Forecast B} \\
\midrule
Mean Absolute Error (MAE) & Similar & Similar \\
Service Reliability (\NSL) & High & Low \\
Failure Severity (\UD) & Low & High \\
Tolerance Stability (\HRtau) & High & Moderate \\
Cost-Weighted Service Loss (\CWSL) & Low & High \\
Composite Readiness (\FRS) & High & Low \\
\bottomrule
\end{tabular}
\end{table}

This example highlights a key limitation of symmetric evaluation: forecasts that appear equivalent
by aggregate error measures may impose radically different operational risk. By decomposing
readiness into reliability, severity, tolerance stability, and economic consequence, the \FRF\
reveals failure modes that symmetric metrics systematically obscure. In doing so, the framework
provides a more faithful assessment of whether a forecast is suitable for operational deployment
in asymmetric environments.