\section{Composite Readiness and Deployment Decisions}

The diagnostic components of the \FRF\ provide complementary perspectives on forecast error,
but operational deployment decisions often require a single, interpretable signal that summarizes
overall readiness. Practitioners must determine whether a forecast is sufficiently reliable to
support execution, compare competing models, and monitor changes in readiness over time.
Relying on individual diagnostics in isolation can obscure trade-offs among reliability, severity,
stability, and economic consequence. For this reason, the \FRF\ includes a composite measure,
the \emph{Forecast Readiness Score} (\FRS), designed to support deployment-level evaluation
and governance decisions.

The \FRS\ synthesizes information from the diagnostic components of the framework, anchoring
the composite on the economic axis provided by \CWSL\ while incorporating complementary
signals of service reliability, failure severity, and tolerance stability. Conceptually, the \FRS\
reflects the extent to which a forecast both limits cost-weighted operational loss and avoids
readiness-critical failure modes. In this sense, the \FRS\ does not replace individual diagnostics;
rather, it aggregates them in a manner that preserves their operational interpretation and
supports consistent comparison across models, entities, and time periods.

A defining characteristic of the \FRS\ is its asymmetry. Because \CWSL\ encodes the directional
cost of forecast error, the composite score inherits sensitivity to under-forecasting risk.
Reliability-related diagnostics such as \NSL\ and severity-related measures such as \UD\ further
modulate the score by penalizing forecasts that achieve low cost-weighted loss only by
concentrating risk in infrequent but severe shortfalls. Stability relative to tolerance, as captured
by \HRtau, ensures that forecasts whose deviations are frequently absorbed by the operational
system are distinguished from those that require frequent intervention.

The \FRS\ is intended to be interpreted as an evaluative readiness indicator rather than as a loss
function to be minimized or an objective to be optimized. Higher values correspond to forecasts
that are more consistently aligned with operational requirements under explicitly governed cost
and tolerance assumptions. Because the \FRS\ aggregates multiple diagnostics, its value should
be interpreted alongside the underlying components when diagnosing failure modes or guiding
model refinement. A change in the \FRS\ may reflect deterioration in service reliability, increased
failure severity, reduced tolerance stability, or a shift in the economic alignment of forecast error.

In governance and deployment settings, the \FRS\ may be used to define acceptance thresholds,
trigger review or retraining processes, and communicate readiness to stakeholders who require a
concise but meaningful summary of forecast performance. Where appropriate, composite
readiness signals may be passed through post-evaluation adjustment layers that enforce feasibility
or policy constraints without altering the underlying forecast or diagnostic measurements. By
embedding economic consequence within a broader diagnostic and governance structure, the
\FRS\ enables decision-makers to balance service reliability and efficiency in a principled,
transparent, and auditable manner.
