\section{Operational Asymmetry and Forecast Readiness}

Many operational decision environments exhibit pronounced asymmetry in the consequences of
forecast error. In such settings, under-forecasting and over-forecasting do not represent symmetric
deviations from a neutral baseline; rather, they trigger qualitatively different operational
responses. Shortfalls may lead to service failures, lost throughput, expedited recovery actions,
or cascading disruptions, while excess capacity or production is often absorbed at comparatively
lower cost. This directional imbalance implies that the operational impact of forecast error depends
not only on magnitude, but also on direction, timing, and frequency.

Operational asymmetry also interacts with the temporal structure of decision-making. Forecasts
are typically used at fixed intervals to support execution under limited flexibility. Errors that occur
during high-impact intervals or periods of constrained recovery may impose disproportionate
burden, even if they are infrequent. Conversely, small deviations during low-impact periods may
be effectively neutralized by buffering mechanisms or operational slack. As a result, averaging
forecast error across time can obscure the specific conditions under which forecasts fail to support
execution.

These characteristics challenge the adequacy of accuracy-based evaluation alone. Symmetric
metrics aggregate deviations without regard to direction or consequence, implicitly treating all
errors as interchangeable. Even when asymmetric loss is considered, a single aggregate measure
cannot distinguish between forecasts that fail frequently but mildly and those that fail rarely but
severely. From an operational perspective, these error structures represent fundamentally
different risk profiles and demand different responses.

In practice, operational systems exhibit varying degrees of tolerance to forecast error. Some
deviations can be absorbed without material degradation, while others exceed the system’s
capacity to adjust. The distinction between tolerable and readiness-critical error is therefore central
to deployment decisions. A forecast may be accurate on average yet consistently violate tolerance
bounds during critical intervals, rendering it unsuitable for use despite favorable aggregate
performance.

These considerations motivate the concept of \emph{forecast readiness}. Readiness emphasizes
whether a forecast is suitable for deployment given the asymmetric and constrained nature of
operational execution. Rather than asking how close forecasts are to realized demand on average,
readiness asks whether forecasts reliably support service objectives, limit exposure to severe
failures, remain stable within tolerance, and align with the economic consequences of error. This
shift in perspective provides the foundation for the multi-dimensional evaluation framework
introduced in the following section.