\section{Implications for Forecast Evaluation and Governance}

The \FRF\ has several implications for how forecasts are evaluated, selected, and governed in
operational environments characterized by asymmetric error cost. By shifting the focus from
numerical accuracy to readiness for deployment under explicit assumptions, the framework
reframes forecast evaluation as an operational risk management problem rather than a purely
statistical exercise.

\subsection{Implications for Model Evaluation and Selection}

Traditional model evaluation procedures often rely on a small set of symmetric accuracy metrics
to compare competing forecasting approaches. Within the \FRF, such comparisons are
necessarily incomplete. Forecasts that perform similarly under symmetric metrics may differ
substantially in their service reliability, failure severity, tolerance stability, or economic exposure,
leading to materially different operational outcomes.

Evaluating models under the \FRF\ encourages practitioners to examine error structure explicitly
and under governed cost and tolerance assumptions. Rather than selecting models based solely
on aggregate accuracy, decision-makers can assess whether a forecast’s deviations align with
operational priorities, such as avoiding deep shortfalls, maintaining stability within tolerance
bounds, or limiting asymmetric economic exposure. This perspective supports more informed and
defensible trade-offs between efficiency and reliability in environments where under-forecasting
risk dominates.

\subsection{Implications for Monitoring and Change Detection}

Because the diagnostic components of the \FRF\ isolate distinct failure modes, the framework
naturally supports ongoing monitoring of deployed forecasting systems. Changes in readiness
may arise from increased frequency of shortfalls, greater failure severity, reduced tolerance
stability, or shifts in the economic alignment of forecast error under the stated evaluation contract.
Monitoring these dimensions separately enables earlier and more interpretable detection of
degradation than aggregate accuracy metrics alone.

At the composite level, trends in the \FRS\ can serve as a concise indicator of deployment
suitability over time. However, interpretation of changes in the composite score should be
informed by the underlying diagnostics and by the stability of the governing assumptions
themselves. A declining \FRS\ may reflect different operational risks depending on whether it is
driven by reliability, severity, tolerance stability, or cost-weighted loss, each of which may warrant
a distinct response, recalibration, or policy intervention.

\subsection{Implications for Governance and Decision Support}

In organizational settings, forecast evaluation often plays a role in governance processes such as
model approval, retraining, escalation, and audit. The \FRF\ provides a structured and auditable
basis for defining readiness criteria through explicit evaluation contracts, including cost
asymmetry, tolerance thresholds, admissibility rules, and acceptance bands.

Acceptance thresholds may be specified for individual diagnostics or for the \FRS, enabling
consistent and defensible deployment decisions across teams, models, and business units.
Because these criteria are explicit and governed, readiness decisions can be reproduced,
reviewed, and revised without conflating evaluation with optimization or post hoc adjustment.

By embedding asymmetric cost considerations and tolerance assumptions directly into the
evaluation framework, the \FRF\ facilitates clearer communication between technical and
operational stakeholders. Rather than debating abstract accuracy improvements, discussions can
focus on service reliability, exposure to shortage risk, and economic consequence under stated
assumptions. This alignment supports more effective decision support and reduces the likelihood
that technically sound forecasts are deployed in contexts for which they are not operationally
suited.

\subsection{Implications for Forecast Design and Use}

Finally, the \FRF\ has implications for how forecasts are designed and used. Awareness of
readiness criteria may influence modeling choices, encouraging approaches that trade marginal
gains in average accuracy for improvements in reliability, stability, or cost-aligned behavior.
At the same time, the framework clarifies that no single diagnostic is universally dominant;
readiness depends on the interaction between error structure, operational context, and governed
assumptions.

By treating readiness as a multi-dimensional and governed construct, the \FRF\ encourages
forecasting practices that are explicitly aligned with execution requirements and policy
constraints. This perspective supports more robust deployment decisions in asymmetric
environments, where the consequences of forecast error are unevenly distributed and operational
failure is costly.
