% ----------------------------------------------------------
% GOVERNANCE AND DIAGNOSTICS
% ----------------------------------------------------------
\section{Governance and Diagnostics}
\label{sec:governance}

Balance-based calibration introduces an explicit decision layer into forecast
evaluation.
While this improves interpretability and operational alignment, it also requires
clear governance standards to prevent misuse or silent drift.
This section outlines recommended diagnostics, reporting conventions, and
controls for responsible deployment.

\subsection{Required diagnostics}
\label{subsec:required_diagnostics}

Any calibrated cost ratio—global or entity-level—should be accompanied by a
minimum diagnostic set.
At a minimum, reported outputs should include:
\begin{itemize}[leftmargin=*]
    \item the selected ratio $\Rstar$ or $\Ri$,
    \item the grid $\Rgrid$ over which calibration was performed,
    \item realized underbuild and overbuild costs at the selected ratio,
    \item the absolute imbalance
    $|\UnderCost(\R) - \OverCost(\R)|$ at selection,
    \item the sample size $n$ used for calibration.
\end{itemize}

These diagnostics allow reviewers to assess whether the calibrated ratio reflects
a stable operating regime or a fragile balance driven by limited data.
Absent this context, calibrated ratios risk being interpreted as intrinsic
properties rather than empirical artifacts.

\subsection{Stability and sensitivity checks}
\label{subsec:stability_checks}

Calibration should be evaluated for stability across reasonable perturbations.
Recommended checks include:
\begin{itemize}[leftmargin=*]
    \item varying the ratio grid resolution or bounds,
    \item recalibrating on rolling or expanding historical windows,
    \item comparing calibration outcomes across adjacent time periods,
    \item evaluating the curvature of the imbalance function near $\Rstar$.
\end{itemize}

Flat imbalance regions or highly unstable selections indicate that the calibrated
ratio is weakly identified.
In such cases, governance rules may dictate defaulting to a global ratio or
reducing reliance on asymmetric evaluation.

\subsection{Separation of calibration and evaluation}
\label{subsec:calibration_vs_evaluation}

A central governance principle is the separation of calibration and evaluation.
Cost ratios should be calibrated on a designated calibration window and held
fixed when evaluating holdout or forward-looking data.
Recalibrating ratios on the same data used for evaluation introduces leakage and
undermines comparability over time.

This separation mirrors best practices in model training and validation and
supports consistent, auditable performance reporting.

\subsection{Operational controls and overrides}
\label{subsec:controls}

In production environments, calibrated ratios may serve as defaults rather than
hard constraints.
Governance frameworks should allow for explicit overrides, provided such
decisions are documented and justified.
For example, leadership may mandate a minimum asymmetry for safety-critical
operations regardless of empirical balance.

Crucially, overrides should be visible.
The combination of empirical calibration, documented adjustments, and persistent
diagnostics enables cost-asymmetric evaluation to function as a decision-support
tool rather than an opaque optimization artifact.

The final section summarizes how balance-based calibration integrates with
cost-weighted metrics to support readiness-oriented forecast evaluation.