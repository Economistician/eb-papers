% ----------------------------------------------------------
% RELATIONSHIP TO FORECAST READINESS
% ----------------------------------------------------------
\section{Relationship to Forecast Readiness}
\label{sec:relationship_to_readiness}

The calibration of asymmetric cost ratios is a foundational component of the
Forecast Readiness Framework (FRF).
Within FRF, readiness is defined not by statistical accuracy alone, but by the
degree to which forecasts align with the operational consequences of error.
Cost asymmetry is the mechanism through which these consequences are encoded.

Cost-Weighted Service Loss (\CWSL{}) evaluates forecast performance relative to a
specified cost ratio $\Rdef$, which determines the relative severity of
underbuild versus overbuild.
However, without a principled method for selecting $\R$, readiness assessment
becomes arbitrary and potentially inconsistent across models, entities, or time
periods.
Grid-based calibration addresses this gap by grounding $\R$ in observed error
behavior.

From a readiness perspective, balance-based calibration performs a normalization
step.
It identifies the point at which the realized operational burden of shortfall and
surplus is equalized under the assumed cost structure.
This does not assert that the organization is risk-neutral, but it establishes a
coherent baseline from which deliberate risk preferences can be applied.

Importantly, calibration occurs upstream of evaluation.
The calibrated ratio defines the evaluation environment in which forecasts are
judged, rather than adapting the metric post hoc to favor a particular model.
This separation preserves the interpretability and comparability of readiness
scores across candidates.

At the entity level, calibrated ratios expose structural differences in forecast
behavior.
Entities that consistently underforecast require higher shortfall penalties to
achieve balance, while entities prone to overforecasting exhibit lower effective
asymmetry.
These differences are not artifacts of the metric, but signals of readiness
heterogeneity that can inform segmentation, governance, and targeted model
intervention.

Within the broader FRF, cost-ratio calibration operates alongside other readiness
primitives such as tolerance calibration (HR@$\tau$), shortfall avoidance (NSL),
and loss depth (UD).
Together with tolerance calibration (HR@$\tau$), cost-ratio calibration defines
the \emph{evaluation envelope} within which readiness is assessed, establishing
the bounds of acceptable error magnitude and asymmetry.

In this sense, grid-based cost-ratio calibration is not a tuning convenience but a
readiness primitive.
It converts raw forecast error into an operationally meaningful evaluation
context, enabling consistent, explainable, and decision-aligned assessment across
the forecasting lifecycle.