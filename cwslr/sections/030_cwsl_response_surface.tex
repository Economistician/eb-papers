% ----------------------------------------------------------
% CWSL RESPONSE SURFACE
% ----------------------------------------------------------
\section{\CWSL{} as a Function of Cost Asymmetry}
\label{sec:response_surface}

Although \CWSL{} is typically reported at a single cost ratio $\R$, the metric is
more naturally understood as a \emph{response surface} over cost asymmetry.
For fixed forecast outcomes $(y_{it}, \hat{y}_{it})$ and a fixed overbuild cost
$\co$, \CWSL{} can be written explicitly as a function of $\R$ by setting
$\cu = \R \cdot \co$:
\[
    \CWSLR = 
    \frac{
        \sumit
        \left[
            \R \cdot \co \cdot \sit
            +
            \co \cdot \oit
        \right]
    }{
        \sumit \yit
    }.
\]
This representation makes clear that \CWSL{} is not a single scalar quantity, but
a family of values indexed by $\R$.

\subsection{Structure of the response surface}
\label{subsec:surface_structure}

For any fixed dataset, \CWSLR{} is a deterministic, piecewise-linear function of
$\R$.
The overbuild component is independent of $\R$, while the underbuild component
scales linearly with $\R$.
As a result, the shape of the response surface is governed entirely by the
distribution and magnitude of historical shortfalls relative to overbuild.

Several qualitative regimes are common in practice:
\begin{itemize}[leftmargin=*]
    \item \textbf{Underbuild-dominated regimes}, where \CWSLR{} increases rapidly
    with $\R$, indicating that shortfalls are frequent or deep.
    \item \textbf{Overbuild-dominated regimes}, where \CWSLR{} is relatively flat
    for moderate $\R$, reflecting excess capacity or conservative forecasting.
    \item \textbf{Mixed regimes}, where curvature changes across the range of
    $\R$, suggesting a tradeoff between shortfall and surplus that depends on
    cost assumptions.
\end{itemize}

These regimes are properties of the historical forecast errors, not of the
forecasting model itself. Consequently, interpreting \CWSL{} at a single
$\R$ without inspecting the surrounding response surface can obscure important
operational behavior.

\subsection{Why response surfaces matter}
\label{subsec:why_surface}

Viewing \CWSL{} as a response surface clarifies two practical points.
First, robustness to cost assumptions can be assessed directly by examining how
rapidly \CWSLR{} changes across plausible values of $\R$.
Flat regions indicate stability: evaluation conclusions are unlikely to change
materially if the true operational cost ratio differs modestly from the assumed
one.
Steep regions indicate fragility: small changes in $\R$ can lead to large changes
in measured performance or model ranking.

Second, the response-surface view separates \emph{evaluation} from
\emph{calibration}.
Rather than selecting $\R$ to minimize \CWSL{}, we can ask how underbuild and
overbuild costs balance across $\R$, and where that balance yields an
operationally interpretable reference point.
This perspective motivates grid-based sensitivity analysis as a first-class
diagnostic, rather than an auxiliary experiment.

The next section formalizes this idea by introducing a grid-based procedure for
evaluating \CWSL{} across a candidate set of cost ratios and extracting
sensitivity diagnostics from the resulting response curve.