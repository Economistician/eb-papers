% ----------------------------------------------------------
% USE CASES
% ----------------------------------------------------------
\section{Use Cases Across Operational Domains}

Underbuild Depth applies to environments in which the \emph{magnitude} of
underbuilding carries operational consequences. By conditioning on shortfall
intervals, \UD\ highlights how severe forecast misses are when they occur, rather
than how often they occur.

\subsection{Production and Manufacturing}

In production and manufacturing systems, deep shortfalls in component availability
or workload forecasts can halt assembly processes, create backlogs, or require
costly rescheduling. \UD\ helps distinguish between systems that occasionally
underbuild by large amounts and those that experience smaller, more manageable
misses.

\subsection{Retail and Replenishment}

In retail and distribution contexts, large stockout depths often translate directly
to lost sales, degraded service levels, or unstable replenishment cycles. \UD\
provides insight into whether demand surges, promotions, or seasonal effects are
being systematically underestimated in magnitude. In this sense, \UD\ is
conceptually related to inventory service-level and stockout severity measures
commonly discussed in production and inventory management
\citep{silver1998inventory}.

\subsection{Service and Staffing Operations}

Service environments such as call centers, hospitality operations, and field
service organizations rely on forecast-aligned staffing. Deep mismatches between
forecasted and realized workload can drive excessive queueing delays or service
failures. \UD\ quantifies the severity of these mismatches when staffing shortfalls
occur.

\subsection{Logistics and Transportation}

In logistics and transportation systems, capacity shortfalls during peak intervals
may lead to bottlenecks, rerouting, or delayed dispatch. \UD\ indicates how severe
these capacity mismatches are and helps diagnose structural underforecasting in
high-impact intervals.

\subsection{Energy and Utilities}

In energy and utility systems, deep load underforecasts may trigger costly reserve
activation or balancing actions. \UD\ can reveal patterns of underbuilding during
periods of high volatility or peak demand, supporting more resilient planning.