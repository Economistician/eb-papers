% ----------------------------------------------------------
% OPERATIONAL MOTIVATION
% ----------------------------------------------------------
\section{Operational Motivation}

In readiness-oriented forecasting environments, the operational impact of a miss
depends not only on whether a shortfall occurs, but on its depth. Shallow
underbuilding may be absorbed with limited disruption, while deep shortfalls can
trigger extended recovery periods, throughput degradation, or service failures.
Underbuild Depth captures this severity dimension directly by quantifying how far
realized demand exceeds forecasted values in intervals where underbuilding occurs.

Beyond measuring operational risk, \UD\ provides insight into the structural
behavior of a forecasting system. A process that underbuilds infrequently but by
large margins may pose greater operational risk than one that misses more often
but by smaller amounts. By conditioning on shortfall events, \UD\ isolates this
severity signal without dilution from intervals in which demand is fully covered.

In contexts where the cost of unmet demand increases with the size of the miss,
\UD\ serves as a focused diagnostic for evaluating forecast robustness and
identifying patterns of undercoverage. While commonly used magnitude-based
accuracy metrics summarize average deviation \citep{hyndman2006another}, they do
not distinguish between shallow and deep underbuilding events. When used
alongside frequency-based measures of shortfall occurrence, \UD\ contributes a
complementary perspective on forecast readiness by highlighting the magnitude of
operational insufficiency.