% ----------------------------------------------------------
% PROPERTIES AND BEHAVIOR
% ----------------------------------------------------------
\section{Properties and Behavior}

Underbuild Depth exhibits several mathematical and operational properties that
govern its interpretation and appropriate use.

\subsection{Non-Negativity}

By construction, shortfall magnitudes satisfy $s_{et} \ge 0$ for all intervals.
Consequently, \UD\ is always non-negative. A value of zero indicates that no
underbuilding occurred during the evaluation horizon.

\subsection{Conditionality}

\UD\ is computed only over intervals in which shortfalls occur. The metric
therefore captures the \emph{severity} of underbuilding rather than its frequency.
Two entities may exhibit identical \UD\ values despite having different numbers
of shortfall intervals.

\subsection{Sensitivity to Shortfall Magnitude}

\UD\ responds directly and linearly to the depth of shortfalls. Deeper misses
increase the metric proportionally, making \UD\ effective for identifying systems
that experience large, operationally consequential underbuilds.

\subsection{Sensitivity to Temporal Patterns}

Although \UD\ does not encode temporal ordering explicitly, clustered deep
shortfalls often correspond to periods of heightened operational strain. Elevated
\UD\ values may therefore indicate structural underforecasting during critical
intervals, even when shortfalls are relatively infrequent.

\subsection{Scale Dependence}

\UD\ is sensitive to the scale of the underlying demand. Systems forecasting
higher-volume entities may naturally exhibit larger potential shortfalls. For
comparisons across entities with differing demand levels, normalization or
supplemental metrics may be required.

\subsection{Relationship to Forecast Bias}

Downward-biased forecasting systems tend to produce deeper and more frequent
shortfalls, which increases \UD. The metric can therefore help surface
bias-related patterns that may not be evident from symmetric accuracy measures
that focus on average error magnitude.

\subsection{Robustness to Overbuilding}

Intervals in which forecasts exceed realized demand do not contribute to \UD.
As a result, the metric exclusively reflects the severity of underbuilding and is
insensitive to surplus magnitude. Evaluations concerned with the balance between
overbuilding and underbuilding require additional diagnostics.