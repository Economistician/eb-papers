% ----------------------------------------------------------
% GRID-BASED SENSITIVITY ANALYSIS FOR HR@τ
% ----------------------------------------------------------
\section{Grid-Based Sensitivity Analysis for HR@\(\tau\)}
\label{sec:sensitivity_analysis}

To operationalize the response-surface view of \HRtau{}, we evaluate the metric
across a finite, user-specified grid of tolerance values
\[
    \TauGrid = \{\tau_1, \tau_2, \ldots, \tau_K\}.
\]
Rather than fixing a single acceptability band a priori, this approach exposes
how readiness conclusions vary as the tolerance threshold changes, enabling
explicit robustness and stability checks.

This framing aligns with established principles of sensitivity analysis, which
emphasize understanding how conclusions depend on uncertain assumptions rather
than optimizing parameters in isolation \citep{saltelli2008}.
Here, the uncertain assumption is not a model parameter but the definition of
what constitutes an operationally acceptable error.

\subsection{Candidate tolerance grid construction}
\label{subsec:tau_grid_construction}

The tolerance grid $\TauGrid$ should span the range of error magnitudes considered
plausibly acceptable for the application domain.
In practice, $\TauGrid$ is often constructed from empirical quantiles of the
absolute error distribution, for example from the $0^{\text{th}}$ to
$99^{\text{th}}$ percentile, or as a fixed-width grid over a bounded interval.

The methods discussed here impose no restrictions on spacing or resolution.
Determinism is ensured as long as the grid is fixed and applied consistently.
As with cost-ratio grids, the tolerance grid should be treated as a governed
artifact, recorded alongside evaluation outputs to ensure reproducibility and
auditability of readiness assessments.

\subsection{Evaluating the HR@\(\tau\) sensitivity curve}
\label{subsec:hrtau_sensitivity_curve}

For each $\tau_k \in \TauGrid$, we compute $\HRtau(\tau_k)$ using identical data
and normalization.
The resulting set
\[
    \{(\tau_k, \HRtau(\tau_k))\}_{k=1}^{K}
\]
defines a discrete approximation to the continuous response surface described in
Section~\ref{sec:response_surface}.

The sensitivity curve provides substantially more information than reporting
\HRtau{} at a single tolerance.
In particular, it allows practitioners to:
\begin{itemize}[leftmargin=*]
    \item identify tolerance ranges where readiness conclusions are stable,
    \item detect sharp transitions where small changes in $\tauv$ yield large
          increases in hit rate,
    \item compare forecasting approaches under identical acceptability stress
          tests.
\end{itemize}

Because the curve is computed solely from historical residuals, it reflects
realized forecast behavior rather than model-specific assumptions.

\subsection{Stability diagnostics and interpretability}
\label{subsec:hrtau_stability}

Flat regions of the $\HRoftau{}$ curve indicate that increasing tolerance yields
little additional coverage, suggesting that readiness conclusions are insensitive
to the precise choice of $\tauv$ within that range.
Conversely, steep regions signal concentrations of error mass near the tolerance
boundary, where readiness claims are fragile and highly dependent on threshold
selection.

These diagnostics are particularly important when tolerance thresholds are used
to justify operational commitments.
If readiness appears acceptable only within a narrow tolerance window, the
evaluation should be treated with caution.
Grid-based sensitivity analysis makes such fragility explicit and supports
transparent communication of risk.

\subsection{Role of sensitivity analysis in tolerance calibration}
\label{subsec:tau_sensitivity_role}

Sensitivity analysis does not prescribe a single tolerance value.
Instead, it establishes the context within which calibration rules can be applied.
By inspecting the $\HRoftau{}$ surface, practitioners can ensure that any selected
reference tolerance lies in a region that is interpretable, stable, and aligned
with operational expectations.

In the following section, we introduce deterministic, data-driven calibration
rules for selecting reference tolerances from the sensitivity curve, formalizing
readiness standards without relying on ad hoc thresholding.