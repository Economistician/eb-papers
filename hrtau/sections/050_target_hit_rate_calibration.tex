% ----------------------------------------------------------
% TOLERANCE CALIBRATION RULES
% ----------------------------------------------------------
\section{Tolerance Calibration Rules}
\label{sec:tolerance_calibration}

Framing HR@\(\tau\) as a response surface clarifies that tolerance-based evaluation
does not yield a single, canonical value.
Practical deployment therefore requires a governed rule for selecting a
\emph{reference tolerance} at which readiness is assessed and reported.

This section presents three principled, data-driven tolerance calibration rules.
Each rule operates solely on historical residuals, imposes no model assumptions,
and is deterministic given inputs and a specified tolerance grid.
Importantly, these rules do not estimate a ``true'' tolerance; rather, they define
explicit, auditable conventions for choosing \(\tau\) under different operational
objectives.
This framing follows the broader principle that evaluation criteria should be
treated as first-class decision constructs, rather than tuned parameters
\citep{elkan2001}.

\subsection{Target hit-rate calibration}
\label{subsec:target_hit_rate}

The most direct approach specifies a target hit rate \(\hstar\), representing an
explicit readiness standard.
The reference tolerance is chosen as the corresponding quantile of absolute
forecast errors:
\[
    \taustar = Q\!\left(|e|, \hstar\right).
\]
This rule answers the question:
\emph{What tolerance is required to achieve a desired level of coverage?}

Target hit-rate calibration is highly interpretable and aligns naturally with
service-level definitions.
However, it does not account for the shape of the HR@\(\tau\) response surface and
may select tolerances in regions where readiness conclusions are sensitive to
small changes in \(\tau\).

\subsection{Knee-based calibration}
\label{subsec:knee_calibration}

An alternative approach selects \(\taustar\) at a point of diminishing returns in
the HR@\(\tau\) response surface.
Operationally, this corresponds to a tolerance beyond which additional widening
produces minimal gains in hit rate.

Knee-based calibration emphasizes stability rather than coverage targets.
It is particularly useful when explicit readiness standards are unavailable, or
when robustness to tolerance mis-specification is prioritized over absolute hit
rates.

\subsection{Utility-based calibration}
\label{subsec:utility_calibration}

Utility-based calibration introduces an explicit tradeoff between coverage and
tolerance width.
A simple utility function takes the form
\[
    \Utility(\tauv) = \HR(\tauv) - \lambdau \cdot \left(\frac{\tauv}{\taumax}\right),
\]
where \(\lambdau\) governs the penalty applied to wider tolerances.
The selected tolerance maximizes \(\Utility(\tauv)\) over the candidate grid.

This approach makes preferences explicit and tunable, but requires careful
governance of the penalty parameter to avoid implicit bias toward overly narrow
or overly permissive tolerances.

\subsection{Comparison of calibration rules}
\label{subsec:calibration_comparison}

The three calibration rules are best understood as complementary readiness
primitives rather than competing estimators:

\begin{itemize}[leftmargin=*]
    \item \textbf{Target hit-rate} defines an explicit readiness standard
    (\(\hstar\)) and selects \(\taustar\) to meet that coverage level.

    \item \textbf{Knee detection} selects a stability-oriented reference tolerance
    at diminishing returns on the HR@\(\tau\) response surface.

    \item \textbf{Utility maximization} makes the coverage--tolerance tradeoff
    explicit by introducing a governed penalty \(\lambdau\) on tolerance width.
\end{itemize}

Together, these calibration rules provide a structured vocabulary for tolerance
selection.
They enable readiness assessment to be grounded in explicit assumptions, rather
than implicit defaults, and support consistent reporting and governance across
models, entities, and time periods.

The next section extends tolerance calibration to heterogeneous settings by
allowing entity-level tolerances while introducing safeguards against noise,
instability, and tolerance inflation.