% ----------------------------------------------------------
% OVERVIEW
% ----------------------------------------------------------
\section{Overview}
\label{sec:overview}

Hit Rate within Tolerance (\HRtau{}) evaluates forecast performance by measuring
the fraction of intervals whose absolute error falls within an acceptable band
of width \(\tau\).
Unlike loss-based metrics that aggregate error magnitude, \HRtau{} encodes a
binary notion of operational success: whether a forecast is ``close enough''
to be considered usable in practice.

The interpretability of \HRtau{} makes it attractive in operational settings,
but its meaning depends entirely on how the tolerance parameter \(\tau\) is
specified.
A small \(\tau\) enforces strict precision and penalizes even modest deviations,
while a large \(\tau\) admits substantial error without consequence.
As a result, \HRtau{} is not a self-contained metric: it is an evaluation
functional whose behavior is governed by an externally chosen acceptability
band.

This technical note addresses a practical and often overlooked question:
\emph{how should the tolerance \(\tau\) be selected and governed in readiness
evaluation?}
In many applications, \(\tau\) is fixed heuristically, borrowed from convention,
or chosen without reference to historical forecast behavior.
Such choices can materially affect reported performance, model comparisons, and
downstream readiness conclusions.

We argue that tolerance selection should be treated as a calibration problem,
not a formatting choice.
Just as asymmetric cost ratios require governance when using \CWSL{}, tolerance
bands require principled specification when using \HRtau{}.
Without such governance, hit-rate metrics risk becoming either overly punitive
or trivially permissive, undermining their role in readiness assessment.

\subsection{Scope and non-goals}
\label{subsec:scope}

This note assumes familiarity with the definition and interpretation of
\HRtau{}.
We do not propose \HRtau{} as a replacement for loss-based metrics, nor do we
advocate optimizing forecasts to maximize hit rate.
Instead, we treat \HRtau{} as a diagnostic and evaluative primitive whose utility
depends on disciplined tolerance calibration.

The methods presented here:
\begin{itemize}[leftmargin=*]
    \item require \emph{no exogenous data}, relying only on historical
          \((y, \hat{y})\) pairs,
    \item impose \emph{no model assumptions} and apply uniformly across model
          classes,
    \item are \emph{deterministic} given inputs and a specified candidate grid,
    \item produce diagnostics intended for transparent reporting and governance.
\end{itemize}

The goal is not to define a universal ``correct'' tolerance, but to provide
defensible, auditable mechanisms for selecting \(\tau\) in a way that reflects
observed forecast behavior and operational expectations.

\subsection{Contributions}
\label{subsec:contributions}

This note introduces a sensitivity- and calibration-oriented framework for
tolerance selection in hit-rate evaluation:

\begin{enumerate}[leftmargin=*]
    \item \textbf{Response-surface framing.}
    We analyze \HRtau{} as a function of the tolerance parameter, highlighting
    that the relevant object of interest is the curve \(\HR(\tau)\), not a
    single-point estimate.

    \item \textbf{Grid-based sensitivity analysis.}
    We formalize evaluation of \HRtau{} across a candidate tolerance grid to
    expose regions of stability, fragility, and diminishing returns.

    \item \textbf{Data-driven tolerance calibration.}
    We present three principled selection rules for \(\tau\), based solely on
    historical residuals:
    (i) target hit-rate (quantile-based) calibration,
    (ii) knee detection at diminishing returns, and
    (iii) utility maximization trading coverage against tolerance width.

    \item \textbf{Entity-level extensions and safeguards.}
    We extend tolerance calibration to heterogeneous entities and introduce
    governance mechanisms such as minimum sample requirements, global caps, and
    diagnostic reporting to prevent tolerance inflation or overfitting.
\end{enumerate}

\subsection{Organization of the note}
\label{subsec:organization}

Section~\ref{sec:operational_problem} motivates tolerance selection as an
operational decision rather than a cosmetic parameter choice.
Section~\ref{sec:hrtau_response_surface} frames \HRtau{} as a response surface
over \(\tau\).
Section~\ref{sec:sensitivity_analysis} presents grid-based sensitivity analysis.
Sections~\ref{sec:target_hit_rate_calibration} through
\ref{sec:utility_based_calibration} introduce alternative calibration rules.
Section~\ref{sec:entity_level_heterogeneity} extends the approach to entity-level
tolerances, and Section~\ref{sec:governance_and_diagnostics} discusses governance
and reporting.
Section~\ref{sec:limitations} outlines limitations and non-goals, and
Section~\ref{sec:relationship_to_readiness} situates tolerance calibration within
the broader Forecast Readiness Framework.
Section~\ref{sec:conclusion} concludes.