% ----------------------------------------------------------
% GOVERNANCE, DIAGNOSTICS, AND REPORTING
% ----------------------------------------------------------
\section{Governance, Diagnostics, and Reporting}
\label{sec:governance}

Tolerance calibration directly shapes how forecast readiness is defined,
measured, and communicated.
As such, HR@$\tau$ cannot be treated as a purely technical artifact.
This section outlines governance principles and diagnostic practices that ensure
tolerance calibration supports consistent, interpretable, and auditable readiness
assessment.

\subsection{Tolerance as a governed artifact}
\label{subsec:tau_governance}

The tolerance parameter $\tauv$ defines the boundary between acceptable and
unacceptable forecast deviation.
Once calibrated, it establishes a readiness standard that influences model
selection, monitoring, and escalation decisions.
For this reason, $\tauv$ should be treated as a governed artifact rather than an
implicit byproduct of analysis.

At minimum, any deployed tolerance should be accompanied by:
\begin{itemize}[leftmargin=*]
    \item the calibration rule used (e.g., target hit-rate, knee, utility),
    \item the candidate grid $\TauGrid$ over which calibration was performed,
    \item the calibration dataset and sample size,
    \item any applied floors, caps, or exclusions.
\end{itemize}

Recording this information ensures that readiness assessments are reproducible
and that changes in tolerance standards can be traced and reviewed over time.

\subsection{Core diagnostics}
\label{subsec:tau_diagnostics}

Tolerance calibration naturally produces diagnostics that support governance and
interpretation.
Key diagnostics include:
\begin{itemize}[leftmargin=*]
    \item the achieved hit rate $\HRtau(\taustar)$ on the calibration window,
    \item sensitivity of $\HRtau(\tauv)$ around $\taustar$,
    \item the location of $\taustar$ relative to $\taufloor$ and $\taucap$,
    \item the effective sample size used in calibration.
\end{itemize}

These diagnostics allow practitioners to distinguish between robust readiness
standards and fragile thresholds that are sensitive to small changes in error
distribution.
A tolerance that lies in a flat region of the response surface is more defensible
than one chosen near a sharp inflection.

\subsection{Entity-level reporting}
\label{subsec:entity_reporting}

When entity-level tolerances are used, reporting requirements become more
stringent.
Entity-specific tolerances $\taustari$ should always be presented alongside:
\begin{itemize}[leftmargin=*]
    \item the number of observations used for calibration,
    \item the selected calibration rule and parameters,
    \item any global caps or floors applied,
    \item comparison to the global tolerance $\taustar$.
\end{itemize}

Absent this context, entity-level tolerances risk being misinterpreted as
performance endorsements rather than diagnostic signals.
Transparent reporting ensures that heterogeneity is understood as an empirical
property of the system, not an excuse for degraded readiness.

\subsection{Change management and review cadence}
\label{subsec:change_management}

Tolerance calibration should not be updated opportunistically.
Changes to $\tauv$ or $\taustari$ redefine readiness standards and therefore
require explicit review.
A governed process should specify:
\begin{itemize}[leftmargin=*]
    \item the conditions under which recalibration is permitted,
    \item the evaluation windows used for recalibration,
    \item approval or review checkpoints for tolerance changes.
\end{itemize}

In stable environments, tolerances may remain fixed for extended periods.
In rapidly evolving regimes, scheduled recalibration may be appropriate.
In all cases, tolerance drift should be monitored and justified rather than
implicitly accepted.

\subsection{Role in readiness assessment}
\label{subsec:role_in_readiness}

Within the Forecast Readiness Framework, HR@$\tau$ serves as a gatekeeping metric.
It determines whether forecasts satisfy minimum alignment requirements before
more nuanced performance metrics are considered.
Governance of $\tauv$ therefore directly governs what it means for a forecast to
be considered ``ready.''

By pairing tolerance calibration with explicit diagnostics and reporting
standards, organizations can ensure that readiness assessments remain stable,
defensible, and aligned with operational intent.
The next section discusses limitations and non-goals of tolerance-based readiness
metrics, clarifying what HR@$\tau$ is—and is not—designed to capture.