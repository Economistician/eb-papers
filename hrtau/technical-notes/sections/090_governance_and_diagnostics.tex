% ----------------------------------------------------------
% GOVERNANCE, DIAGNOSTICS, AND REPORTING
% ----------------------------------------------------------
\section{Governance, Diagnostics, and Reporting}
\label{sec:governance}

Tolerance calibration plays a foundational role in readiness evaluation.
Because HR@$\tau$ directly defines what constitutes an acceptable deviation
between forecast and realization, the choice of $\tau$ determines the evaluation
context within which readiness is assessed.
As such, tolerance calibration must be governed with the same care as other
operational assumptions, rather than treated as an incidental technical detail.

This section outlines governance principles and diagnostics that support
transparent, stable, and auditable use of HR@$\tau$ in production environments.

\subsection{Separation of calibration and evaluation}
\label{subsec:separation}

A core governance principle is the strict separation between
\emph{calibration} and \emph{evaluation}.
Tolerance calibration determines the reference band $\taustar$ within which
forecast performance is assessed; it does not alter the forecasts themselves
nor adapt the metric to favor a particular model ex post.

Calibration must therefore be performed on a fixed historical window and held
constant when evaluating candidate models or future forecasts.
This separation preserves comparability across models, entities, and time
periods, and prevents circular evaluation in which the tolerance adapts to the
model under inspection.

\subsection{Required diagnostic outputs}
\label{subsec:diagnostics}

Every calibrated tolerance should be accompanied by a minimal set of diagnostic
artifacts.
At a minimum, reporting should include:
\begin{itemize}[leftmargin=*]
    \item the candidate grid $\TauGrid$ used for calibration,
    \item the selected tolerance $\taustar$ (or $\taustari$),
    \item the achieved hit rate $\HR(\taustar)$ on the calibration window,
    \item the number of valid observations used in calibration,
    \item any applied floors, caps, or exclusions.
\end{itemize}

When knee-based or utility-based selection rules are used, additional diagnostics
such as local slope estimates or utility curves should be retained.
These diagnostics ensure that the selected tolerance can be inspected and
justified independently of the calibration procedure itself.

\subsection{Stability and sensitivity checks}
\label{subsec:stability_checks}

Tolerance calibration should not be interpreted as a point estimate.
Instead, practitioners should assess whether the selected $\taustar$ lies in a
region where HR@$\tau$ is locally stable.

Flat regions of the response surface indicate robustness: moderate changes in
$\tau$ do not materially affect readiness conclusions.
Conversely, steep regions signal fragility, where small changes in tolerance
produce large swings in hit rate.
In such cases, broader reporting or more conservative tolerance selection may be
warranted.

These stability considerations are particularly important when tolerances are
used as contractual thresholds, service-level objectives, or escalation triggers.

\subsection{Entity-level reporting and safeguards}
\label{subsec:entity_reporting}

When entity-level tolerances are computed, governance requirements become more
stringent.
Entity-level outputs should always be reported alongside:
\begin{itemize}[leftmargin=*]
    \item entity-specific sample sizes,
    \item achieved hit rates at $\taustari$,
    \item any global caps or overrides applied,
    \item the corresponding global reference tolerance $\taustar$.
\end{itemize}

This contextual information is essential for distinguishing meaningful readiness
heterogeneity from noise-driven variation.
Absent such diagnostics, entity-level tolerances risk being misinterpreted as
prescriptive targets rather than descriptive indicators.

\subsection{Auditability and lifecycle management}
\label{subsec:auditability}

Tolerance calibration should be treated as a governed artifact with a defined
lifecycle.
Calibration inputs, rules, and outputs should be versioned, timestamped, and
recomputed only when material changes occur in demand behavior, forecast systems,
or operational objectives.

In regulated or high-stakes environments, the ability to reconstruct the
calibration decision—given the historical data and declared rules—is essential.
HR@$\tau$ calibration is therefore designed to be fully deterministic and
reproducible, supporting retrospective audit and forward-looking governance.

Taken together, these practices ensure that tolerance calibration strengthens
readiness assessment rather than obscuring it.
In the next section, we situate HR@$\tau$ calibration within the broader
Forecast Readiness Framework and clarify its role alongside other readiness
primitives.