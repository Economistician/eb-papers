% ----------------------------------------------------------
% ENTITY-LEVEL TOLERANCE AND SAFEGUARDS
% ----------------------------------------------------------
\section{Entity-Level Tolerance and Safeguards}
\label{sec:entity_level}

Operational systems often exhibit heterogeneous error behavior across entities.
Products, locations, services, or demand segments may differ substantially in
their intrinsic volatility, signal-to-noise ratio, or operational tolerance for
forecast deviation.
Applying a single global tolerance $\tauv$ in such settings can obscure
systematic differences and mask localized readiness risks.
This section extends tolerance calibration to the entity level while introducing
explicit safeguards to preserve interpretability and governance.

\subsection{Motivation for entity-specific tolerance}
\label{subsec:entity_motivation}

Let $\entity \in I$ index entities such as items, stores, or service classes.
Historical residuals frequently reveal persistent differences in dispersion
across entities.
Some entities exhibit tight, stable error distributions, while others experience
broader variability due to demand intermittency, substitution effects, or
exogenous shocks.

Allowing entity-specific tolerances $\taui$ enables readiness evaluation to
reflect these differences.
Under a global tolerance, a volatile entity may appear perpetually ``not ready,''
while a stable entity may be evaluated too leniently.
Entity-level calibration therefore improves diagnostic fidelity by aligning
tolerance bands with observed error structure.

However, unconstrained entity-level tolerance selection risks normalizing poor
forecast performance.
Without safeguards, high-variance entities may receive inflated tolerances that
redefine readiness downward rather than improving forecasting behavior.
The objective is thus not maximal flexibility, but controlled heterogeneity.

\subsection{Entity-level calibration rule}
\label{subsec:entity_rule}

For each entity $\entity$, we apply the same calibration rules described in
Section~\ref{sec:calibration} to the subset of residuals associated with that
entity.
Given a minimum sample requirement $\nmin$, we compute
\[
    \taustari
    \in
    \arg\max_{\tauv \in \TauGrid}
    \mathcal{C}_{\entity}(\tauv),
\]
where $\mathcal{C}_{\entity}(\tauv)$ denotes a chosen calibration criterion
(e.g., target hit rate, knee-point, or utility) evaluated using only observations
for entity $\entity$.

Entities with fewer than $\nmin$ valid observations are excluded from
entity-level calibration and assigned no entity-specific tolerance.
This constraint prevents unstable estimates driven by sparse data and makes the
calibration behavior explicit.

\subsection{Global caps and guardrails}
\label{subsec:entity_safeguards}

Even when sufficient data are available, entity-level calibration can yield
excessively large tolerances for entities with persistent forecast dispersion.
To prevent tolerance inflation and preserve cross-entity comparability, we
introduce optional global safeguards.

A common approach is to cap entity-level tolerances by a global reference value
$\taustar$ derived from the full dataset:
\[
    \taustari \leftarrow \min(\taustari, \taustar).
\]
More generally, caps may be defined using high quantiles of the empirical
distribution of entity-level tolerances or through domain-specific maximum
acceptable error thresholds.

Lower bounds may also be imposed to avoid pathological contraction of tolerance
bands in low-variance regimes:
\[
    \taustari \ge \taufloor.
\]
Together, these guards ensure that entity-level calibration refines evaluation
without undermining system-wide readiness standards.

\subsection{Interpretation and reporting}
\label{subsec:entity_interpretation}

Entity-level tolerances should be reported alongside diagnostics including sample
size, achieved hit rate, calibration rule used, and any applied caps or floors.
These diagnostics are essential for distinguishing meaningful heterogeneity from
noise-driven variation.

Importantly, entity-level tolerance calibration is intended primarily as a
\emph{diagnostic and governance tool}, not as a universal deployment default.
In many production environments, a hybrid strategy—global tolerance with
entity-level overrides for well-supported cases—offers the best balance between
fidelity, comparability, and operational discipline.

The next section formalizes governance workflows and diagnostic practices that
support responsible use of tolerance calibration in readiness assessment.