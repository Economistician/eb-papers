\section*{Business Impact}

When structural incompatibility goes unrecognized, organizations pay twice. First, through continued operational disruption as readiness fails to improve. Second, through escalating mitigation costs as buffers, surplus, and intervention effort increase without commensurate benefit.

These dynamics erode confidence in forecasting systems. Teams lose trust not because forecasts are imperfect, but because repeated adjustment fails to produce stability. Over time, evaluation becomes performative, and readiness metrics are discounted or ignored.

Recognizing incompatibility changes the decision calculus. It prevents unproductive tuning, constrains the use of readiness interventions, and redirects attention toward alternative representations or decision rules better aligned with demand behavior.

By distinguishing structural mismatch from tunable deficiency, FPC protects both operational outcomes and governance credibility. It ensures that readiness tools are applied where they are effective and withheld where they are not.
