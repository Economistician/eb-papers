\section*{Where FPC Fits in Forecast Governance}

Forecast Primitive Compatibility operates as a governance boundary within the Forecast Readiness Framework. It sits downstream of metric evaluation but upstream of policy enforcement, constraining when readiness interventions are structurally defensible.

FPC does not measure performance and is not an optimization target. Its output is a categorical assessment of whether a forecast primitive admits meaningful responsiveness to admissible readiness adjustment. This classification informs, but does not replace, governance decisions.

When demand is classified as Compatible, readiness policies such as scale adjustment, tolerance interpretation, and cost calibration are structurally valid. When demand is Marginal or Incompatible, these policies must be constrained, annotated, or disabled to preserve interpretability and auditability.

Positioned this way, FPC completes the readiness evaluation stack. It provides a principled stop condition that prevents misapplication of otherwise valid tools and ensures that governance decisions remain grounded in structural validity rather than metric behavior alone.
