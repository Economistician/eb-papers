\section*{The Problem: Tuning Cannot Fix Structural Mismatch}

When forecasts fail operationally, the default response is often to tune harder. Scale adjustments are increased, buffers are added, and asymmetric penalties are recalibrated in an attempt to improve coverage and reduce risk. In many cases, these interventions are effective. In others, they are not.

The difficulty is that persistent failure is often interpreted as insufficient tuning rather than structural mismatch. When demand exhibits intermittent, bursty, or discrete behavior, incremental adjustment within a point forecasting paradigm may deliver little improvement, even as surplus, volatility, and cost exposure increase.

Traditional evaluation metrics can obscure this distinction. Poor readiness may appear similar whether it arises from bias, under-scaling, or fundamental incompatibility between the demand process and the forecast representation. Without a way to diagnose this difference, organizations risk applying valid interventions in regimes where they are structurally ineffective.

The problem is not that readiness tools fail, but that they are sometimes applied beyond their domain of validity. Without an explicit mechanism to detect this boundary, tuning efforts can become unproductive and misleading.
