\section*{The Core Insight: Severity Drives Disruption}

Operational disruption is driven less by the occurrence of shortfalls than by their depth. When forecasts underbuild demand by a wide margin, the system does not merely experience a miss; it experiences stress. Recovery takes longer, secondary effects compound, and normal operating rhythms are broken.

The key insight is that severity matters independently of frequency. Two forecasting systems may fail the same number of times, yet impose vastly different operational burdens depending on how deep those failures are. A shallow miss may be corrected quickly, while a deep shortfall can destabilize multiple downstream intervals.

Severity also determines the effectiveness of buffers and safeguards. Many operational protections are designed to absorb modest variability. Once those protections are overwhelmed, recovery becomes nonlinear and expensive. Understanding how deep failures run when they occur is therefore essential to assessing true operational risk.

By isolating underbuild depth as a distinct diagnostic, UD captures a dimension of forecast performance that is otherwise conflated with average error or failure frequency. It focuses attention on the failures that matter most, rather than treating all misses as equivalent.
