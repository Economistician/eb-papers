\section*{Business Impact}

Severe shortfalls impose costs that extend beyond the interval in which they occur. Recovery consumes labor, capacity, and managerial attention, often spilling into subsequent periods. Throughput losses compound, service levels degrade, and operational plans are disrupted well after demand has passed.

These events also shape organizational behavior. When deep failures occur, teams lose confidence in forecasts and compensate by adding manual buffers, overrides, or conservative heuristics. While these adaptations may reduce immediate risk, they increase variability and reduce the efficiency gains that forecasting is meant to deliver.

From a financial perspective, a small number of severe underbuilds can dominate operational loss. The cost of recovery, lost revenue, and degraded service often outweighs the cumulative impact of many smaller deviations. Yet without explicit measurement, these events remain anecdotal rather than diagnosable.

By making severity visible, organizations can identify where forecasting systems expose them to outsized disruption. Reducing the depth of the worst failures—even without changing average accuracy or failure frequency—can materially improve stability, resilience, and trust in execution.
