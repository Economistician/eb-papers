\section*{The Problem: Not All Shortfalls Are Equal}

Operational shortfalls are often discussed as binary events: demand was met or it was not. While this distinction is important, it overlooks a critical reality of execution. When shortfalls occur, their severity varies widely, and that variation largely determines the operational impact.

In many environments, shallow shortfalls can be absorbed with limited disruption. Teams adjust, buffers are used, and operations continue with minor friction. Deep shortfalls, however, trigger a very different response. They create extended recovery periods, degrade throughput, and force escalations that ripple across adjacent intervals and systems.

Traditional evaluation does not distinguish between these outcomes. Forecasts that miss narrowly and forecasts that miss by large margins may be treated as equally deficient if they fail in the same number of intervals. This masks an important source of risk: a forecasting system that underbuilds infrequently but deeply may be far more damaging than one that misses more often but by small amounts.

The problem, therefore, is not simply whether shortfalls occur. It is how severe those shortfalls are when they happen. Without a way to isolate and measure the depth of underbuilding, organizations lack visibility into one of the primary drivers of operational disruption and recovery cost.
