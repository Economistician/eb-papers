\section*{Where UD Fits in Forecast Readiness}

Underbuild Depth is a conditional diagnostic designed to assess the severity of failures when they occur. It is not a standalone objective and should not be optimized in isolation. Its role is to complement measures of accuracy, frequency, and cost by revealing how fragile a forecasting system becomes under stress.

Used alongside reliability metrics such as NSL and economic measures such as CWSL, UD helps complete the picture of forecast readiness. It distinguishes between systems that fail gracefully and those that fail catastrophically, even when headline performance appears similar.

Within a readiness framework, UD acts as a risk indicator. Elevated underbuild depth signals exposure to severe disruption and warrants caution, mitigation, or additional safeguards before deployment. Stable or improving UD values, by contrast, indicate that failures—when they occur—remain manageable.

By positioning UD as a diagnostic rather than a target, organizations preserve its interpretability and avoid unintended incentives. In this role, UD strengthens forecast governance by making severity explicit, measurable, and governable—ensuring that forecasting systems are evaluated not just on how often they fail, but on how damaging those failures truly are.
