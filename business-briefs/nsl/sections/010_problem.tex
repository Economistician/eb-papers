\section*{The Problem: Accuracy Does Not Measure Reliability}

Operational leaders depend on forecasts to ensure that demand is met consistently, interval by interval. In these environments, the most disruptive failures are not large errors on average, but moments when demand is not covered at all. Yet forecasting performance is still most often judged using accuracy metrics that summarize deviation rather than reliability.

This creates a mismatch between how forecasts are evaluated and how they are experienced in practice. A forecast can achieve strong average accuracy while still failing repeatedly in specific intervals. Each of these failures introduces friction into operations, triggering delays, bottlenecks, or reactive recovery actions that compound over time.

Because accuracy metrics aggregate error across intervals, they obscure how often forecasts actually fall short. Small but frequent shortfalls may be averaged away, while clustering of failures during critical periods remains hidden. From a reporting perspective, performance appears stable. From an operational perspective, reliability remains uncertain.

The real issue is not whether forecasts are close to demand on average. It is whether they reliably meet demand when it matters. Without a way to measure how frequently shortfalls occur, organizations may continue to improve reported accuracy while experiencing little improvement in operational stability or service quality.
