\section*{The Core Insight: Reliability Is About Failure Frequency}

Operational reliability is experienced as a sequence of discrete events. In each interval, demand is either met or it is not. From the perspective of execution, these outcomes matter far more than the average size of forecast error accumulated over time.

The key insight is that reliability is fundamentally about how often forecasts fail, not how far they miss when they do. A small shortfall can be just as disruptive as a large one if it triggers the same operational response. What determines reliability, therefore, is the frequency with which shortfall events occur.

This perspective reframes how forecast performance should be interpreted. Two forecasts may exhibit similar average accuracy, yet behave very differently in practice. One may meet demand in most intervals, while the other falls short repeatedly. From an operational standpoint, these forecasts are not interchangeable, even if traditional metrics suggest otherwise.

The No-Shortfall Level captures this distinction by focusing on the occurrence of shortfalls rather than their magnitude. It provides a direct measure of interval-level reliability, aligned with how operations experience success and failure. By separating the question of how often forecasts fail from how severely they fail, NSL makes reliability explicit rather than implicit.
