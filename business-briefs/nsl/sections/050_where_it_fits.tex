\section*{Where NSL Fits in Forecast Readiness}

The No-Shortfall Level is a diagnostic measure of interval-level reliability, not a standalone objective to be optimized. Its purpose is to make visible how often forecasts fail to meet operational requirements, complementing metrics that describe error magnitude or economic cost.

Used alongside accuracy and cost-based measures, NSL helps distinguish between forecasts that are numerically precise and those that are operationally dependable. It provides a clear signal of failure frequency, enabling teams to assess whether a forecast supports stable execution before it is deployed into production environments.

Within a broader readiness framework, NSL functions as an early-warning indicator. Declining reliability can be detected even when average accuracy remains unchanged, allowing intervention before repeated shortfalls translate into sustained operational disruption. Conversely, high NSL values indicate that forecasts are consistently meeting baseline coverage requirements, supporting confidence in downstream decision-making.

By treating NSL as an evaluative reliability diagnostic rather than a performance target, organizations preserve its interpretability and avoid unintended incentives. Positioned this way, NSL strengthens forecast readiness by making reliability explicit, measurable, and governable—without conflating evaluation with execution or control.
