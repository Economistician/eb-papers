\section*{Business Impact}

When forecasts fail to meet demand repeatedly, even by small amounts, the impact on operations is immediate and cumulative. Each shortfall introduces disruption—queues form, throughput drops, staff intervene, and managers are pulled into reactive problem-solving. Over time, these interruptions erode operational stability and consume capacity that could otherwise be used productively.

Because shortfalls are event-driven, their cost is not proportional to their magnitude. A brief failure can trigger the same recovery actions as a larger one, especially in tightly coupled systems. As a result, frequent small shortfalls often impose a greater burden than occasional larger deviations, even though they may be nearly invisible in aggregate accuracy reporting.

This pattern also affects organizational behavior. When forecasts appear accurate but fail operationally, trust erodes. Teams begin to rely on manual buffers, overrides, or informal heuristics to protect against repeated failures. These workarounds increase complexity and variability, further distancing forecasting from its intended role as a decision-support input.

By making the frequency of shortfalls explicit, a reliability-focused view highlights where operational friction actually originates. Reducing the number of failure events—even without improving average accuracy—can materially improve service consistency, lower firefighting overhead, and restore confidence in forecasting systems as reliable contributors to execution.
