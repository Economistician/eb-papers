\section*{Why Advisory Governance Fails}

When governance is treated as advisory rather than authoritative, structural constraints are easily bypassed. Diagnostics are mixed, interpretations shift, and policies drift without detection.

Common failure modes include applying raw-unit metrics where snapped interpretation is required, widening tolerances to improve reported performance, or applying readiness adjustments despite structural incompatibility. These actions may appear locally reasonable but undermine the integrity of the decision process.

Advisory governance also weakens accountability. When outcomes are unfavorable, responsibility is diffused across models, metrics, and operators. No single decision point exists where assumptions were declared and enforced.

Without deterministic closure, governance becomes another signal rather than a boundary. The result is semantic drift, non-reproducible decisions, and loss of trust in readiness assessment.
