\section*{Business Impact}

Lack of explicit governance increases operational risk. Decisions based on ambiguous interpretation are difficult to audit, reproduce, or defend, particularly in automated or high-stakes environments.

Over time, teams compensate by adding manual review, conservative buffers, or informal overrides. These practices reduce efficiency and undermine the scalability of forecasting systems. More critically, they obscure the true source of failure when outcomes degrade.

Conversely, explicit governance improves clarity and trust. Decisions become repeatable, traceable, and defensible. When action is restricted, the reason is explicit and structural—not subjective or political.

By enforcing a single authoritative decision artifact, governance transforms analytics into accountable policy. This enables organizations to scale automation while preserving control, safety, and institutional memory.
