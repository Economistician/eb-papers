\section*{The Problem: Diagnostics Without Authority}

Modern forecasting systems produce increasingly sophisticated diagnostics. Measures of accuracy, reliability, tolerance compliance, cost exposure, and readiness are readily available. Yet despite this analytical richness, operational decisions often remain ambiguous.

The difficulty is not a lack of information, but a lack of authority. Diagnostics describe behavior, but they do not determine what actions are permitted. In the absence of an explicit decision layer, teams interpret metrics heuristically, reconcile contradictions informally, and apply adjustments opportunistically.

This ambiguity creates inconsistency. Identical conditions may lead to different actions depending on who is interpreting the diagnostics, which metrics are emphasized, or how units and tolerances are implicitly understood. Over time, this erodes auditability and accountability.

The core problem is that diagnostics alone cannot close a decision. Without governance, evaluation remains advisory, interpretation remains negotiable, and responsibility for readiness outcomes becomes diffuse.
