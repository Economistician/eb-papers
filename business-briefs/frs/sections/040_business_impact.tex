\section*{Business Impact}

When readiness is misjudged, the consequences emerge during execution rather than evaluation. Forecasts that are not truly ready introduce volatility into operations, forcing teams to react rather than execute. Service levels fluctuate, recovery actions multiply, and managerial attention is diverted to firefighting.

These disruptions carry both direct and indirect costs. Throughput loss, labor inefficiency, and waste accumulate alongside erosion of trust in forecasting systems. Teams compensate by adding buffers, manual controls, or conservative overrides, reducing the benefits that forecasting was intended to provide.

Conversely, when readiness is assessed explicitly and consistently, deployment decisions improve. Forecasts that meet readiness criteria behave predictably under stress, allowing operations to plan with confidence. Failures still occur, but they are contained within known and tolerable bounds.

Over time, this discipline reduces operational volatility and shortens recovery cycles. Forecasting becomes a reliable input to decision-making rather than a recurring source of surprise, enabling more stable execution and better alignment between analytics and operations.
