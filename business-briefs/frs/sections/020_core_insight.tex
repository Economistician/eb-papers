\section*{The Core Insight: Readiness Requires Reliability and Cost Discipline}

Operational readiness is not a single attribute. It emerges from the interaction of reliability, failure severity, and economic exposure. A forecast is ready not because it is accurate, but because its errors are predictable, manageable, and aligned with operational tolerance.

This means readiness must be evaluated across multiple dimensions. How often does the forecast fail? When it fails, how severe are those failures? What is the economic impact of being wrong in each direction? Each of these questions captures a distinct aspect of operational risk.

The Forecast Readiness Score integrates these dimensions into a unified assessment. Rather than replacing individual diagnostics, it provides a structured way to interpret them together. Readiness is determined by whether a forecast meets minimum standards across all critical dimensions, not by excelling in any single one.

This approach reflects how operations actually function. Systems fail not because one metric is slightly off, but because weaknesses align. Readiness, therefore, is a composite condition, requiring balanced performance rather than isolated optimization.
