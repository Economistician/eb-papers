\section*{The Problem: Accuracy Does Not Answer Readiness}

Forecast evaluation often produces a paradox. Models appear accurate, performance metrics improve, and yet teams remain uncertain about whether a forecast can be safely deployed to run operations. Despite extensive analysis, the final decision still relies on judgment, debate, or conservative overrides.

This uncertainty exists because accuracy alone does not answer the question operations actually face. Being close to demand on average does not guarantee that forecasts will behave acceptably under real execution conditions. Reliability, failure patterns, and cost exposure matter just as much as numerical proximity.

As a result, organizations frequently discover readiness issues only after deployment. Forecasts that look strong in evaluation fail under pressure, leading to service disruptions, recovery actions, and loss of confidence. These failures are not surprises in hindsight, but they are rarely visible ahead of time.

The core problem is not a lack of metrics, but a lack of synthesis. Without a way to integrate multiple dimensions of performance into a readiness judgment, teams are left optimizing numbers that do not directly correspond to operational feasibility.
