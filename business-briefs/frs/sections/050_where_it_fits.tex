\section*{Where FRS Fits in Forecast Governance}

The Forecast Readiness Score is an evaluative construct, not an optimization target. Its purpose is to support deployment decisions by summarizing whether a forecast meets minimum standards across critical dimensions of performance.

Within a governance framework, FRS serves as a gate rather than a goal. Forecasts are not improved by maximizing readiness scores, but by addressing the underlying issues revealed by their components. This preserves interpretability and prevents gaming of composite metrics.

FRS also supports consistency and auditability. By applying the same readiness criteria across models, time periods, and operational contexts, organizations can make deployment decisions that are defensible and repeatable. Changes in readiness reflect real shifts in performance or conditions, not changes in evaluation philosophy.

Positioned this way, FRS provides the missing link between detailed diagnostics and executive judgment. It answers the operational question directly—whether a forecast is ready to run the operation—while remaining grounded in transparent, component-level evaluation.
