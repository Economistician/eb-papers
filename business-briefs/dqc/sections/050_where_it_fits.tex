\section*{Where DQC Fits in Forecast Governance}

Demand Quantization Compatibility operates at the foundation of forecast evaluation. It is a prerequisite check that determines whether performance metrics can be meaningfully interpreted at all.

Within a governance framework, DQC sits upstream of accuracy, reliability, tolerance, cost, and readiness metrics. It does not assess performance; it establishes admissibility. When demand is continuous, standard evaluation proceeds unchanged. When demand is quantized or packed, evaluation must be adapted or constrained accordingly.

DQC is not a tuning mechanism and should not be optimized. Its role is binary and structural: to confirm that forecasts are evaluated in a unit space that aligns with executable demand. Once this condition is met, downstream metrics regain their interpretability and relevance.

By formalizing this check, organizations ensure that forecast evaluation rests on a valid structural foundation. DQC prevents misinterpretation before it occurs, strengthening governance and preserving trust in all subsequent readiness and performance assessments.
