\section*{Business Impact}

When evaluation is structurally misaligned with demand units, decision-making suffers. Forecasts may be rejected for apparent inaccuracy despite being operationally acceptable, or approved based on metrics that do not reflect executable outcomes. In both cases, the business bears the cost of misinterpretation.

These errors propagate into governance. Teams debate metrics rather than actions, thresholds are adjusted to reconcile contradictions, and confidence in evaluation erodes. Over time, structured assessment gives way to informal judgment, undermining the benefits of systematic forecasting.

Conversely, ensuring unit compatibility restores clarity. When evaluation respects demand structure, performance signals become stable and interpretable. Differences in metrics reflect real operational variation rather than artifacts of representation.

By establishing admissibility upfront, organizations avoid false positives and false negatives in readiness decisions. This improves trust in evaluation, reduces unnecessary intervention, and aligns forecasting outputs more closely with how operations actually function.
