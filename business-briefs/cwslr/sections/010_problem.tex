\section*{The Problem: Cost Asymmetry Is Assumed, Not Governed}

Operational forecasting decisions routinely depend on assumptions about the relative cost of being short versus being long. These assumptions shape how forecast performance is interpreted and which models are considered acceptable for deployment. Yet in practice, they are rarely made explicit, calibrated, or governed.

Instead, cost asymmetry is often treated as an informal intuition. Teams select a cost ratio based on precedent, convenience, or a rough sense of operational priorities. Once chosen, that assumption is embedded into evaluation results, frequently without visibility into how sensitive conclusions are to the choice itself.

This creates a fragility problem. Small changes in assumed cost asymmetry can materially alter which forecasts appear preferable, even when underlying performance is similar. Because this sensitivity is not examined, organizations may approve or reject forecasts based on assumptions that are neither stable nor defensible under scrutiny.

The risk is not that cost asymmetry is the wrong concept. The risk is that it is treated as a fixed input rather than a governed decision. Without a disciplined way to surface sensitivity and anchor assumptions, forecast evaluation becomes dependent on unexamined choices, limiting comparability across time, teams, and deployment contexts.
