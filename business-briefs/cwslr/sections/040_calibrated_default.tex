\section*{A Defensible Default: Balance-Based Calibration}

While sensitivity analysis exposes fragility, operational decisions still require a concrete reference point. Teams need a default cost ratio that can be used consistently for comparison, reporting, and governance, without turning evaluation into a search for the most favorable assumption.

Balance-based calibration provides that anchor. Instead of selecting a cost ratio to optimize apparent performance, the calibration identifies a neutral point where opposing error directions are economically balanced. This produces a reference assumption that is stable, interpretable, and independent of any single model’s behavior.

The value of this approach is not precision, but discipline. A calibrated default allows forecasts to be compared under a common contract, ensuring that differences in evaluation reflect differences in performance rather than differences in assumptions. It also creates a clear separation between defining business priorities and measuring how well forecasts align with them.

By establishing a defensible default, organizations avoid the trap of assumption drift. Changes in performance can be attributed to changes in forecasts or operating conditions, not to silent shifts in cost weighting. This consistency strengthens trust in evaluation results and supports more reliable decision-making over time.
