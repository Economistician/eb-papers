\section*{The Core Insight: Treat Cost Ratio as an Evaluation Contract}

The relative cost of being short versus being long is not a modeling choice. It is a business decision that defines how forecast error should be interpreted. Once that decision is made, it effectively acts as a contract between the operation and the evaluation process.

When cost ratio is treated like a tunable parameter, evaluation outcomes become unstable. Model rankings can shift as assumptions change, and it becomes unclear whether differences in performance reflect genuine operational advantage or simply the choice of cost weighting. This ambiguity undermines confidence in deployment decisions and complicates governance.

Reframing the cost ratio as an evaluation contract resolves this problem. Instead of asking which ratio produces the best-looking results, the organization explicitly states how much it is willing to pay for shortfalls relative to excess. That assumption is then held fixed for the purpose of comparison, making evaluation outcomes interpretable and defensible.

Under this view, sensitivity is not a flaw to be avoided but a property to be examined. If conclusions change dramatically under small shifts in assumed cost, that fragility becomes visible and actionable. Treating cost ratio as a governed contract ensures that forecast evaluation reflects deliberate business priorities rather than implicit or accidental assumptions.
