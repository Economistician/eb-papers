\section*{What This Enables: Auditable and Comparable Evaluation Decisions}

Treating cost ratio as a governed evaluation contract enables forecast assessment that is both auditable and comparable across time, teams, and deployment contexts. Evaluation outcomes can be traced back to explicit assumptions rather than inferred after the fact, reducing ambiguity in both approval and escalation decisions.

With a calibrated default and an explicit sensitivity view, organizations gain a structured way to reason about uncertainty. Forecasts can be compared not only on point performance, but on robustness to plausible variation in business priorities. This allows decision-makers to distinguish between forecasts that are consistently reliable and those whose apparent advantage depends on fragile assumptions.

This discipline also improves comparability across operational contexts. When different products, regions, or time horizons are evaluated under the same governed contract, differences in performance reflect real operational variation rather than shifts in evaluation philosophy. Over time, this consistency supports clearer benchmarking and more credible trend analysis.

Within a broader readiness and governance framework, CWSLR acts as the mechanism that stabilizes economic interpretation. It ensures that cost asymmetry is handled deliberately rather than implicitly, allowing forecast evaluation to scale without losing interpretability or trust. In doing so, it turns cost assumptions from a source of hidden fragility into a transparent and manageable part of operational decision-making.
