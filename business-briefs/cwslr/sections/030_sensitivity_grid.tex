\section*{Make Fragility Visible: Sensitivity Over a Governed Grid}

When evaluation outcomes depend on assumed cost asymmetry, robustness matters as much as point performance. A forecast that appears optimal under one assumption but degrades sharply under nearby assumptions represents a fragile decision, even if its headline performance looks strong.

CWSLR addresses this by evaluating performance across a governed grid of cost ratios rather than at a single point. This approach makes sensitivity explicit. It reveals whether conclusions are stable across a reasonable range of business assumptions or whether they hinge on a narrow and potentially arbitrary choice.

Viewed this way, sensitivity analysis is not about tuning performance. It is about exposing risk. When forecast rankings remain consistent across the grid, decision-makers can act with confidence that deployment choices are not overly dependent on a single assumption. When rankings flip or diverge, that instability becomes a signal in its own right, indicating where additional scrutiny or caution is warranted.

By making fragility visible rather than implicit, a governed sensitivity grid turns an abstract concern into an interpretable diagnostic. It allows organizations to distinguish between forecasts that are robust under uncertainty and those whose apparent advantage depends on assumptions that may not hold consistently in practice.
