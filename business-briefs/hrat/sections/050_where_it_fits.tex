\section*{Where HR@τ Fits in Forecast Readiness}

HR@τ is a diagnostic measure of operational sufficiency, not a target for optimization. Its role is to indicate how often forecasts fall within acceptable tolerance bounds, complementing metrics that measure frequency, severity, and economic cost of failure.

Used alongside measures such as NSL, UD, and CWSL, HR@τ helps complete the readiness picture. It distinguishes between forecasts that are usable most of the time and those that require frequent adjustment, even when other metrics appear similar.

Within a readiness framework, HR@τ provides a clear signal of intervention burden. Declines in tolerance compliance can be detected early, allowing corrective action before operational stability degrades. Improvements indicate that forecasts are becoming more dependable from an execution standpoint.

By treating HR@τ as an evaluative signal rather than an optimization goal, organizations preserve its interpretability and avoid perverse incentives. Positioned correctly, HR@τ strengthens forecast readiness by aligning evaluation with operational tolerance and real-world usability.
