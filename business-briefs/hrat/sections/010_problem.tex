\section*{The Problem: Precision Is Not the Same as Sufficiency}

Forecast evaluation is often framed around precision: how close predictions are to realized demand. While precision is important, it is not the standard by which operations actually succeed or fail. In many operational environments, forecasts do not need to be exact to be usable.

Operations are designed with tolerance. Small deviations can be absorbed through buffers, slack capacity, or routine adjustments without disrupting execution. Yet traditional accuracy metrics treat all deviation as equally problematic, failing to distinguish between errors that matter and those that do not.

This creates a disconnect between evaluation and experience. Forecasts may appear inaccurate on paper while functioning adequately in practice, or appear accurate while requiring frequent manual intervention. Precision alone does not indicate whether a forecast can be relied upon to run the operation smoothly.

The core problem is that accuracy metrics do not measure sufficiency. Without a way to assess whether forecasts fall within acceptable tolerance bands, organizations lack visibility into how often forecasts are operationally usable without adjustment.
