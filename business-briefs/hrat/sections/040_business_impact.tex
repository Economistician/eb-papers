\section*{Business Impact}

When forecasts remain within operational tolerance, execution is smoother and less reactive. Teams can rely on forecasts without frequent adjustments, allowing capacity, labor, and inventory decisions to proceed as planned.

Conversely, forecasts that frequently exceed tolerance thresholds demand attention. Managers intervene, buffers are added, and informal workarounds emerge. While these actions may prevent immediate failure, they increase complexity and reduce efficiency over time.

By measuring tolerance compliance explicitly, organizations gain a clearer view of forecast usability. Improving HR@τ often delivers outsized benefits by reducing the frequency of manual intervention, even if average accuracy remains unchanged.

Over time, this leads to more stable operations, lower adjustment overhead, and greater trust in forecasting systems. Forecasts become reliable inputs rather than recurring sources of friction, improving both performance and organizational confidence.
