\section{The Core Insight: Readiness Is Not Accuracy}

Forecast accuracy describes how close predictions are to realized demand on average. Operational readiness describes whether forecasts can be relied upon to support execution when it matters. These are not the same thing, and treating them as interchangeable creates systematic risk.

In operational environments, forecast error is directional and consequential. Under-forecasting and over-forecasting do not produce symmetric outcomes. Shortfalls trigger service failures, recovery actions, and lost throughput, while excess is often absorbed with limited disruption. A forecast can therefore appear accurate in aggregate while repeatedly failing in the direction that carries the greatest operational cost.

Readiness depends not only on how large forecast errors are, but on how they behave. How often do shortfalls occur? How severe are they when they arise? How frequently do deviations remain within tolerable bounds that the operation can absorb without intervention? These characteristics determine whether a forecast supports reliable execution, regardless of its average error.

The key insight is that forecasts must be evaluated as operational inputs, not statistical artifacts. A readiness-based view explicitly distinguishes between forecasts that are numerically accurate and forecasts that are fit for deployment under asymmetric cost and limited tolerance for failure. Without this distinction, organizations optimize for accuracy while remaining exposed to the very risks forecasts are meant to manage.
