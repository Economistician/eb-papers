\section*{What This Enables}

Evaluating forecasts through the lens of readiness enables a more disciplined approach to deployment, monitoring, and governance. Rather than relying on ad hoc judgment or retrospective explanations, organizations can define explicit readiness standards aligned with operational risk tolerance and service objectives.

These standards support consistent approval and escalation decisions across models, products, and time periods. Changes in readiness can be monitored over time, providing early warning of degradation before service failures occur. When intervention is required, the underlying readiness dimensions make it clear whether the issue stems from reliability, severity, or stability, allowing responses to be targeted and proportionate.

More broadly, readiness-based evaluation establishes a common language between technical teams and operational leaders. Discussions shift away from abstract accuracy improvements toward concrete questions of service risk, tolerance, and economic exposure. This alignment improves trust in forecasting systems and clarifies accountability for deployment decisions.

By treating readiness as a governed evaluation objective rather than an implicit assumption, organizations lay the foundation for scalable and auditable forecast governance. This perspective supports more reliable execution today while enabling more advanced decision-support systems in the future, without conflating model development, control, and policy enforcement.
