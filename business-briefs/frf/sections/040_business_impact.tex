\section{Business Impact}

A readiness-based approach to forecast evaluation reduces operational risk that is often invisible under traditional accuracy metrics. By identifying forecasts that fail in service-critical intervals, organizations can prevent disruptions that would otherwise emerge only after deployment, when corrective action is costly and reactive.

The most immediate impact is improved service reliability. Forecasts that meet readiness criteria are less likely to produce deep or poorly timed shortfalls, reducing the frequency of service failures, recovery actions, and throughput loss. This stability translates into smoother execution during peak periods, when the cost of failure is highest.

Readiness-based evaluation also reduces hidden economic loss. While accuracy improvements may appear marginal, preventing a small number of severe under-forecasting events can materially reduce waste, rework, and operational firefighting. These gains are often realized not through higher average performance, but through the avoidance of high-impact failure modes that dominate operational cost.

Finally, readiness provides a more defensible basis for deployment decisions. Leaders can distinguish between forecasts that are statistically accurate and those that are operationally safe, enabling clearer trade-offs between efficiency and reliability. Over time, this leads to more consistent execution, lower volatility, and greater confidence in forecasting as a decision-support input rather than a source of recurring operational surprise.
