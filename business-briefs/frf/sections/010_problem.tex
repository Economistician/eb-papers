\section{The Problem We Are Actually Solving}

Operational forecasts are rarely used as predictions in isolation. They are used to decide whether sufficient product, labor, or capacity will be available at specific moments in time. In these settings, the cost of being wrong is not symmetric. Shortages disrupt service, force recovery actions, and create downstream volatility. Excess, by contrast, is often absorbed quietly.

Despite this reality, most forecasting systems are evaluated using symmetric accuracy metrics. These measures treat over-forecasting and under-forecasting as equivalent deviations from truth. As a result, forecasts that appear accurate on paper may still fail during the very intervals where operational reliability matters most.

This gap creates a false sense of confidence. Teams approve and deploy forecasting models based on aggregate accuracy, only to discover later that service failures cluster in critical periods. The issue is not that the forecasts are poorly built, but that the evaluation criteria do not reflect how forecasts are actually used in execution.

The real problem is therefore not forecast accuracy. It is whether a forecast is \emph{ready} to support operational decisions under asymmetric cost and limited tolerance for failure. Without an explicit way to assess readiness, organizations remain exposed to hidden service risk even as reported accuracy improves.
