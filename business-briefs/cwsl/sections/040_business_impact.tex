\section*{Business Impact}

When forecast evaluation fails to account for directional cost, the business systematically underestimates its true exposure. A small number of shortfall events—often concentrated in peak or constrained intervals—can drive a disproportionate share of operational loss, even when overall accuracy appears strong.

The impact shows up in multiple forms. Service failures reduce throughput and revenue in the moment, but they also trigger recovery actions that consume labor, capacity, and managerial attention. These recovery costs compound over time, introducing volatility into operations that is difficult to diagnose and even harder to eliminate after the fact.

By making directional cost explicit, a cost-weighted view of forecast error surfaces where loss is actually occurring. Preventing or mitigating a handful of severe under-forecasting events can deliver greater operational benefit than marginal improvements in average accuracy. In many environments, the return on avoiding one critical shortfall outweighs the benefit of dozens of small accuracy gains elsewhere.

Just as importantly, this perspective changes how performance improvements are interpreted. Instead of celebrating incremental accuracy gains that do not move operational outcomes, leaders can focus on reducing exposure to the failures that matter most. Over time, this leads to more stable execution, lower firefighting overhead, and forecasting systems that contribute directly to cost control and service reliability rather than obscuring their true economic impact.
