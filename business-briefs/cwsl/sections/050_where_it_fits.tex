\section*{Where CWSL Fits in Forecast Governance}

Cost-Weighted Service Loss is not intended to replace traditional accuracy metrics, nor to serve as a standalone decision rule. Its role is to complement existing evaluation by revealing the economic consequences of directional forecast error under explicit assumptions about operational cost.

Used correctly, CWSL provides a stable and interpretable signal for comparing forecasts in environments where under-forecasting and over-forecasting carry different consequences. It allows teams to distinguish between forecasts that are numerically accurate and those that are economically risky, without prescribing how forecasts should be translated into execution decisions.

CWSL is most effective when applied as part of a broader evaluation and governance process. Symmetric accuracy metrics continue to provide useful information about overall fit and stability, while frequency- and tolerance-based diagnostics explain how often and how severely failures occur. Within this context, CWSL serves as the economic lens that connects forecast error to operational loss.

By making cost assumptions explicit and auditable, CWSL also improves governance discipline. Changes in performance can be interpreted in terms of exposure rather than abstract error, and comparisons across models, products, or time periods can be made under a consistent evaluation contract. This clarity supports more defensible deployment decisions and reduces reliance on post hoc explanations when forecasts fail in practice.

Positioned this way, CWSL becomes a durable component of forecast evaluation rather than a transient metric. It helps organizations see where forecast error actually costs them money, while remaining compatible with broader readiness, monitoring, and decision-support frameworks.
