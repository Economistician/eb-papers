\section*{The Problem We’re Actually Paying For}

Forecast performance is often reported as an improvement story. Accuracy goes up, error goes down, and models appear to be getting better over time. Yet in many operational environments, these improvements do not translate into better outcomes. Service failures persist, recovery costs remain high, and operational teams continue to firefight during critical periods.

The disconnect lies in how forecast error is experienced versus how it is measured. In execution, not all errors are equal. Shortfalls lead to lost throughput, service degradation, and cascading recovery actions. Excess, by contrast, is frequently absorbed through buffers, slack capacity, or unused inventory. The business does not pay for forecast error symmetrically, even though it is evaluated that way.

As a result, organizations often pay a hidden cost. A small number of under-forecasting events—especially during peak or constrained intervals—can dominate operational loss, even when average accuracy looks strong. Because these failures are episodic rather than constant, they are easy to miss in aggregate metrics and post-hoc reporting.

The real problem is not that forecasts are inaccurate. It is that the way forecast error is evaluated obscures where cost actually comes from. Without a way to measure the economic impact of being wrong in the wrong direction, organizations continue to optimize accuracy while repeatedly paying for the same operational failures.
