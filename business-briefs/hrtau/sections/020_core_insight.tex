\section*{The Core Insight: Tolerance Must Be Calibrated, Not Assumed}

Tolerance defines the boundary between acceptable and unacceptable forecast error. As such, it should be treated as a calibrated readiness standard rather than an arbitrary constant.

The key insight is that tolerance selection can be grounded in historical forecast behavior without introducing model assumptions or external inputs. By examining how hit rates change as tolerance widens, organizations can make the dependence of readiness claims explicit and inspectable.

Framing tolerance as a calibration problem shifts the question from “What tolerance should we use?” to “Which tolerance reflects a defensible acceptability boundary given observed performance?” This distinction separates evaluation assumptions from evaluation outcomes.

Calibrated tolerance transforms hit-rate metrics from descriptive summaries into governed readiness signals. It ensures that readiness conclusions reflect explicit standards rather than implicit conveniences.
