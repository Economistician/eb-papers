\section*{Business Impact}

When tolerance is poorly specified, organizations experience inconsistent and unstable readiness signals. Forecasts may appear ready under one tolerance assumption and unready under another, creating confusion rather than clarity for decision-makers.

This instability erodes trust in forecasting systems. Teams compensate by discounting readiness metrics, introducing manual overrides, or reverting to informal judgment. Over time, the value of structured evaluation diminishes, and forecasting becomes less actionable.

Conversely, calibrated tolerance improves alignment between evaluation and execution. Readiness standards become explicit, comparable, and defensible. Changes in readiness reflect real shifts in forecast behavior rather than shifting assumptions.

By governing tolerance explicitly, organizations reduce ambiguity, improve auditability, and ensure that tolerance-based metrics support consistent, decision-aligned deployment choices.
