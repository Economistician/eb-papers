\section*{Why Fixed Thresholds Fail}

Fixed tolerance thresholds create an illusion of objectivity while concealing fragility. When tolerance is selected without sensitivity analysis, readiness conclusions may hinge on a narrow band of values where small changes produce large swings in hit rate.

This fragility is particularly problematic when tolerance thresholds are reused across models, entities, or time periods. Differences in error distributions mean that a single fixed threshold can be overly punitive in some contexts and trivially permissive in others.

Moreover, optimizing tolerance implicitly or implicitly widening it to improve reported performance undermines the credibility of readiness assessment. Because hit rate increases monotonically with tolerance, unconstrained adjustment is always possible and always misleading.

Without explicit calibration and sensitivity diagnostics, tolerance-based evaluation risks becoming a reporting artifact rather than a meaningful operational standard.
