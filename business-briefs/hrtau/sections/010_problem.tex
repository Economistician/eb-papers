\section*{The Problem: Tolerance Is Often Chosen Arbitrarily}

Tolerance-based metrics such as Hit Rate within Tolerance are appealing because they align naturally with operational thinking. A forecast is either close enough to use or it is not. However, this apparent simplicity conceals a critical weakness: the meaning of “close enough” is rarely defined in a disciplined or transparent way.

In practice, tolerance thresholds are often inherited from convention, chosen heuristically, or fixed without reference to historical forecast behavior. Small changes in these thresholds can materially alter reported performance, model rankings, and readiness conclusions, even when underlying forecasts are unchanged.

Because tolerance directly defines what counts as success or failure, its specification is not a technical detail. It is a policy decision. When tolerance is poorly governed, readiness assessment becomes unstable, subjective, and vulnerable to post hoc adjustment.

The core problem is that organizations frequently evaluate tolerance-based metrics without evaluating the tolerance itself. This leaves a critical assumption implicit, unexamined, and unauditable.
