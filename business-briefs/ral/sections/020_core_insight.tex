\section*{The Core Insight: Readiness Can Be Improved Without Changing the Forecast}

The key insight behind the Readiness Adjustment Layer is that improving execution does not always require improving prediction. In environments with asymmetric cost and constrained flexibility, modest, bounded adjustments applied at decision time can materially reduce readiness risk.

Rather than retraining models or modifying forecasting logic, RAL operates downstream of prediction. It treats the forecast as an input and applies a deterministic adjustment that reflects operational cost asymmetry and feasibility constraints present at execution time.

This reframing separates predictive responsibility from decision responsibility. Forecasting models remain accountable for estimating demand, while readiness adjustment becomes an explicit, governed control layer that aligns forecasts with executional priorities.

By making this adjustment explicit, bounded, and loss-governed, RAL replaces informal discretion with a transparent mechanism. It formalizes what operators often do implicitly, while preserving interpretability, auditability, and control.
