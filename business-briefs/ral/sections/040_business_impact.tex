\section*{Business Impact}

When readiness risk is left unaddressed, organizations experience recurring under-preparation despite acceptable forecast performance. This leads to lost revenue, service failures, emergency interventions, and erosion of trust in forecasting systems.

Informal mitigation often emerges in response. Operators apply buffers inconsistently, override forecasts manually, or rely on intuition rather than structured guidance. These practices reduce scalability, obscure accountability, and make outcomes difficult to audit or reproduce.

RAL replaces informal discretion with a controlled, auditable process. By applying bounded adjustments grounded in explicit cost and feasibility assumptions, it reduces under-readiness risk while preserving organizational control.

The result is improved execution stability without sacrificing transparency. Adjustments are explainable, repeatable, and reviewable, allowing organizations to improve operational outcomes while maintaining confidence in their forecasting and governance processes.
