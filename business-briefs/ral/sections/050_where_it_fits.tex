\section*{Where RAL Fits in Forecast Readiness}

The Readiness Adjustment Layer sits between forecast evaluation and operational execution. It consumes evaluated forecasts and readiness diagnostics, but does not reinterpret or override them. Its role is to translate assessment into controlled action.

Within the Forecast Readiness Framework, RAL operates only when structural compatibility has been established. Demand must be admissible under Demand Quantization Compatibility, and the forecast primitive must admit meaningful adjustment under Forecast Primitive Compatibility. Governance determines whether and how RAL is authorized to act.

RAL is not an optimization engine and not a substitute for modeling. It is a readiness-preserving control layer that applies bounded, deterministic adjustments under declared policy constraints. When these constraints prevent improvement, RAL explicitly does nothing.

By formalizing this adjustment step, RAL completes the readiness pipeline. It ensures that forecasts do not merely describe expected demand, but support execution decisions in a way that is consistent, auditable, and aligned with asymmetric operational risk.
