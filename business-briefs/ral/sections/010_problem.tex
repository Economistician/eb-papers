\section*{The Problem: Accurate Forecasts Can Still Fail at Execution}

Forecasting systems are typically evaluated on predictive accuracy, yet operational failures often persist even when forecasts are unbiased or well-calibrated. The issue is not that predictions are incorrect, but that decisions based on those predictions expose the system to asymmetric and irreversible risk.

In many operational environments, under-preparation carries a far greater penalty than modest over-preparation. Lost demand, service degradation, or capacity shortfalls cannot always be recovered after the fact, while excess capacity or inventory may be absorbed at lower cost. Accuracy-focused evaluation does not capture this imbalance.

Because forecasting models are often trained, validated, and deployed on cadences that do not align with real-time operational decisions, there may be limited opportunity to revise upstream predictions when risk becomes apparent. Decision-makers are left with forecasts that are technically sound but operationally fragile.

The problem is that traditional forecasting pipelines stop at prediction, leaving a gap between evaluation and execution. Without a controlled mechanism to translate forecasts into readiness-aware actions, even high-quality forecasts can produce systematically under-prepared outcomes.
