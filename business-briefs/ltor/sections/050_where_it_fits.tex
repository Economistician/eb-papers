\section*{Where LTOs Fit in Forecast Readiness}

Within the Forecast Readiness Framework, LTO handling is contextual rather than predictive. LTO activation enters the system as a governed signal that modulates readiness through the Readiness Adjustment Layer.

Structural admissibility is preserved through Demand Quantization Compatibility, adjustment allowability is constrained by Forecast Primitive Compatibility, and execution behavior is shaped by asymmetric loss under degraded readiness.

Importantly, LTO context does not override forecasts. It alters how forecasts are evaluated and selected, ensuring that production behavior reflects known risk rather than speculative belief.

Positioned this way, LTO management becomes auditable, repeatable, and robust. Planned instability is acknowledged explicitly, and operational decisions remain aligned with asymmetric risk even when forecasts are known to be unreliable.
