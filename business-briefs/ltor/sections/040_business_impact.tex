\section*{Business Impact}

When LTOs are governed through belief-driven production, organizations experience volatile outcomes. Stockouts during peaks, excess production during lulls, and oscillatory adjustments erode service quality and operational confidence.

Teams respond by overriding forecasts, adding informal buffers, or chasing accuracy mid-launch. These interventions reduce scalability and obscure accountability. Postmortems often blame forecasting accuracy without distinguishing between modeling error and inappropriate risk posture.

A readiness-centric approach stabilizes execution. Conservative bias is applied intentionally during high-risk windows, then relaxed as empirical demand signals resolve uncertainty. Production adapts without committing to fragile assumptions.

The result is improved preparedness during launches, reduced late-day shortfall, and clearer attribution of outcomes—regardless of whether hype materializes as expected.
