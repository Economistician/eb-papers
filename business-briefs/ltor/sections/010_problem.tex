\section*{The Problem: LTOs Break Forecast Assumptions}

Limited-Time Offers are designed to disrupt demand. They introduce novelty, marketing pressure, substitution effects, and execution variability that intentionally break steady-state behavior. As a result, the assumptions under which forecasting systems are typically calibrated no longer hold.

Despite this, LTO periods are often treated as standard forecasting problems. Demand is adjusted upward using promotional lift estimates, and production decisions are coupled directly to these speculative predictions. When realized demand diverges from expectation—as it frequently does—operations absorb the consequences.

The failure is not a lack of analytical effort. It is a mismatch between how LTOs behave and how forecasting systems are asked to support execution. Accuracy degrades structurally during promotions, while the cost of being wrong simultaneously increases.

The core problem is that LTOs are planned regime shifts, yet they are governed as if they were ordinary demand variation.
