\section*{The Core Insight: LTOs Are Readiness Shocks, Not Forecasting Problems}

LTO activation conveys reliable information, even when demand magnitude is uncertain. It signals elevated variance, weakened calibration, and heightened asymmetry in operational risk. These properties hold regardless of whether promotional lift is large or small.

The key insight is that LTOs should be treated as exogenous readiness shocks. Rather than embedding assumptions about uplift into demand forecasts, the system should adjust how forecasts are trusted and acted upon during promotional windows.

By separating prediction from readiness posture, operations can respond conservatively to known instability without asserting specific beliefs about demand. This reframing preserves robustness when forecasts are least reliable.

Electric Barometer operationalizes this insight by allowing LTO context to degrade readiness through governed policy, rather than altering forecasts directly.
