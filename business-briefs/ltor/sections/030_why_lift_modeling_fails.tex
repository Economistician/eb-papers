\section*{Why Lift Modeling Fails at Execution Time}

Promotional lift models are inherently speculative at the intraday level. Realized demand depends on execution quality, local conditions, substitution across menu items, and supply constraints—many of which are only partially observable at decision time.

When production decisions are tied directly to uplift estimates, belief error propagates immediately into waste or stockouts. Corrections often arrive too late to prevent service degradation, especially during peak promotional windows.

Intraday production is fast, partially reversible, and continuously informed by realized demand. In this context, speculative belief adds less value than controlled risk posture. Modeling effort that is appropriate for long-horizon planning becomes brittle when applied to real-time control.

Decoupling production from uplift belief does not abandon forecasting. It recognizes that execution requires different guarantees than prediction.
