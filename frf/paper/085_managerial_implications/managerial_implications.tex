\section{Managerial Implications}

The Forecast Readiness Framework is designed not only as an evaluative tool for analysts, but
as a decision-support structure for operational managers and governance bodies responsible
for deploying forecasting systems. By reframing forecast evaluation around readiness rather
than numerical accuracy alone, the framework provides a common language for discussing
risk, service reliability, and economic consequence across technical and operational teams.

From a governance perspective, the diagnostic decomposition of readiness supports more
transparent deployment decisions. Rather than relying on a single accuracy threshold, managers
can assess whether a forecast meets minimum standards for service reliability, failure severity,
tolerance stability, and cost alignment. This enables the definition of readiness criteria that are
explicitly linked to operational priorities, such as acceptable shortfall frequency or maximum
tolerable service loss, and reduces reliance on ad hoc judgment.

\subsection{Implications for Deployment Governance and Threshold Setting}

A central managerial implication of the Forecast Readiness Framework is its support for
explicit deployment governance. Rather than approving or rejecting forecasting systems
based solely on aggregate accuracy metrics, managers can define minimum readiness standards
aligned with operational risk tolerance. For example, a forecasting system may be required to
meet a minimum reliability threshold, remain within tolerance bounds for a specified fraction
of intervals, or limit cost-weighted service loss below an acceptable level before being approved
for operational use.

This threshold-based approach enables more consistent and defensible deployment decisions.
Forecasts that achieve high numerical accuracy but exhibit frequent shortfalls or unstable
behavior can be identified as unready, while forecasts with modest average error but stable
and reliable performance may be deemed suitable for execution. Importantly, readiness
thresholds can be adapted to context, allowing different standards for peak periods,
critical items, or constrained operational environments.

The framework also supports graduated escalation rather than binary approval decisions.
Early warning signals—such as declining tolerance stability or increasing shortfall depth—
can trigger targeted interventions, including model retraining, buffer adjustments, or
temporary operational safeguards. By distinguishing among failure modes, the framework
helps managers respond proportionately to forecast degradation rather than reacting only
after service failures occur.

The composite readiness signal provided by the framework further supports monitoring and
escalation. Changes in overall readiness can be tracked over time, while shifts in individual
diagnostics indicate whether deterioration arises from increased shortage risk, deeper failures,
or reduced stability. This decomposition allows managers to distinguish between routine model
drift and structurally concerning changes that warrant intervention, retraining, or operational
mitigation.

More broadly, the framework encourages a shift in how forecasting performance is communicated
to stakeholders. By grounding evaluation in readiness for execution, the Forecast Readiness
Framework aligns technical assessment with operational outcomes, supporting clearer trade-offs
between efficiency and reliability. In doing so, it provides a principled basis for integrating
forecast evaluation into operational planning, governance, and continuous improvement
processes.