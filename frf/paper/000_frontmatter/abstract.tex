\begin{abstract}
Forecast performance in operational environments is commonly evaluated using symmetric
accuracy metrics such as MAE, RMSE, and MAPE. These measures implicitly assume that
over-forecasting and under-forecasting impose equivalent cost, an assumption that rarely holds
in high-frequency systems where shortages generate substantially greater operational disruption
than excess. As a result, forecasts that appear accurate by traditional metrics may still fail to
support reliable execution.

We introduce the \emph{Forecast Readiness Framework} (FRF), a multi-dimensional approach
for evaluating forecast performance under asymmetric error structures. Rather than relying on
a single loss function, the framework decomposes readiness into complementary dimensions
capturing service reliability, failure severity, tolerance stability, and economic consequence.
Within this framework, Cost-Weighted Service Loss (CWSL) serves as the primary economic
axis, explicitly quantifying the asymmetric operational impact of forecast error, while supporting
diagnostics characterize how frequently shortfalls occur, how severe they are when they arise,
and whether deviations fall within operationally absorbable bounds.

By reframing forecast evaluation around readiness for deployment rather than numerical accuracy
alone, the Forecast Readiness Framework provides a structured basis for selecting, monitoring,
and governing forecasting systems in environments where service reliability and asymmetric
cost are central concerns. The framework is applicable across a range of operational domains,
including production planning, staffing, inventory management, logistics, and other short-horizon
decision systems.
\end{abstract}