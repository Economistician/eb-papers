\section{Limitations and Scope}

The Forecast Readiness Framework is intended as an evaluative and governance-oriented structure
for forecasting systems operating under asymmetric error cost. Its applicability is therefore
subject to several deliberate limitations that clarify scope and prevent misinterpretation of its
intended use.

First, the framework depends on the specification of asymmetric cost and tolerance assumptions,
particularly within the \CWSL\ and \HRtau\ components. These assumptions encode how forecast
errors are valued once they interact with real operational systems, and they may vary across
organizations, products, or decision contexts. While precise monetary calibration is not required,
poorly governed assumptions can misrepresent true operational priorities. For this reason, the
\FRF\ treats such parameters as governed evaluation inputs, to be inspected through sensitivity
analysis, calibrated using historical behavior, and reviewed explicitly rather than selected
implicitly or optimized post hoc.

Second, the \FRF\ is designed for forecast evaluation, comparison, and deployment governance
rather than model training or optimization. Although its diagnostics may inform modeling choices,
the framework does not prescribe how forecasts should be generated, nor does it define objectives
to be minimized during model fitting. In particular, the composite \FRS\ functions as an
interpretive readiness signal and governance aid, not as a loss function or control target.

Third, the framework evaluates forecast performance conditional on realized demand and focuses
on point-forecast behavior. It does not explicitly represent predictive uncertainty or probabilistic
calibration. In environments where full predictive distributions are available, readiness
assessment may benefit from complementary probabilistic diagnostics. The \FRF\ does not
replace such approaches, but addresses a distinct evaluative need centered on execution
feasibility, error structure, and asymmetric operational cost.

Finally, the framework abstracts from certain system-level dynamics, such as downstream
dependencies, feedback effects, or cross-entity capacity interactions. While the diagnostic
dimensions capture critical aspects of readiness at the forecast level, additional analysis may be
required to assess system-wide performance in highly coupled or adaptive operational settings.

These limitations reflect intentional design boundaries rather than deficiencies. By focusing on
deployment readiness under asymmetric and explicitly governed evaluation assumptions, the
\FRF\ addresses a specific and practical gap in forecast evaluation while remaining compatible
with broader modeling, optimization, and decision-support methodologies.
