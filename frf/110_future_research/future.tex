\section{Future Research Directions}

The Forecast Readiness Framework opens several avenues for future research at the intersection
of forecasting, operations, and decision support. One natural extension concerns the systematic
elicitation and learning of asymmetric penalty structures used within the \CWSL\ component.
While the framework treats penalty ratios as externally specified, future work may explore
data-driven or adaptive approaches for inferring operational cost asymmetry from observed
decisions, service outcomes, or recovery behavior.

A second direction involves integrating readiness-based evaluation with probabilistic forecasting.
The current framework focuses on realized demand and point forecasts, emphasizing execution
outcomes rather than uncertainty representation. Extending readiness diagnostics to incorporate
predictive distributions, coverage properties, or tail risk measures may further enhance
deployment decisions in environments where probabilistic forecasts are available.

Future research may also examine the use of readiness diagnostics in model development and
selection workflows. Although the \FRF\ is not designed as an optimization objective, its components
may inform training strategies, regularization schemes, or ensemble construction that explicitly
balance accuracy, reliability, and stability. Understanding how readiness-aware evaluation shapes
model behavior remains an open and promising area of investigation.

Finally, the framework may be extended to more complex operational settings involving dynamic
feedback, multi-stage decisions, or interactions across items and locations. Studying how readiness
diagnostics aggregate or propagate in such coupled systems could broaden the applicability of
the \FRF\ to networked and large-scale operational environments.

Together, these directions highlight the potential for readiness-based evaluation to inform both
the theory and practice of forecasting in asymmetric operational contexts, providing a foundation
for continued methodological and applied research.