\section{Managerial Implications}

The Forecast Readiness Framework is designed not only as an evaluative tool for analysts, but
as a decision-support structure for operational managers and governance bodies responsible
for deploying forecasting systems. By reframing forecast evaluation around readiness rather
than numerical accuracy alone, the framework provides a common language for discussing
risk, service reliability, and economic consequence across technical and operational teams.

From a governance perspective, the diagnostic decomposition of readiness supports more
transparent deployment decisions. Rather than relying on a single accuracy threshold, managers
can assess whether a forecast meets minimum standards for service reliability, failure severity,
tolerance stability, and cost alignment. This enables the definition of readiness criteria that are
explicitly linked to operational priorities, such as acceptable shortfall frequency or maximum
tolerable service loss, and reduces reliance on ad hoc judgment.

The composite readiness signal provided by the framework further supports monitoring and
escalation. Changes in overall readiness can be tracked over time, while shifts in individual
diagnostics indicate whether deterioration arises from increased shortage risk, deeper failures,
or reduced stability. This decomposition allows managers to distinguish between routine model
drift and structurally concerning changes that warrant intervention, retraining, or operational
mitigation.

More broadly, the framework encourages a shift in how forecasting performance is communicated
to stakeholders. By grounding evaluation in readiness for execution, the Forecast Readiness
Framework aligns technical assessment with operational outcomes, supporting clearer trade-offs
between efficiency and reliability. In doing so, it provides a principled basis for integrating
forecast evaluation into operational planning, governance, and continuous improvement
processes.