\section{The Forecast Readiness Framework}

In asymmetric operational environments, forecast evaluation cannot be reduced to a single notion
of accuracy. Forecasts are used to support execution under constraints, uncertainty, and time
pressure, and different patterns of error impose qualitatively different operational consequences.
A forecast that occasionally produces deep shortages poses a different type of risk than one that
is consistently biased but stable, even if both achieve similar aggregate accuracy. Evaluating such
forecasts therefore requires a framework that distinguishes among the dimensions of error that
matter for operational readiness.

The \emph{Forecast Readiness Framework} (FRF) formalizes this perspective by decomposing
readiness into a small set of complementary diagnostic dimensions. Each dimension captures a
distinct aspect of how forecast error affects execution, and no single dimension is sufficient on
its own. Together, these diagnostics provide a structured basis for assessing whether a forecast is
fit for deployment in environments where under-forecasting and over-forecasting have asymmetric
cost.

\subsection{Readiness as a Multi-Dimensional Construct}

Forecast readiness is defined here as the degree to which a forecast reliably supports operational
execution under asymmetric error cost. This definition emphasizes deployment suitability rather
than numerical precision. A forecast may achieve low average error while still failing to meet
readiness requirements if its errors are concentrated in the wrong direction, occur too frequently,
or exceed the system’s ability to absorb deviations.

Within FRF, readiness is decomposed into four primary dimensions:
\begin{enumerate}
    \item \textbf{Service reliability}: how often the forecast avoids shortfalls entirely.
    \item \textbf{Failure severity}: how large shortfalls are when they occur.
    \item \textbf{Tolerance stability}: how frequently deviations fall within operationally
    acceptable bounds.
    \item \textbf{Economic consequence}: the cost-weighted impact of directional forecast error.
\end{enumerate}

Figure~\ref{fig:frf_overview} summarizes the structure of the \FRF\ and the relationship between
its diagnostic dimensions, the economic axis of evaluation, and the composite readiness signal
used to support deployment and governance decisions.

% frf/figures/frf_overview.tex
\begin{figure}[t]
\centering
\resizebox{\linewidth}{!}{%
\begin{tikzpicture}[
  font=\small,
  box/.style={draw, rounded corners, align=center, inner sep=8pt, text width=0.90\linewidth},
  sbox/.style={draw, rounded corners, align=center, inner sep=6pt, text width=0.27\linewidth, font=\scriptsize},
  arrow/.style={-Latex, thick},
  node distance=10pt
]

% --- Main spine ---
\node[box] (forecasts) {\textbf{Forecasts}\\[2pt]
Point or probabilistic predictions used for execution};

\node[box, below=10pt of forecasts] (error) {\textbf{Forecast Error}\\[2pt]
Directional and time-varying deviations relative to realized demand};

% --- SPACER: forces everything below to shift down ---
\node[inner sep=0pt, minimum height=14pt, below=0pt of error] (spacer) {};

% --- Diagnostics row ---
\node[sbox, below=14pt of spacer] (ud)
{\textbf{Severity}\\(\UD)\\[2pt]
Shortfall depth};

\node[sbox, left=6pt of ud] (nsl)
{\textbf{Reliability}\\(\NSL)\\[2pt]
Shortfall incidence};

\node[sbox, right=6pt of ud] (hrtau)
{\textbf{Tolerance Stability}\\(\HRtau)\\[2pt]
Violations beyond tolerance};

% --- Full-width diagnostics container ---
\node[
  draw, rounded corners,
  fit=(nsl)(ud)(hrtau),
  inner sep=18pt
] (diag) {};

% Title INSIDE the diagnostics box
\node[font=\small\bfseries, align=center] at ([yshift=-16pt]diag.north)
{Error Diagnostics (FRF Dimensions)\\[2pt]};

% --- Remaining boxes ---
\node[box, below=14pt of diag] (cwsl) {\textbf{Economic Impact}\\(\CWSL)\\[2pt]
Asymmetric cost aggregation of forecast error};

\node[box, below=10pt of cwsl] (frs) {\textbf{Composite Readiness Signal}\\(\FRS)\\[2pt]
Deployment-oriented synthesis of diagnostics};

\node[box, below=10pt of frs] (gov) {\textbf{Deployment \& Governance}\\[2pt]
Model selection, threshold monitoring, risk escalation, operational approval};

% --- Arrows ---
\draw[arrow] (forecasts) -- (error);
\draw[arrow] (error) -- (diag.north);
\draw[arrow] (diag.south) -- (cwsl);
\draw[arrow] (cwsl) -- (frs);
\draw[arrow] (frs) -- (gov);

\end{tikzpicture}%
}
\caption{\textbf{Structure of the Forecast Readiness Framework (FRF).}
Forecast performance is evaluated across complementary diagnostic dimensions capturing reliability, severity,
and tolerance stability. These diagnostics are integrated with asymmetric economic impact through \CWSL\ and
synthesized into a composite readiness signal (\FRS) to support deployment and governance decisions.}
\label{fig:frf_overview}
\end{figure}

Each dimension captures information that is obscured when performance is summarized by a
single symmetric metric. For example, two forecasts with similar average error may differ sharply
in the frequency of shortfalls, the depth of those shortfalls, or the economic burden they impose
during peak intervals. FRF is designed to make these distinctions explicit.

\subsection{Diagnostic Components of the Framework}

FRF operationalizes the four readiness dimensions using a set of interpretable metrics. Service
reliability is measured by the \emph{No--Shortfall Level} (NSL), which quantifies the proportion of
intervals in which forecasted demand meets or exceeds realized demand. NSL isolates the
frequency of service-risk events without regard to their magnitude.

Failure severity is captured by \emph{Underbuild Depth} (UD), which measures the average
magnitude of shortfalls conditional on a shortfall occurring. UD distinguishes between forecasts
that miss frequently but shallowly and those that occasionally miss by large amounts, a distinction
with important implications for recovery dynamics and throughput loss.

Tolerance stability is measured by the \emph{Hit Rate within Tolerance} (\HRtau), which
quantifies the proportion of intervals in which forecast error falls within an operationally acceptable
bound. This metric reflects the fact that many operational systems can absorb small deviations
without degradation, and that readiness depends not on exact accuracy but on remaining within
decision-relevant tolerances.

Economic consequence is captured by \emph{Cost-Weighted Service Loss} (CWSL), which aggregates
directional forecast error using asymmetric penalties for shortfalls and overbuilds and normalizes
the result by realized demand. CWSL provides a cost-aligned measure of how forecast error
translates into effective operational loss under a specified penalty structure.

Each of these diagnostics isolates a distinct behavioral property of forecast error. None is intended
to stand alone, and each addresses a failure mode that symmetric accuracy measures systematically
obscure.

\subsection{CWSL as the Economic Axis of Readiness}

Among the diagnostic components, CWSL plays a central role by providing an explicit economic
lens on forecast performance. Whereas NSL, UD, and \HRtau describe the structure and
frequency of error, CWSL translates that structure into cost-weighted consequence. This makes
CWSL the primary axis along which asymmetric operational risk is expressed within the framework.

Importantly, CWSL does not replace the other diagnostics. Two forecasts may exhibit similar
cost-weighted loss while differing substantially in the frequency or severity of shortfalls, leading
to different operational responses. Conversely, forecasts with similar reliability profiles may
produce very different economic impact depending on the depth and timing of errors. FRF
therefore treats CWSL as foundational but not sufficient, embedding it within a broader diagnostic
context.

\subsection{Framework Interpretation and Use}

The Forecast Readiness Framework is intended to support evaluation, comparison, and governance
of forecasting systems rather than optimization alone. By examining multiple dimensions of
readiness simultaneously, practitioners can identify why a forecast fails to support execution,
distinguish among competing failure modes, and select models whose error structure aligns with
operational priorities.

At an aggregate level, the framework supports comparison across models, items, locations, or
time periods using a consistent set of diagnostics. At a diagnostic level, it highlights whether poor
performance arises from frequent shortfalls, severe shortfalls, instability relative to tolerance, or
misalignment between error structure and cost asymmetry. This decomposition is essential for
targeted intervention, model refinement, and informed trade-offs between service reliability and
efficiency.

By reframing forecast evaluation around readiness rather than numerical accuracy alone, the
Forecast Readiness Framework provides a principled and operationally grounded basis for assessing
forecast performance in asymmetric environments.