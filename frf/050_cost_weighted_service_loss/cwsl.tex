\section{Cost-Weighted Service Loss as the Economic Axis}

Among the diagnostic components of the \FRF, Cost-Weighted Service Loss (\CWSL) provides
the primary economic perspective on forecast performance. While other diagnostics describe
the frequency, severity, and stability of forecast error, \CWSL\ explicitly quantifies how
directional deviations translate into operational cost under asymmetric error structures. In
environments where shortages impose substantially greater burden than excess, this cost
alignment is essential for evaluating readiness.

\CWSL\ measures forecast error by decomposing deviations into shortfall and overbuild
components and applying asymmetric penalties to each. Shortfalls are penalized more heavily
than overbuilds, reflecting their disproportionate impact on service reliability, throughput,
and recovery dynamics. The resulting penalties are aggregated across items and evaluation
intervals and normalized by realized demand, yielding a scale-free measure that is comparable
across locations, products, and time periods. Interpreted directly, \CWSL\ answers the
question: \emph{what fraction of total demand was effectively lost due to the cost-weighted
impact of forecast error?}

A key feature of \CWSL\ is its interval-level granularity. Forecast deviations are evaluated at
the same temporal resolution at which operational decisions are made, preserving the timing
and directional structure of error. This prevents benign overbuilds from offsetting critical
shortfalls that occur during high-impact intervals, a common failure mode of symmetric
accuracy metrics. By normalizing cost-weighted penalties by total demand, \CWSL\ further
ensures that performance is driven by operationally meaningful volume rather than by
low-impact noise.

Penalty weights within \CWSL\ encode the relative operational cost of shortfalls and
overbuilds. These parameters need not correspond to precise monetary values; rather, they
summarize the organization’s tolerance for shortage risk relative to excess. In practice,
penalty ratios may be elicited through structured operational judgment or explored through
sensitivity analysis to assess the robustness of conclusions across plausible asymmetry
assumptions. This flexibility allows \CWSL\ to adapt to heterogeneous operational contexts
while retaining a consistent evaluative interpretation.

Within the \FRF, \CWSL\ functions as the economic axis rather than a standalone evaluation
criterion. Two forecasts may exhibit similar cost-weighted loss while differing substantially
in the frequency or severity of shortfalls, leading to different operational responses.
Conversely, forecasts with similar reliability profiles may impose very different economic
burden depending on the depth and timing of deviations. For this reason, the \FRF\ embeds
\CWSL\ alongside complementary diagnostics that isolate reliability, severity, and tolerance
behavior.

The formal definition, properties, and illustrative examples of \CWSL\ are provided in a
companion technical note. In the present framework, \CWSL\ serves as the mechanism by
which asymmetric operational cost is incorporated into readiness evaluation, ensuring that
forecast assessment reflects not only how forecasts deviate from realized demand, but how
those deviations affect execution in practice.