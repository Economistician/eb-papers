% ----------------------------------------------------------
% PROPERTIES AND BEHAVIOR
% ----------------------------------------------------------
\section{Properties and Behavior}

\subsection{Boundedness and Interpretability}
HR@\(\tau\) is a proportion and therefore lies within the closed interval
\([0,1]\). This boundedness makes the metric easy to interpret: values near 1
indicate that the forecast satisfies the tolerance condition in most intervals,
whereas values near 0 indicate frequent operationally significant deviations.

\subsection{Dependence on the Tolerance Parameter}
The tolerance level \(\tau\) determines which intervals qualify as hits. Larger
values of \(\tau\) classify more intervals as within tolerance, yielding higher
HR@\(\tau\) values; smaller values impose stricter requirements and reduce the hit
rate. Interpretation of HR@\(\tau\) therefore depends on the operational meaning
of \(\tau\), which may vary across domains or entities. Selecting an appropriate
tolerance is critical for ensuring that the metric reflects true operational
capability rather than an artifact of parameter choice.

\subsection{Sensitivity to the Error Distribution}
HR@\(\tau\) depends on the frequency with which absolute forecast errors fall
inside the tolerance band. Systems that produce stable, low-variance errors
typically yield high hit rates. In contrast, systems characterized by volatility,
intermittent spikes, or heavy-tailed error distributions may exhibit lower hit
rates even when their average accuracy is comparable. HR@\(\tau\) therefore
provides a complementary perspective to magnitude-based accuracy metrics.

\subsection{Symmetric Treatment of Over- and Underestimation}
Because HR@\(\tau\) is defined in terms of absolute error, it treats positive and
negative deviations symmetrically. The metric is therefore most appropriate in
environments where overforecasting and underforecasting have similar operational
consequences. If asymmetries exist—such as environments where underbuilding is
more costly than overbuilding—HR@\(\tau\) should be interpreted alongside
directionally sensitive diagnostics.

\subsection{Scale and Magnitude Considerations}
The meaningfulness of a fixed tolerance level depends on the magnitude of the
underlying demand. A tolerance that is reasonable for a high-volume entity may be
too permissive or too restrictive for a low-volume one. When comparing entities or
aggregating results across heterogeneous units, the tolerance parameter may
require normalization or entity-specific calibration to preserve interpretability.

\subsection{Sampling Variability}
As an empirical proportion computed over a finite number of intervals, HR@\(\tau\)
is subject to sampling variability. Standard binomial methods may be used to
construct confidence intervals, evaluate statistical differences between
forecasting systems, or assess the reliability of HR@\(\tau\) estimates in
short-horizon evaluation settings.