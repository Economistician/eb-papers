% ----------------------------------------------------------
% OPERATIONAL MOTIVATION
% ----------------------------------------------------------
\section{Operational Motivation}

In many operational environments, forecasts do not need to match realized demand
exactly. Instead, they must remain within a range that allows planned execution to
proceed without disruption. Small deviations are often absorbed naturally through
batching rules, resource granularity, service-time flexibility, or built-in
buffers. These mechanisms of operational tolerance and absorption are well
established in classical production and inventory theory
\citep{silver1998inventory}. Only errors that exceed a meaningful operational
threshold risk creating delays, inefficiencies, or degraded service levels.

HR@\(\tau\) captures this perspective by measuring how often forecast deviations
remain within the zone of operational adequacy. A high HR@\(\tau\) value indicates
that the system can typically follow its planned trajectory without requiring
corrective action, while a low value signals more frequent intervals in which
forecast errors challenge execution or elevate risk.

Because the tolerance level \(\tau\) is defined with respect to the operational
context—such as capacity adjustment increments, inventory sensitivities, labor
scheduling granularity, or workflow rigidity—HR@\(\tau\) serves as a tailored,
decision-oriented measure of practical forecast reliability. It highlights whether
a forecasting system meets the precision standards that matter in real-world
applications, rather than an abstract statistical notion of accuracy.