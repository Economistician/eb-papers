% ----------------------------------------------------------
% CONCLUSION
% ----------------------------------------------------------
\section{Conclusion}

Hit Rate within Tolerance (HR@\(\tau\)) provides a clear and operationally grounded
measure of how often forecast errors remain small enough to support stable
execution. By evaluating accuracy relative to an application-specific tolerance
threshold, the metric emphasizes practical reliability rather than strict numerical
precision. This perspective aligns with many real-world environments in which
modest deviations are inconsequential but larger errors can disrupt workflows or
increase operational risk.

HR@\(\tau\) is most informative when interpreted alongside complementary metrics
that capture magnitude, directionality, variance, or tail behavior in forecast
errors. Together, these measures offer a more complete understanding of
forecasting system performance and its implications for downstream operations.

Within the Forecast Readiness Framework, HR@\(\tau\) serves as a simple,
interpretable indicator of how consistently a forecasting system meets the
tolerance requirements of the domain. Its emphasis on operational adequacy rather
than strict statistical precision aligns with broader movements in forecasting
research that prioritize practical relevance, robustness, and real-world decision
support \citep{makridakis2018m4}. These qualities make HR@\(\tau\) a valuable
diagnostic in interval-based forecasting contexts.