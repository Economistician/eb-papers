% ----------------------------------------------------------
% DEFINITION
% ----------------------------------------------------------
\section{Definition and Mathematical Formulation}

Let \(e\) denote the entity being forecast (e.g., an item, workload, or capacity
stream), and let \(T\) represent the set of evaluation intervals. For each
interval \(t \in T\), let \(y_{et}\) denote realized demand and \(\hat{y}_{et}\)
the corresponding forecast. The interval-level forecast error is
\[
e_{et} = \hat{y}_{et} - y_{et}.
\]

Given a tolerance level \(\tau > 0\), an interval is classified as a
\emph{hit} if the absolute forecast error lies within the tolerance band:
\[
|e_{et}| \le \tau.
\]

The Hit Rate within Tolerance for entity \(e\) is then defined as
\[
\mathrm{HR@}_{\tau,e}
=
\frac{
\left| \{\, t \in T : |e_{et}| \le \tau \, \} \right|
}{
|T|
}.
\]

By construction, the metric lies in the interval \([0,1]\). Values near 1
indicate that the forecast remains within the acceptable tolerance band for most
intervals, while lower values reflect more frequent deviations. Because the
tolerance level \(\tau\) represents the degree of precision required for stable
operations, HR@\(\tau\) serves as an empirical measure of how consistently the
forecast meets these practical requirements.