% ----------------------------------------------------------
% USE CASES
% ----------------------------------------------------------
\section{Use Cases Across Operational Domains}

HR@\(\tau\) is especially useful in environments where modest forecast deviations
do not materially disrupt execution, but larger deviations trigger inefficiency,
risk, or the need for corrective action. Because the metric focuses on the
frequency of acceptable performance rather than the magnitude of errors,
HR@\(\tau\) provides an intuitive and operationally grounded diagnostic across a
broad range of domains.

\subsection{Production and Manufacturing}
Batching constraints, equipment cycle times, and workflow granularity often allow
systems to absorb small discrepancies between forecasted and realized demand.
HR@\(\tau\) quantifies how reliably a forecasting system remains within these
operational tolerances, supporting decisions related to scheduling, material
planning, and production sequencing.

\subsection{Retail and Replenishment}
Small forecast deviations may not meaningfully affect ordering behavior or stock
availability. By measuring how often forecasts fall within an acceptable margin of
error, HR@\(\tau\) identifies when the system is sufficiently accurate to avoid
reactive adjustments or unnecessary safety stock expansions.

\subsection{Service and Staffing}
Labor cannot typically be adjusted at single-unit resolution. Slight workload
misalignment can often be absorbed within shift structures or service buffers.
HR@\(\tau\) therefore helps assess whether demand forecasts are reliable enough to
maintain staffing efficiency without excessive understaffing or overstaffing risk.

\subsection{Logistics and Transportation}
Parcel volumes, passenger counts, and routing needs often tolerate modest
variations. HR@\(\tau\) reflects how consistently forecasts stay within
operationally manageable bounds, informing decisions on fleet allocation, routing
stability, and network planning.

\subsection{Energy and Utilities}
Short-term load forecasts frequently allow for some deviation before requiring
corrective action. HR@\(\tau\) indicates how often the forecasting system stays
within this acceptable operating window, supporting decisions around dispatch,
capacity planning, and reliability assurance.

\subsection{Digital and Online Systems}
Traffic forecasting for applications, APIs, and digital services often includes
built-in elasticity or buffering. HR@\(\tau\) highlights whether demand forecasts
stay within tolerance levels that avoid throttling, latency degradation, or
unnecessary autoscaling events.