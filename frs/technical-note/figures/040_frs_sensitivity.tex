\begin{figure}[htbp]
\centering
\begin{tikzpicture}
\begin{groupplot}[
    group style={group size=2 by 1, horizontal sep=1.2cm},
    width=0.46\linewidth,
    height=0.38\linewidth,
    xmin=1, xmax=8,
    ymin=8, ymax=16,
    xlabel={$t$},
    ylabel={Quantity},
    axis lines=left,
    grid=both,
    minor tick num=1,
    tick align=outside,
    legend style={draw=none, fill=none, at={(0.02,0.98)}, anchor=north west},
]

\nextgroupplot[title={Forecast A (higher NSL)}]
\addplot+[thick, mark=*] coordinates {(1,10) (2,12) (3,11) (4,9) (5,13) (6,10) (7,14) (8,12)};
\addlegendentry{Actual $y_t$}
\addplot+[thick, mark=square*] coordinates {(1,10.8) (2,12.3) (3,11.2) (4,9.6) (5,13.4) (6,10.4) (7,14.2) (8,12.5)};
\addlegendentry{Forecast $\hat{y}_t$}

\nextgroupplot[title={Forecast B (lower NSL)}]
\addplot+[thick, mark=*] coordinates {(1,10) (2,12) (3,11) (4,9) (5,13) (6,10) (7,14) (8,12)};
\addlegendentry{Actual $y_t$}
\addplot+[thick, mark=square*] coordinates {(1,9.2) (2,11.0) (3,10.2) (4,8.6) (5,12.0) (6,9.3) (7,13.0) (8,11.1)};
\addlegendentry{Forecast $\hat{y}_t$}

\end{groupplot}
\end{tikzpicture}
\caption{Illustration of FRS sensitivity to error type. Two forecasts can exhibit similar symmetric accuracy (e.g., RMSE/MAE) while differing materially in shortfall behavior and asymmetric cost exposure, producing different FRS values.}
\label{fig:frs_sensitivity}
\end{figure}