% ----------------------------------------------------------
% OPERATIONAL MOTIVATION
% ----------------------------------------------------------
\section{Operational Motivation}

Operational forecasting differs fundamentally from statistical forecasting. While
classical accuracy metrics quantify numerical deviation, operational performance
depends on whether a forecast enables the system to function reliably, safely, and
efficiently under real conditions. In many production, staffing, retail, and
short-horizon service environments, the consequences of forecast error are highly
asymmetric and can produce cascading effects on service quality and cost.

Shortfalls create immediate operational risk: service failures, queueing
instability, lost throughput, labor inefficiency, and degraded customer
experience. Overbuild, while typically less harmful, still generates waste,
excess labor utilization, spoilage, and reduced flexibility. Because these
consequences differ in severity, a metric that treats all numerical deviations
symmetrically does not reflect operational priorities.

The Forecast Readiness Score (FRS) addresses this gap by combining two
complementary signals:

\begin{itemize}
    \item \textbf{NSL}, which captures how often the forecast avoids shortfalls,
    reflecting a frequency-based notion of service protection, and
    \item \textbf{scaled CWSL}, which quantifies the asymmetric economic cost of
    forecast deviations, normalized to a stable and interpretable range.
\end{itemize}

These components reflect the expectations of operators and managers:

\begin{enumerate}
    \item \textit{Shortfalls must be rare.} Even a small number of shortfall
    events can destabilize workflows, reduce service quality, and trigger costly
    corrective action. NSL captures this requirement.

    \item \textit{Not all deviations matter equally.} Shortfalls and overbuild
    incur different operational and economic penalties. CWSL incorporates these
    asymmetries by assigning different weights (\cu, \co) to each error type.

    \item \textit{Readiness requires both reliability and cost discipline.} A
    forecast that avoids shortfalls but regularly overbuilds is wasteful, while a
    forecast that minimizes numerical error but frequently underbuilds may be
    operationally unacceptable. FRS blends these considerations into a single
    readiness indicator.
\end{enumerate}

A normalized composite metric such as FRS is practical for real-world use:
operators often require a single interpretable score that summarizes whether a
forecast is suitable for deployment under service, economic, and operational
constraints. FRS provides this functionality, aligning the evaluation of
forecasting systems with the objectives and risks inherent in operational
decision-making.