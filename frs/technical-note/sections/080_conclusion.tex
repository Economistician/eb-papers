% ----------------------------------------------------------
% CONCLUSION
% ----------------------------------------------------------
\section{Conclusion}

Operational forecasting requires evaluation criteria that extend beyond
aggregate numerical accuracy. In many real-world settings, the central question
is not merely how close a forecast is on average, but whether it can reliably
support execution without incurring unacceptable service failures or asymmetric
economic costs. Traditional accuracy metrics alone cannot answer this question,
as they fail to distinguish between operationally harmless deviations and those
that destabilize workflows.

The Forecast Readiness Score (\FRS) addresses this gap by combining two
complementary dimensions of operational quality: the frequency with which demand
is fully covered (\NSL) and the normalized asymmetric economic exposure
(\CWSLscaled). Together, these components yield a bounded, interpretable
readiness indicator that reflects both service protection and cost discipline.

\FRS{} is not intended to replace traditional accuracy metrics. Instead, it
provides an operational lens through which forecast performance can be evaluated
more holistically. When used alongside measures such as RMSE, MAE, and wMAPE,
\FRS{} helps clarify why certain models perform better in deployment despite
similar statistical accuracy, and why others pose hidden operational risks.

With appropriate cost calibration and normalization, \FRS{} enables clearer
communication with stakeholders, more robust model selection, and early
detection of readiness degradation in production environments. Future extensions
may include probabilistic readiness measures, data-driven cost learning, and
integration with post-processing adjustment layers, further strengthening the
alignment between statistical forecasting and real-world operational needs.