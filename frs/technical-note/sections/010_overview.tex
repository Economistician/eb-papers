% ----------------------------------------------------------
% OVERVIEW
% ----------------------------------------------------------
\section{Overview}

Operational forecasting environments—such as food production, staffing,
inventory, transportation, and short-horizon retail demand—impose constraints
that are not reflected in classical statistical accuracy metrics. Measures such as
RMSE, MAE, and MAPE quantify numerical deviation, but they do not address the
operationally fundamental question: \emph{Is this forecast good enough to run the
system?} As emphasized in the forecast evaluation literature, traditional point
accuracy measures often fail to capture the decision-relevant qualities of a
forecast \citep{gneiting2011}. In practice, many small errors are inconsequential,
while shortfalls and cost-asymmetric deviations can create disproportionate
service and economic disruption.

The Forecast Readiness Framework (FRF) addresses this gap by evaluating forecasts
in terms of their operational feasibility rather than purely statistical accuracy.
Two components are central to this perspective. The \emph{No–Shortfall Level
(NSL)} measures reliability by capturing the fraction of intervals in which the
forecast fully covers realized demand. The \emph{Cost–Weighted Service Loss
(CWSL)} quantifies asymmetric penalties associated with shortfalls and overbuild,
reflecting the economic reality that these deviations have different operational
consequences. Together, NSL and CWSL describe whether a forecasting system is not
only accurate, but \emph{operationally viable}.

The Forecast Readiness Score (FRS) synthesizes these complementary dimensions into
a single, interpretable readiness metric. By combining (i) the frequency with
which the forecast avoids shortfalls and (ii) a normalized measure of asymmetric
economic exposure, FRS evaluates how well a forecast protects service while
maintaining cost discipline. This produces a bounded metric with clear directional
interpretation, enabling operators and analysts to assess readiness at a glance.

This technical note develops the formal definition, interpretation, and operational
motivation of FRS, showing how it supports decision-making in domains where
service reliability, asymmetric cost, and consistent execution are as critical as
numerical accuracy.