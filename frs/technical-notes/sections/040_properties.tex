% ----------------------------------------------------------
% PROPERTIES AND BEHAVIOR
% ----------------------------------------------------------
\section{Properties and Behavior}

The Forecast Readiness Score (\FRS) inherits interpretable and operationally
meaningful behavior from its two constituent components—\NSL{} and the scaled
cost term \CWSLscaled{}. This section summarizes the key mathematical and
practical properties governing how the score responds to different patterns of
forecast error.

\subsection{Boundedness}

Because both \NSL{} and \CWSLscaled{} lie in \([0,1]\), their difference satisfies:
\[
-1 \;\le\; \FRS = \NSL - \CWSLscaled \;\le\; 1.
\]

This bounded range facilitates clear interpretation and consistent comparison
across forecasting systems.

\subsection{Monotonicity}

\FRS{} behaves monotonically with respect to its components:

\begin{itemize}
    \item Increasing \NSL{} increases \FRS.
    \item Increasing \CWSL{} (and therefore \CWSLscaled{}) decreases \FRS.
    \item With identical \NSL{}, the forecast with lower \CWSL{} has the higher \FRS.
    \item With identical \CWSL{}, the forecast with higher \NSL{} has the higher \FRS.
\end{itemize}

These relationships ensure that \FRS{} respects intuitive operational
preferences: reliability is rewarded, and cost exposure is penalized.

\subsection{Sensitivity to Error Types}

\FRS{} differentiates sharply between shortfalls and overbuild errors:

\begin{enumerate}
    \item \textbf{Shortfalls} reduce \NSL{} and increase \CWSL{}, typically causing
    significant declines in \FRS.
    \item \textbf{Overbuild errors} leave \NSL{} unchanged but increase \CWSLscaled{},
    producing more moderate reductions in \FRS.
\end{enumerate}

This structural asymmetry reflects real operational priorities, where shortfalls
create significantly higher service and cost risk.

\begin{figure}[htbp]
\centering
\begin{tikzpicture}
\begin{groupplot}[
    group style={group size=2 by 1, horizontal sep=1.2cm},
    width=0.46\linewidth,
    height=0.38\linewidth,
    xmin=1, xmax=8,
    ymin=8, ymax=16,
    xlabel={$t$},
    ylabel={Quantity},
    axis lines=left,
    grid=both,
    minor tick num=1,
    tick align=outside,
    legend style={draw=none, fill=none, at={(0.02,0.98)}, anchor=north west},
]

\nextgroupplot[title={Forecast A (higher NSL)}]
\addplot+[thick, mark=*] coordinates {(1,10) (2,12) (3,11) (4,9) (5,13) (6,10) (7,14) (8,12)};
\addlegendentry{Actual $y_t$}
\addplot+[thick, mark=square*] coordinates {(1,10.8) (2,12.3) (3,11.2) (4,9.6) (5,13.4) (6,10.4) (7,14.2) (8,12.5)};
\addlegendentry{Forecast $\hat{y}_t$}

\nextgroupplot[title={Forecast B (lower NSL)}]
\addplot+[thick, mark=*] coordinates {(1,10) (2,12) (3,11) (4,9) (5,13) (6,10) (7,14) (8,12)};
\addlegendentry{Actual $y_t$}
\addplot+[thick, mark=square*] coordinates {(1,9.2) (2,11.0) (3,10.2) (4,8.6) (5,12.0) (6,9.3) (7,13.0) (8,11.1)};
\addlegendentry{Forecast $\hat{y}_t$}

\end{groupplot}
\end{tikzpicture}
\caption{Illustration of FRS sensitivity to error type. Two forecasts can exhibit similar symmetric accuracy (e.g., RMSE/MAE) while differing materially in shortfall behavior and asymmetric cost exposure, producing different FRS values.}
\label{fig:frs_sensitivity}
\end{figure}

\subsection{Smoothness and Stability}

Because \CWSL{} is a continuous function of the forecast and \NSL{} is a
step function, \FRS{} exhibits the following behaviors:

\begin{itemize}
    \item \FRS{} is piecewise continuous, with discontinuities arising only when
    \(\hat{y}_t\) crosses \(y_t\), changing the shortfall indicator.
    \item Small forecast adjustments that do not change shortfall occurrence
    produce smooth changes in \FRS.
    \item Structural shifts (such as systematic underforecasting) induce
    predictable, interpretable shifts in both \NSL{} and \CWSL{}, resulting in
    corresponding changes in \FRS.
\end{itemize}

\subsection{Response to Systematic Bias}

\FRS{} responds asymmetrically to forecast bias:

\begin{itemize}
    \item \textbf{Underforecast bias} reduces \NSL{} and increases \CWSL{},
    resulting in sharp decreases in \FRS.
    \item \textbf{Overforecast bias} preserves \NSL{} but increases \CWSL{},
    reducing \FRS{} more gradually.
\end{itemize}

This behavior aligns with operational reality in environments where shortages
carry disproportionately greater cost than excess production.

\subsection{Perfect Forecasts}

If the forecast matches actual demand in every interval
(\(\hat{y}_t = y_t\)), then:
\[
\NSL = 1, \qquad \CWSL = 0, \qquad \CWSLscaled = 0,
\]
so that:
\[
\FRS = 1.
\]

\subsection{Extreme Miscalibration}

If a forecasting system persistently underbuilds demand, then:
\[
\NSL \approx 0, \qquad \CWSLscaled \approx 1,
\]
and thus:
\[
\FRS \approx -1.
\]

Such values signal a severe readiness deficit and a forecasting system misaligned
with operational service and cost requirements.