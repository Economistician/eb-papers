% ----------------------------------------------------------
% ILLUSTRATIVE EXAMPLE
% ----------------------------------------------------------
\section{Illustrative Example}

This section demonstrates how the Forecast Readiness Score (\FRS) behaves in a
concrete numerical example. The worked example illustrates how reliability
(\NSL) and asymmetric economic exposure (\CWSLscaled) interact to produce a
unified readiness measure.

\subsection{Worked Numerical Example}

We consider a short-horizon scenario involving \(T = 6\) evaluation intervals.
Actual demand values \(y_t\) and corresponding forecasts \(\hat{y}_t\) are shown
in Table~\ref{tab:frs_example_y}. The associated shortfall indicators appear in
Table~\ref{tab:frs_example_shortfall}.

\begin{table}[h!]
\centering
\begin{tabular}{c|cccccc}
\hline
$t$ & 1 & 2 & 3 & 4 & 5 & 6 \\ \hline
$y_t$ (actual)        & 10 & 12 & 11 & 9 & 13 & 10 \\
$\hat{y}_t$ (forecast) & 11 & 10 & 11 & 8 & 14 & 9 \\ \hline
\end{tabular}
\caption{Actual and forecast values across six evaluation intervals.}
\label{tab:frs_example_y}
\end{table}
\input{tables/frs_example_shortfall}

A shortfall occurs when \(\hat{y}_t < y_t\). Three of the six intervals exhibit
no shortfall, yielding:
\[
\NSL = \frac{3}{6} = 0.5.
\]

To compute the cost-weighted service loss, we evaluate the magnitudes of
shortfall and overbuild for each interval. These appear in
Table~\ref{tab:frs_example_cost}.

\input{tables/frs_example_cost}

Using asymmetry parameters \(\cu = 2\) (shortfall penalty) and \(\co = 1\)
(overbuild penalty), the total cost-weighted error is:
\[
\text{Shortfall cost} = 2(2 + 1 + 1) = 8, \qquad
\text{Overbuild cost} = 1(1 + 1) = 2,
\]
for a total cost of \(10\). Total realized demand is:
\[
\sum_{t=1}^6 y_t = 65.
\]

Thus:
\[
\CWSL = \frac{10}{65} \approx 0.154.
\]

Suppose the application specifies a normalization bound of
\(\CWSLmax = 0.30\), representing the largest economically meaningful loss in
this context. The scaled cost exposure becomes:
\[
\CWSLscaled = \min\!\left( 1,\, \frac{0.154}{0.30} \right) = 0.513.
\]

Finally, the Forecast Readiness Score is:
\[
\FRS = \NSL - \CWSLscaled
     = 0.5 - 0.513
     = -0.013.
\]

\subsection*{Interpretation}

Although the forecast avoids shortfalls in half of the intervals, the asymmetric
penalties assign substantial cost to shortfall events relative to overbuild. As
a result, the normalized cost exposure slightly exceeds the reliability signal,
yielding a readiness score just below zero. This illustrates how \FRS{}
synthesizes both service protection and cost discipline: even moderate shortfall
frequency can dominate readiness when underbuild penalties are high.