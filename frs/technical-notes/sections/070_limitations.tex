% ----------------------------------------------------------
% LIMITATIONS
% ----------------------------------------------------------
\section{Limitations and Appropriate Use}

While the Forecast Readiness Score (\FRS) provides an operationally aligned
measure of forecast quality, several limitations should be considered when
interpreting or comparing readiness across systems. These limitations reflect
the modeling assumptions and context-dependent nature of the metric.

\subsection{Dependence on Cost Assumptions}

\FRS{} incorporates asymmetric penalties through the cost parameters \(\cu\) and
\(\co\). Its interpretation is therefore conditioned on the selected cost
structure. If these parameters do not reflect the true operational trade-offs,
the resulting readiness score may misrepresent decision relevance.

Accordingly:

\begin{itemize}
    \item \(\cu\) and \(\co\) should be selected with domain knowledge,
    \item sensitivity analysis is recommended when cost parameters are uncertain,
    \item \FRS{} values should be compared only when cost assumptions are consistent across forecasts.
\end{itemize}

\subsection{Not a Pure Accuracy Metric}

\FRS{} is intentionally not a symmetric accuracy measure. Two forecasts with
similar RMSE, MAE, or wMAPE may yield very different \FRS{} values if their
shortfall frequency or asymmetric cost exposure differs. As such, \FRS{} is best
used alongside traditional accuracy metrics to provide a fuller assessment of
forecast performance.

\subsection{Sensitivity to the Normalization Bound}

The scaled cost term \(\CWSLscaled\) relies on a user-selected upper bound
\(\CWSLmax\), which represents the worst economically meaningful loss in the
application. If this bound is chosen poorly:

\begin{itemize}
    \item cost variation may be compressed, obscuring meaningful differences, or
    \item cost differences may be exaggerated, overstating readiness deficits.
\end{itemize}

As operational conditions evolve, \CWSLmax{} should be reviewed and recalibrated.

\subsection{Unsuitability for Long-Horizon Strategic Forecasting}

\FRS{} is designed for short- to medium-horizon operational forecasting, where
service failures and asymmetric penalties are salient. It is generally not
appropriate for long-horizon strategic forecasting, trend analysis, or settings
where execution feasibility is not the primary concern.

\subsection{Appropriate Role of FRS}

\FRS{} is best viewed as an \emph{operational readiness indicator}. Its strength
lies in:

\begin{itemize}
    \item deployment decision-making,
    \item model evaluation under asymmetric risk,
    \item early detection of degradation in production forecasting systems.
\end{itemize}

Used in combination with diagnostic accuracy metrics and operational KPIs, \FRS{}
supports more resilient and informed forecasting practices.