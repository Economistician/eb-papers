% ==========================================================
% 020_scope_and_nongoals.tex
% Scope and Non-Goals
% ==========================================================

\section{Scope and Non-Goals}
\label{sec:governance_scope_nongoals}

Governance in the Electric Barometer framework is intentionally narrow in scope
and deliberately restrictive in function. Its role is not to generate insight,
improve forecasts, or explore tradeoffs, but to issue an authoritative operational
decision based on upstream diagnostics.

This section defines the precise responsibilities of Governance and explicitly
documents the capabilities it excludes. These boundaries are essential to
prevent misuse, scope creep, and interpretive ambiguity.

\subsection{Scope}

Governance operates as the terminal decision layer in the Electric Barometer
stack. Within this role, its responsibilities are limited to the following:

\begin{itemize}
  \item \textbf{Decision closure.}  
  Governance maps a fixed set of diagnostic inputs and thresholds to exactly one
  authoritative policy outcome. It exists to close the decision surface, not to
  explore it.

  \item \textbf{Enforcement of admissible representation.}  
  Governance determines which unit system (raw or snapped) is authoritative for
  interpretation, based on upstream \DQC{} classification, and enforces that
  choice consistently across diagnostics and policy.

  \item \textbf{Policy declaration.}  
  Governance emits an explicit, structured decision artifact specifying:
  \begin{itemize}
    \item tolerance interpretation semantics,
    \item readiness adjustment allowability,
    \item and a governance status suitable for downstream control systems.
  \end{itemize}

  \item \textbf{Exclusivity and consistency.}  
  Governance enforces exclusivity across diagnostic interpretations:
  exactly one \FPC{} assessment governs allowability, and raw and snapped
  reasoning are never mixed.

  \item \textbf{Auditability and accountability.}  
  Governance decisions are deterministic, threshold-driven, and traceable to
  observable inputs. Each decision can be reviewed, reproduced, and justified
  without reliance on model internals or subjective judgment.
\end{itemize}

Governance operates per entity, per evaluation context, and per decision window.
It does not pool, smooth, or aggregate decisions across entities unless such
aggregation is explicitly performed downstream.

\subsection{Non-Goals}

Equally important are the capabilities Governance explicitly excludes. Governance
does \textbf{not} attempt to:

\begin{itemize}
  \item \textbf{Optimize outcomes.}  
  Governance does not optimize cost, service level, or accuracy. It does not
  select policies based on objective functions or tradeoff curves.

  \item \textbf{Learn or adapt.}  
  Governance contains no learning, adaptation, or stateful behavior. All logic
  is deterministic and driven solely by current inputs and declared thresholds.

  \item \textbf{Reinterpret diagnostics.}  
  Governance does not reweight, smooth, or reinterpret \DQC{} or \FPC{} outputs.
  It consumes them exactly as produced and applies fixed decision rules.

  \item \textbf{Override upstream conclusions.}  
  Governance cannot override a snapping requirement, tolerance interpretation,
  or incompatibility determination. Manual or discretionary overrides are
  explicitly out of scope.

  \item \textbf{Act as a monitoring or analytics layer.}  
  Governance is not a dashboard, reporting system, or exploratory analytics
  tool. Its outputs are prescriptive, not descriptive.

  \item \textbf{Guarantee downstream performance.}  
  A permissive governance decision does not guarantee improved outcomes, nor
  does a restrictive decision imply poor forecasting quality. Governance
  addresses admissibility, not success.
\end{itemize}

By maintaining this strict scope, Governance preserves the separation of
concerns that underpins the Electric Barometer framework. Diagnostics diagnose,
Governance decides, and downstream systems execute—each within a clearly defined
and enforceable boundary.
