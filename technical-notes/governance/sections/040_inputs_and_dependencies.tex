% ==========================================================
% 040_inputs_and_dependencies.tex
% Inputs and Dependencies
% ==========================================================

\section{Inputs and Dependencies}
\label{sec:governance_inputs_dependencies}

Governance in the Electric Barometer framework is intentionally narrow in what it
accepts as input and intentionally strict in what it depends upon. This section
defines the admissible inputs to the Governance layer, clarifies its upstream
dependencies, and explicitly documents what information is excluded by design.

\subsection{Required Inputs}

Governance consumes only structured diagnostic artifacts produced upstream. All
inputs must be observable, versioned, and attributable to a specific evaluation
context.

The required inputs are:

\begin{itemize}[leftmargin=1.5em]
  \item \textbf{Demand Quantization Compatibility (\DQC{}) output}, including:
  \begin{itemize}
    \item the assigned demand-structure class,
    \item the detected grid unit (if applicable),
    \item and supporting evidence signals.
  \end{itemize}

  \item \textbf{Forecast Primitive Compatibility (\FPC{}) diagnostics}, evaluated
  in the admissible unit system determined by \DQC{}:
  \begin{itemize}
    \item raw-unit diagnostics when demand is continuous-like,
    \item snapped-unit diagnostics when demand is quantized or piecewise-packed.
  \end{itemize}

  \item \textbf{Governance parameters and thresholds}, supplied explicitly or via
  a named policy preset, including:
  \begin{itemize}
    \item DQC classification thresholds,
    \item FPC compatibility thresholds,
    \item readiness adjustment constraints.
  \end{itemize}
\end{itemize}

All inputs must be present and internally consistent. Governance does not infer,
approximate, or backfill missing diagnostics.

\subsection{Upstream Diagnostic Dependencies}

Governance depends on upstream diagnostics only through their declared outputs,
not through their internal mechanics.

Specifically:
\begin{itemize}[leftmargin=1.5em]
  \item Governance depends on \DQC{} to resolve the admissible representation and
        unit system for interpretation.
  \item Governance depends on \FPC{} to assess whether readiness adjustment is
        structurally coherent under that representation.
\end{itemize}

Governance does not recompute, reinterpret, or partially apply upstream logic.
It treats diagnostic outputs as authoritative contracts, not as advisory signals.

\subsection{Explicit Non-Dependencies}

Equally important are the inputs that Governance explicitly does \emph{not}
consume. These exclusions are deliberate and non-negotiable.

Governance does \textbf{not} depend on:
\begin{itemize}[leftmargin=1.5em]
  \item model identifiers, architectures, or training procedures,
  \item feature definitions or data preprocessing pipelines,
  \item forecast accuracy metrics, cost summaries, or service-level outcomes,
  \item optimization objectives, loss functions, or learned parameters,
  \item exploratory analyses or analyst judgment.
\end{itemize}

Governance is therefore isolated from optimization and learning loops by design.
It neither tunes thresholds based on performance nor adapts its behavior in
response to downstream outcomes. All policy logic is fixed, declarative, and
version-controlled.

This isolation ensures that governance decisions remain structurally grounded,
auditable, and stable over time, and that responsibility for policy outcomes
cannot be obscured by adaptive behavior or implicit feedback mechanisms.

\subsection{Failure on Missing or Invalid Inputs}

When required diagnostics are missing, inconsistent, or evaluated in an
inadmissible unit system, Governance fails explicitly.

It does not:
\begin{itemize}[leftmargin=1.5em]
  \item degrade gracefully,
  \item substitute alternative diagnostics,
  \item or infer intent from partial information.
\end{itemize}

This failure mode is intentional. Governance prefers explicit non-decision over
silent policy drift, ensuring that unresolved structural ambiguity is surfaced
to the owning organization rather than embedded into automated control.
