% ==========================================================
% 050_decision_logic.tex
% Decision Logic
% ==========================================================

\section{Decision Logic}
\label{sec:governance_decision_logic}

The governance decision logic is a deterministic composition of structural
diagnostics. It maps admissible diagnostic inputs to a single, authoritative
\texttt{GovernanceDecision} artifact without optimization, learning, or
stochastic elements.

This section describes the ordered decision logic that binds Demand Quantization
Compatibility (\DQC{}) and Forecast Primitive Compatibility (\FPC{}) into an
explicit governance outcome. The purpose of this logic is not to rank options,
but to eliminate inadmissible interpretations and close the policy space.

\subsection{Stage 1: Demand Structure Resolution}

Governance begins by evaluating \DQC{} on the realized demand series
$\{y_{it}\}_{t \in \Tset}$ for a single entity.

This stage assigns exactly one demand structure class:
\begin{itemize}[leftmargin=*]
  \item \ContinuousLike,
  \item \Quantized, or
  \item \PiecewisePacked.
\end{itemize}

Based on this classification, Governance declares the authoritative
\emph{representation and unit system} under which all downstream evaluation must
occur.

If demand is classified as \Quantized{} or \PiecewisePacked{}, Governance:
\begin{itemize}[leftmargin=*]
  \item declares snapping to the detected grid or admissible value set mandatory,
  \item requires tolerance interpretation in grid-consistent units,
  \item prohibits raw-unit evaluation for policy decisions.
\end{itemize}

If demand is classified as \ContinuousLike{}, Governance:
\begin{itemize}[leftmargin=*]
  \item declares raw-unit evaluation admissible,
  \item waives snapping requirements.
\end{itemize}

This stage establishes the authoritative representation for all subsequent
diagnostics. No downstream logic may override or reinterpret this declaration.

\subsection{Stage 2: Forecast Primitive Compatibility Evaluation}

Given the authoritative unit system declared by \DQC{}, Governance evaluates
\FPC{} in exactly one admissible representation.

\begin{itemize}[leftmargin=*]
  \item If snapping is \emph{not} required, \FPC{} is evaluated using raw-unit
        signals.
  \item If snapping \emph{is} required, \FPC{} is evaluated using snapped,
        grid-aligned signals.
\end{itemize}

Raw and snapped \FPC{} signals are never mixed. Exactly one \FPC{} evaluation is
authoritative in any governance decision.

The resulting \FPC{} classification assigns the forecast primitive to one of the
following compatibility states:
\begin{itemize}[leftmargin=*]
  \item \textbf{Compatible},
  \item \textbf{Marginal},
  \item \textbf{Incompatible}.
\end{itemize}

This classification determines whether readiness adjustment is structurally
admissible under the declared demand representation.

\subsection{Stage 3: Policy Determination and Encoding}

Governance maps the authoritative \FPC{} classification into an explicit policy
bundle and encodes the result in the \texttt{GovernanceDecision} artifact.

This mapping is deterministic:

\begin{itemize}[leftmargin=*]
  \item \textbf{Compatible} $\rightarrow$ readiness adjustment allowed,
  \item \textbf{Marginal} $\rightarrow$ readiness adjustment allowed only under
        explicit constraints,
  \item \textbf{Incompatible} $\rightarrow$ readiness adjustment disallowed.
\end{itemize}

In addition, Governance encodes:
\begin{itemize}[leftmargin=*]
  \item the snapping requirement and governing unit system,
  \item the authoritative tolerance interpretation policy,
  \item a consolidated governance status indicator,
  \item an ordered list of reasoning statements linking diagnostics to policy.
\end{itemize}

The traffic-light status is a summary signal for downstream systems and does not
introduce additional logic beyond the explicitly declared policies.

\subsection{No Fallbacks, Mixing, or Heuristics}

Governance intentionally provides:
\begin{itemize}[leftmargin=*]
  \item no averaging between raw and snapped diagnostics,
  \item no heuristic reconciliation of incompatible signals,
  \item no implicit fallbacks or soft overrides.
\end{itemize}

If required inputs are unavailable (e.g., snapped \FPC{} signals when snapping is
mandatory), Governance fails explicitly rather than degrading silently.

\subsection{Determinism and Accountability}

For fixed diagnostic inputs and governance thresholds, the decision logic always
produces the same \texttt{GovernanceDecision}. This determinism ensures:
\begin{itemize}[leftmargin=*]
  \item auditability of historical decisions,
  \item reproducibility across environments,
  \item accountability for policy outcomes,
  \item stable integration with automated control systems.
\end{itemize}

Governance behavior may change only through explicit, versioned updates to
diagnostic definitions or policy thresholds. There is no hidden state, learning,
or adaptive behavior within the decision logic itself.
