% ==========================================================
% 070_policy_semantics_and_constraints.tex
% Governance — Electric Barometer
% ==========================================================

\section{Policy Semantics and Constraints}
\label{sec:governance_policy_semantics_constraints}

Governance in the Electric Barometer framework exists to impose
\emph{semantic constraints} on downstream evaluation and control.
It does not generate forecasts, optimize objectives, or propose actions.
Instead, it determines which interpretations, transformations, and
interventions are admissible given the diagnosed structure of demand
and forecast behavior.

This section defines the meaning of Governance policies and the
constraints they impose on system behavior.

\subsection{Authoritative Semantics}

A Governance decision establishes the \emph{authoritative semantics}
under which all downstream logic must operate. This includes:

\begin{itemize}
  \item the unit system in which quantities are interpreted,
  \item the admissible representation of forecasts and demand,
  \item the conditions under which readiness adjustment may occur.
\end{itemize}

Once issued, these semantics are binding. Downstream components are
not permitted to reinterpret units, bypass transformations, or
select alternative diagnostic views.

\subsection{Representation Constraints}

Governance constrains the representation of forecasts and diagnostics
based on the resolved demand structure:

\begin{itemize}
  \item If demand is classified as \emph{continuous-like}, evaluation and
        adjustment may occur in raw units.
  \item If demand is classified as \emph{quantized} or
        \emph{piecewise-packed}, snapping to an admissible support is
        mandatory, and raw-unit interpretation is prohibited.
\end{itemize}

These constraints apply uniformly across diagnostics, tolerance
interpretation, and readiness policy. Representation is not a local
choice—it is a system-level obligation.

\subsection{Tolerance Interpretation Constraints}

Tolerance parameters are not free tuning knobs under Governance.
Their interpretation is constrained by the admissible demand support:

\begin{itemize}
  \item Tolerances must be interpretable in the governing unit system.
  \item Tolerances smaller than the minimum admissible demand increment
        are invalid by construction.
  \item Reported service metrics must correspond to realizable outcomes.
\end{itemize}

Governance prohibits tolerance configurations that produce
structurally meaningless or misleading diagnostics, even if such
configurations improve apparent performance metrics.

\subsection{Readiness Adjustment Constraints}

Readiness adjustment policies (e.g.\ scale-based interventions such as
\RAL{}) are governed by explicit allowability rules:

\begin{itemize}
  \item Adjustment is forbidden if the governing \FPC{} classification
        is incompatible.
  \item Adjustment may be conditionally allowed under marginal
        compatibility, subject to declared constraints.
  \item Adjustment is permitted only when compatibility is affirmed
        under the authoritative representation.
\end{itemize}

Governance does not prescribe how adjustments are computed. It
determines only whether adjustment is admissible at all.

\subsection{Prohibited Behaviors}

The following behaviors constitute explicit governance violations:

\begin{itemize}
  \item mixing raw-unit and snapped diagnostics within a single decision,
  \item applying readiness adjustments under a non-authoritative
        representation,
  \item interpreting tolerance metrics outside admissible units,
  \item overriding governance outcomes with heuristic or discretionary logic.
\end{itemize}

Governance is designed to fail closed. If admissible conditions are
not met, policy execution must halt rather than degrade silently.

\subsection{Semantic Stability}

Governance policies are intentionally stable over time. They may evolve
only through explicit, versioned changes to diagnostic definitions,
thresholds, or mapping logic.

This stability ensures that changes in outcomes reflect changes in
demand structure, forecast behavior, or declared policy—not drift in
interpretation or semantics.

\subsection{Role in the Decision Stack}

Within the Electric Barometer stack, Governance serves as the final
semantic gate. As formalized in the Governance Decision Contract
(Section~\ref{sec:governance_decision_contract}), all policy semantics
derive from a deterministic, exclusive mapping from diagnostics to
authority—not from ad hoc logic or discretionary interpretation.

Governance does not improve forecasts or optimize decisions.
It ensures that any decision taken is \emph{well-defined}, admissible,
and accountable under the structural realities diagnosed upstream.

In doing so, Governance converts analytical signals into enforceable
operational constraints and closes the decision loop defined by the
contract.
