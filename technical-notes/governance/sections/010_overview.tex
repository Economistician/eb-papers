% ==========================================================
% 010_overview.tex
% Overview
% ==========================================================

\section{Overview}
\label{sec:governance_overview}

Modern operational forecasting systems increasingly embed automated control
levers, tolerance-based evaluation, and asymmetric cost tradeoffs directly into
decision workflows. In such settings, analytic diagnostics alone are
insufficient. What ultimately matters is not whether a forecast appears
reasonable, but whether a downstream action is \emph{structurally admissible},
\emph{policy-compliant}, and \emph{accountable}.

Within the Electric Barometer framework, this responsibility is assigned to
\emph{Governance}. Governance is the final decision layer that binds structural
diagnostics into an authoritative operational outcome. It does not generate new
signals, optimize objectives, or reinterpret performance. Instead, it resolves
diagnostic inputs into a single, enforceable policy decision.

This technical note formalizes governance as a deterministic \emph{decision
contract}. Given a fixed set of diagnostic inputs and thresholds, governance
produces exactly one authoritative artifact that specifies:
\begin{itemize}[leftmargin=1.5em]
  \item the admissible unit system for interpretation,
  \item the authoritative tolerance semantics,
  \item the allowability of readiness adjustment,
  \item and an explicit governance status.
\end{itemize}

Governance consumes—but does not redefine—two upstream diagnostics:
\begin{itemize}[leftmargin=1.5em]
  \item \textbf{Demand Quantization Compatibility (DQC)}, which determines whether
        realized demand admits continuous interpretation or requires discrete,
        grid-aware semantics; and
  \item \textbf{Forecast Primitive Compatibility (FPC)}, which determines whether
        a forecast primitive can respond coherently to readiness adjustment under
        admissible perturbations.
\end{itemize}

Crucially, governance does not average, negotiate, or arbitrate between these
diagnostics. It enforces strict exclusivity: one unit system governs, one
diagnostic interpretation is authoritative, and one readiness policy applies.
When required inputs are missing or incompatible, governance fails explicitly
rather than degrading silently.

The purpose of this design is closure. Governance exists to \emph{terminate
interpretation, not extend it}: once a \texttt{GovernanceDecision} is issued, no
further diagnostic reasoning is admissible.

By enforcing this closure, governance prevents common operational failure modes
such as:
\begin{itemize}[leftmargin=1.5em]
  \item mixing raw and snapped evaluation semantics,
  \item interpreting tolerances smaller than realizable demand increments,
  \item applying readiness adjustments that cannot be operationally executed.
\end{itemize}

This note describes the governance contract, the structure of the decision
artifact it emits, the deterministic logic by which decisions are made, and the
interpretive boundaries that preserve accountability. Together with the DQC and
FPC technical notes, it completes the Electric Barometer framework by ensuring
that readiness decisions are not only cost-aware or service-aware, but
\emph{structurally valid, auditable, and enforceable by design}.
