% ==========================================================
% 120_conclusion.tex
% Governance — Electric Barometer
% ==========================================================

\section{Conclusion}
\label{sec:governance_conclusion}

This technical note formalizes governance as the final and binding layer of the
Electric Barometer framework. While Demand Quantization Compatibility (DQC) and
Forecast Primitive Compatibility (FPC) diagnose structural properties of demand
and forecast behavior, governance determines what actions are \emph{permitted},
\emph{constrained}, or \emph{forbidden} as a result.

By framing governance as a deterministic decision contract, the framework
ensures that operational actions are not merely data-informed, but
\emph{structurally admissible by design}. All downstream policies—unit
interpretation, tolerance semantics, and readiness adjustment—are derived from
explicit diagnostics and resolved into a single authoritative decision artifact.

A central contribution of this framework is the enforcement of separation:
\begin{itemize}[leftmargin=1.5em]
  \item Structural diagnosis is separated from performance evaluation,
  \item Evaluation is separated from optimization,
  \item Optimization is separated from policy authority.
\end{itemize}

This separation prevents common failure modes in operational analytics, where
implicit assumptions about continuity, tolerance, or adjustability are silently
violated in pursuit of apparent gains. Governance makes those assumptions
explicit, testable, and enforceable.

Within Electric Barometer, governance is not an advisory overlay or reporting
convenience. It is the mechanism by which accountability is assigned, audit
trails are preserved, and unsafe decision pathways are closed. When governance
restricts action, it does so deliberately—not as a reflection of model quality,
but as a statement about structural validity.

As forecasting systems increasingly automate control levers, incorporate
asymmetric cost structures, and operate at finer temporal and operational
granularity, such governance mechanisms become essential. Without them,
organizations risk optimizing against ill-defined objectives in inadmissible
spaces.

In this sense, governance is not an optional extension of forecasting—it is its
necessary conclusion. Before asking whether a decision improves performance, we
must first ensure that the decision itself is well-defined, defensible, and
accountable. The Electric Barometer governance framework exists to enforce that
principle.
