% ==========================================================
% 090_auditability_and_accountability.tex
% Governance — Electric Barometer
% ==========================================================

\section{Auditability and Accountability}
\label{sec:governance_auditability_accountability}

A core objective of governance in the Electric Barometer framework is to ensure
that readiness decisions are not only structurally valid, but also
\emph{auditable, accountable, and defensible} over time. Governance is therefore
designed as a deterministic, traceable system whose outputs can be inspected,
reproduced, and reviewed independently of the forecasting models or operational
actors that consume them.

This section describes how auditability and accountability are enforced by
construction, rather than retrofitted through logging or oversight processes.

\subsection{Deterministic Decision Reproduction}

Governance decisions are fully deterministic. Given:
\begin{itemize}[leftmargin=1.5em]
  \item the realized demand series,
  \item the required diagnostic signals (\DQC{} and \FPC{}),
  \item and a fixed set of governance thresholds,
\end{itemize}
the governance system will always produce the same decision artifact.

There is no randomness, adaptive learning, or hidden state in governance logic.
This property enables exact replay of historical decisions and eliminates
ambiguity about why a particular readiness action was allowed or disallowed at a
given point in time.

\subsection{Explicit Traceability of Decisions}

Every governance outcome is accompanied by an explicit reasoning trace embedded
in the \texttt{GovernanceDecision} artifact. This trace links:
\begin{itemize}[leftmargin=1.5em]
  \item observed demand structure (\DQC{}),
  \item the governing \FPC{} classification,
  \item unit system and snapping requirements,
  \item tolerance interpretation rules,
  \item and the resulting readiness policy.
\end{itemize}

As a result, no governance decision relies on implicit assumptions, undocumented
heuristics, or post-hoc interpretation. Each outcome can be traced directly to
observable inputs and declared rules.

\subsection{Separation of Responsibility}

Accountability in Electric Barometer is enforced through strict separation of
responsibilities:
\begin{itemize}[leftmargin=1.5em]
  \item Diagnostics (\DQC{}, \FPC{}) describe structure and responsiveness.
  \item Governance determines admissibility and policy.
  \item Downstream systems execute actions subject to governance constraints.
\end{itemize}

No single component is permitted to both diagnose and authorize action. This
separation ensures that errors in modeling, estimation, or interpretation cannot
silently propagate into operational control without crossing an explicit
governance boundary.

\subsection{Prevention of Silent Policy Drift}

A common failure mode in operational systems is \emph{policy drift}: gradual,
untracked changes in how metrics, tolerances, or adjustments are interpreted
across time, teams, or systems.

The governance layer prevents such drift by:
\begin{itemize}[leftmargin=1.5em]
  \item declaring a single authoritative unit system,
  \item enforcing exclusive use of raw or snapped diagnostics,
  \item emitting explicit readiness allowability decisions.
\end{itemize}

Any change in governance behavior must occur through a versioned change to
diagnostic definitions or threshold parameters, making policy evolution visible
and reviewable.

\subsection{Accountability Without Model Introspection}

Governance accountability does not depend on inspecting forecasting model
internals, training data, or estimation procedures. Decisions are justified
solely on the basis of observable behavior and declared rules.

This design choice allows governance to:
\begin{itemize}[leftmargin=1.5em]
  \item remain model-agnostic,
  \item apply consistently across heterogeneous forecasting stacks,
  \item support external audit and review without proprietary disclosure.
\end{itemize}

Accountability is therefore attached to decisions and policy authority, not to
models or individual contributors.

\subsection{Governance as a Risk Control Mechanism}

By enforcing auditability and accountability, governance functions as a formal
risk control layer within Electric Barometer. It ensures that readiness decisions
are:
\begin{itemize}[leftmargin=1.5em]
  \item structurally admissible,
  \item consistently interpreted,
  \item and defensible under retrospective scrutiny.
\end{itemize}

This capability is essential in environments where automated or semi-automated
control decisions carry financial, operational, or reputational risk.

\subsection{Summary}

Auditability and accountability in Electric Barometer are not achieved through
after-the-fact reporting, but through architectural constraint. By enforcing
determinism, traceability, and separation of authority, the governance layer
ensures that readiness decisions can be understood, reproduced, and defended
long after they are made.

In this way, governance transforms structural diagnostics into operationally
responsible decision-making.
