% ==========================================================
% 030_governance_contract.tex
% Governance Contract
% ==========================================================

\section{The Governance Decision Contract}
\label{sec:governance_decision_contract}

Governance within the Electric Barometer framework is formalized as a
\emph{decision contract}: a deterministic, auditable mapping from a fixed set of
diagnostic inputs to a single authoritative policy outcome. This contract is
intentionally explicit and intentionally restrictive. Its purpose is not to
explore a decision space, but to close it.

By construction, Governance does not optimize, infer, or adapt. It enforces
structural admissibility by declaring what interpretations and actions are
permitted given the observed properties of demand and forecast behavior.

\subsection{Contract Inputs}

The Governance Decision Contract consumes only the following categories of
inputs:

\begin{enumerate}[leftmargin=*]
  \item \textbf{Realized demand series}
  \[
    \{\yit\}_{t \in \Tset}
  \]
  for a single entity $i$, used exclusively to determine demand structure via
  Demand Quantization Compatibility (\DQC{}).

  \item \textbf{Forecast Primitive Compatibility (\FPC{}) diagnostics}, computed
  from observable forecast behavior under admissible representations:
  \begin{itemize}
    \item Raw-unit \FPC{} signals (when snapping is not required),
    \item Snapped-unit \FPC{} signals (when required by \DQC{}).
  \end{itemize}

  \item \textbf{Governance parameters and thresholds}, supplied explicitly or via
  a named policy preset, including:
  \begin{itemize}
    \item DQC classification thresholds,
    \item FPC compatibility thresholds,
    \item Readiness policy constraints.
  \end{itemize}
\end{enumerate}

No other information is admissible. In particular, Governance does \emph{not}
consume model identifiers, training metadata, feature definitions, or aggregate
performance summaries beyond the diagnostic artifacts themselves.

\subsection{Contract Output}

The contract produces a single structured artifact, the
\texttt{GovernanceDecision}. This artifact is the sole authoritative output of
the Governance layer and contains:

\begin{itemize}[leftmargin=*]
  \item The assigned \DQC{} classification and detected grid unit (if applicable),
  \item The authoritative \FPC{} classification used for policy evaluation,
  \item A declaration of the governing unit system (raw or snapped),
  \item An explicit snapping requirement flag,
  \item A tolerance interpretation policy (raw-unit vs.\ grid-unit $\tau$),
  \item A readiness adjustment policy
        (\textsc{Allowed}, \textsc{Conditional}, or \textsc{Disallowed}),
  \item A consolidated governance status indicator,
  \item An ordered list of explicit reasoning statements supporting the decision.
\end{itemize}

This artifact is \emph{binding}. Downstream systems must consume it as-is and may
not reconstruct, reinterpret, or partially override Governance logic.

\subsection{Determinism, Exclusivity, and Accountability}

A defining property of the Governance Decision Contract is
\emph{exclusivity of authority}:

\begin{itemize}[leftmargin=*]
  \item Exactly one unit system governs interpretation (raw \emph{or} grid),
  \item Exactly one \FPC{} assessment governs readiness allowability,
  \item Exactly one readiness policy is declared.
\end{itemize}

Mixing raw and snapped reasoning within a single decision is explicitly
disallowed. If snapping is required by \DQC{}, only snapped-unit \FPC{} signals
are admissible for policy determination. Conversely, when demand is
continuous-like, snapped diagnostics are ignored.

Given identical inputs and parameters, the Governance Decision Contract must
always return identical outputs. There is no randomness, statefulness, learning,
or adaptive behavior. This determinism establishes clear accountability: every
policy outcome is traceable to declared inputs, thresholds, and rules, and
therefore to an identifiable Governance owner responsible for the decision.

\subsection{Why a Contract}

Framing Governance as a decision contract enforces several non-negotiable design
principles:

\begin{itemize}[leftmargin=*]
  \item \textbf{Auditability:} every governance outcome is traceable to observable,
        versioned inputs.
  \item \textbf{Composability:} Governance layers cleanly on top of \DQC{} and
        \FPC{} without redefining them.
  \item \textbf{Accountability:} decision authority is centralized and explicit,
        preventing implicit overrides or policy drift.
  \item \textbf{Operational safety:} downstream systems are shielded from hidden
        assumptions about units, tolerances, or admissibility.
\end{itemize}

In short, the Governance Decision Contract exists to ensure that readiness
policies are not merely cost-aware or service-aware, but
\emph{structurally admissible and enforceable by design}.
