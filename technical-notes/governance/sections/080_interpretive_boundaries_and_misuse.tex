% ==========================================================
% 080_interpretive_boundaries_and_misuse.tex
% Governance — Electric Barometer
% ==========================================================

\section{Interpretive Boundaries and Common Misuse}
\label{sec:governance_interpretive_boundaries}

Governance in the Electric Barometer framework is designed to \emph{close}
decision space, not expand it. Because it produces authoritative, categorical
policy outcomes, it is particularly susceptible to misuse if treated as an
advisory signal, a performance summary, or a negotiable recommendation.

This section establishes explicit interpretive boundaries for governance outputs
and documents common failure modes that arise when those boundaries are violated.

\subsection{Governance Is Not a Performance Metric}

Governance outcomes are not measures of forecast quality, operational success,
or economic efficiency. A restrictive or prohibitive governance decision does
not imply poor modeling, suboptimal forecasting, or inadequate performance.

Instead, governance decisions reflect \emph{structural admissibility}: whether a
given class of actions is well-defined under the diagnosed demand structure and
forecast primitive behavior. Treating governance outcomes as performance scores
constitutes a category error.

\subsection{Governance Is Not an Optimization Target}

Governance outputs are not objectives to be maximized, minimized, or tuned
against. Attempts to adjust diagnostics, thresholds, or representations for the
purpose of achieving a more permissive governance status undermine the purpose
of the framework.

Governance exists to constrain optimization, not to be optimized itself. When a
governance decision is unfavorable, the appropriate response is representational
or procedural change—not threshold manipulation.

\subsection{Governance Is Not Advisory}

Governance decisions are \emph{authoritative}. Downstream systems may not treat
them as suggestions, confidence indicators, or contextual inputs.

Partial adoption (e.g., accepting unit semantics but ignoring readiness
constraints) or selective override (e.g., honoring permissive cases while
bypassing restrictive ones) breaks the decision contract and invalidates the
audit trail.

\subsection{No Circumvention via Diagnostic Mixing}

A common and explicitly prohibited misuse pattern is \emph{semantic laundering}
(sometimes referred to as \emph{diagnostic laundering}): the attempt to bypass
governance constraints by selectively reinterpreting, mixing, or substituting
diagnostic contexts after a governance decision has been issued.

Typical examples include:
\begin{itemize}[leftmargin=1.5em]
  \item using raw-unit diagnostics when snapped diagnostics are authoritative,
  \item appealing to exploratory or non-governing metrics after governance
        disallows adjustment,
  \item combining outputs from different evaluation resolutions to justify action.
\end{itemize}

Semantic laundering creates the illusion of admissibility by reintroducing
interpretations that governance has explicitly ruled out. Such practices are
governance violations, not analytical refinements.

\subsection{Governance Is Not a Modeling Prescription}

Governance does not mandate specific forecasting models, loss functions, or
architectures. A disallowed readiness policy does not imply that the underlying
forecast is invalid or that a particular modeling paradigm is required.

Modeling decisions remain outside the scope of governance. Governance constrains
\emph{actionability}, not \emph{representation choice}.

\subsection{Separation from Human Override}

Human judgment may motivate reevaluation, reclassification, or explicit policy
change, but it may not silently override governance outputs.

Any deviation from a governance decision must be:
\begin{itemize}[leftmargin=1.5em]
  \item explicit,
  \item documented,
  \item versioned, and
  \item traceable to revised inputs or thresholds.
\end{itemize}

This separation preserves auditability, accountability, and institutional
memory.

\subsection{Intended Use}

Proper use of governance consists of:
\begin{itemize}[leftmargin=1.5em]
  \item Consuming the governance decision artifact as authoritative,
  \item Enforcing declared unit systems, tolerance semantics, and readiness
        constraints,
  \item Failing explicitly when required diagnostic inputs are missing,
  \item Triggering upstream reevaluation rather than downstream circumvention.
\end{itemize}

By enforcing these boundaries, governance ensures that operational decisions are
not merely data-informed, but structurally valid, auditable, and accountable by
design.
