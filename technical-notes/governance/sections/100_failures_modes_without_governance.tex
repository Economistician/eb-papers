% ==========================================================
% 100_failure_modes_without_governance.tex
% Governance — Electric Barometer
% ==========================================================

\section{Failure Modes Without Governance}
\label{sec:governance_failure_modes}

Absent an explicit governance layer, operational forecasting systems frequently
exhibit failure modes that are subtle, persistent, and difficult to diagnose.
These failures rarely arise from incorrect forecasts or poor optimization, but
from structural ambiguity about how diagnostics, units, and policies should be
interpreted.

This section documents the most common failure modes observed when governance is
implicit, informal, or bypassed.

% ----------------------------------------------------------
% FIGURE: GOVERNANCE FAILURE MODES CONTRAST
% ----------------------------------------------------------
\begin{figure}[htbp]
\centering

\begin{tikzpicture}[x=1cm,y=1cm]

% ----------------------------------------------------------
% Styles
% ----------------------------------------------------------
\tikzset{
  box/.style={draw, rounded corners=3pt, align=center, inner sep=6pt, fill=white},
  smallbox/.style={draw, rounded corners=3pt, align=center, inner sep=5pt, font=\small, fill=white},
  title/.style={font=\small\bfseries, align=center},
  halo/.style={font=\small, align=center, fill=white, inner sep=4pt},
  punch/.style={fill=white, inner sep=2pt},
  note/.style={font=\small, align=center},
  arr/.style={->, thick},
  badarr/.style={->, thick, dashed},
}

% ----------------------------------------------------------
% Layout constants
% ----------------------------------------------------------
\def\panelW{7.2}
\def\panelH{8.8}
\def\xL{0.0}
\def\xR{8.2}
\def\yB{0.0}

% ----------------------------------------------------------
% Panel frames + titles
% ----------------------------------------------------------
\draw[draw, rounded corners=3pt] (\xL,\yB) rectangle ++(\panelW,\panelH);
\draw[draw, rounded corners=3pt] (\xR,\yB) rectangle ++(\panelW,\panelH);

\node[title] at (\xL+0.5*\panelW,\yB+\panelH+0.45) {Without Governance (Failure Mode)};
\node[title] at (\xR+0.5*\panelW,\yB+\panelH+0.45) {With Governance (Decision Closure)};

% ----------------------------------------------------------
% LEFT PANEL (Failure Mode)
% ----------------------------------------------------------
\node[smallbox] (ldqc) at (\xL+2.0,\yB+7.6) {DQC\\(signal)};
\node[smallbox] (lfpc) at (\xL+5.2,\yB+7.6) {FPC\\(signal)};

\node[smallbox] (lraw) at (\xL+2.0,\yB+5.8) {Raw metrics\\\& tolerances};
\node[smallbox] (lsnap) at (\xL+5.2,\yB+5.8) {Snapped metrics\\\& tolerances};

\node[smallbox] (lheur) at (\xL+3.6,\yB+3.8) {Heuristic mixing\\``pick the best''};

\begin{scope}[on background layer]
  \draw[badarr] (lheur.west) to[out=150,in=180] (lraw.west);
  \draw[badarr] (lheur.east) to[out=30,in=0] (lsnap.east);
\end{scope}

\node[smallbox] (lexec) at (\xL+3.6,\yB+1.0) {Execution / Control};

\draw[arr] (lheur) -- (lexec);
\node[halo] at (\xL+3.6,\yB+2.4)
  {\emph{Semantic laundering}\\(diagnostic mixing to bypass constraints)};

\draw[arr] (ldqc) -- (lraw);
\draw[arr] (lfpc) -- (lsnap);
\draw[badarr] (lraw) -- (lheur);
\draw[badarr] (lsnap) -- (lheur);

\node[note] at (\xL+3.6,\yB+0.4)
  {\emph{Policy drift} / \emph{non-reproducible outcomes}};

% ----------------------------------------------------------
% RIGHT PANEL (Decision Closure)
% ----------------------------------------------------------
\begin{scope}[on background layer]
  % Raised compliance boundary
  \draw[draw, dashed, rounded corners=3pt]
    (\xR+0.6,\yB+6.0) -- (\xR+0.6,\yB+3.8) --
    (\xR+\panelW-0.6,\yB+3.8) -- (\xR+\panelW-0.6,\yB+6.0);
  \draw[draw, dashed]
    (\xR+0.6,\yB+6.0) -- (\xR+\panelW-0.6,\yB+6.0);

  \node[punch] at (\xR+2.43,\yB+6.0) {\phantom{x}};
  \node[punch] at (\xR+4.77,\yB+6.0) {\phantom{x}};
\end{scope}

\node[smallbox] (rdqc) at (\xR+2.0,\yB+7.6) {DQC\\(diagnostic)};
\node[smallbox] (rfpc) at (\xR+5.2,\yB+7.6) {FPC\\(diagnostic)};

\node[box] (rgov) at (\xR+3.6,\yB+4.8)
  {Governance Decision Contract\\\texttt{GovernanceDecision}};

\draw[arr] (rdqc.south) -- (rgov.155);
\draw[arr] (rfpc.south) -- (rgov.25);

\node[smallbox] (rexec) at (\xR+3.6,\yB+1.1)
  {Execution / Control / Automation};

\draw[arr] (rgov) -- (rexec);

\node[halo] at (\xR+3.6,\yB+3.3)
  {{\LARGE $\boldsymbol{\times}$}};

\node[halo] at (\xR+3.6,\yB+2.45)
  {No downstream reinterpretation\\or diagnostic substitution};

\node[note] at (\xR+3.6,\yB+0.5)
  {Deterministic \quad \textbullet \quad Traceable \quad \textbullet \quad Owned};

% Label aligned with dashed boundary
\node[halo, inner sep=6pt] at (\xR+3.6,\yB+6.0)
  {\emph{compliance boundary}};

\end{tikzpicture}

\caption{
Governance failure modes versus decision closure.
\emph{Left:} In the absence of an explicit governance layer, Demand Quantization Compatibility (DQC)
and Forecast Primitive Compatibility (FPC) are treated as advisory signals rather than authoritative
constraints. Raw and snapped evaluation semantics coexist, enabling heuristic mixing and post-hoc
reinterpretation (``semantic laundering'') that bypasses structural constraints and leads to policy drift
and non-reproducible outcomes.
\emph{Right:} With governance, DQC and FPC are consumed as diagnostics and deterministically composed
into a single \texttt{GovernanceDecision}. This decision contract establishes a strict compliance boundary
separating diagnostic evaluation from operational execution. Once issued, no downstream reinterpretation,
diagnostic substitution, or heuristic override is admissible; execution proceeds only under the declared unit
system, tolerance semantics, and readiness policy, ensuring determinism, traceability, and accountability.
}
\label{fig:governance_failure_modes_decision_closure}
\end{figure}


\subsection{Unit Ambiguity and Silent Misinterpretation}

Without governance, systems often lack a single authoritative unit system.
Tolerance metrics, readiness adjustments, and diagnostic signals may be computed
in incompatible spaces without detection.

Common manifestations include:
\begin{itemize}[leftmargin=1.5em]
  \item Applying raw-unit tolerances to quantized demand,
  \item Evaluating snapped forecasts against unsnapped diagnostics,
  \item Mixing raw and grid-aligned interpretations within a single decision.
\end{itemize}

These inconsistencies can produce internally coherent metrics that are
structurally meaningless, leading to false confidence in readiness assessments.

\subsection{Illusory Performance Improvements}

In the absence of governance constraints, apparent gains in service or accuracy
may reflect representational artifacts rather than real operational improvement.

Examples include:
\begin{itemize}[leftmargin=1.5em]
  \item Increased hit rates driven solely by tolerance choices smaller than the
        realizable demand increment,
  \item Reduced reported error achieved by snapping without explicit declaration,
  \item Readiness uplift that inflates forecasts without expanding attainable
        coverage.
\end{itemize}

Without governance, these effects are easily misinterpreted as successful
optimization rather than structural compliance.

\subsection{Improper Readiness Intervention}

Scale-based readiness adjustments are particularly sensitive to structural
misalignment. When applied without governance:
\begin{itemize}[leftmargin=1.5em]
  \item Adjustments may be applied in unit systems where they have no effect,
  \item Policies may appear unstable or ineffective despite correct logic,
  \item Operational interventions may be authorized that cannot be realized.
\end{itemize}

Such failures are often attributed incorrectly to forecasting error, parameter
choice, or model inadequacy, rather than to missing structural constraints.

\subsection{Metric Entanglement and Policy Drift}

Without a governance contract, diagnostic metrics are frequently repurposed as
decision rules. Over time, this leads to:
\begin{itemize}[leftmargin=1.5em]
  \item Threshold tuning to achieve desired outcomes,
  \item Implicit policy encoded in metric interpretation,
  \item Gradual drift in how readiness is defined and enforced.
\end{itemize}

Because these changes are rarely versioned or documented, policy evolution
becomes opaque and unreviewable.

\subsection{Non-Reproducible Decisions}

In systems without deterministic governance logic, identical conditions may
produce different readiness decisions depending on:
\begin{itemize}[leftmargin=1.5em]
  \item which diagnostics are inspected,
  \item how unit conventions are interpreted,
  \item or which signals are emphasized by individual operators.
\end{itemize}

This non-reproducibility undermines auditability, accountability, and trust in
the decision process, particularly in automated or high-stakes environments.

\subsection{Misattribution of Responsibility}

When governance is implicit, responsibility for readiness outcomes becomes
diffuse. Model developers, analysts, and operators may each assume that another
layer is enforcing structural validity.

This ambiguity leads to:
\begin{itemize}[leftmargin=1.5em]
  \item delayed detection of systemic issues,
  \item defensive interpretation of metrics,
  \item and difficulty assigning ownership for corrective action.
\end{itemize}

Explicit governance resolves this by declaring which decisions are authorized,
and on what basis.

\subsection{Summary}

These failure modes share a common root: the absence of an explicit, authoritative
decision layer that constrains interpretation before optimization.

The governance framework of Electric Barometer exists precisely to eliminate
these pathologies. By enforcing unit consistency, diagnostic exclusivity, and
deterministic policy mapping, governance transforms readiness evaluation from an
implicit, assumption-driven practice into a structurally grounded decision
system. Future extensions may expand the diagnostic vocabulary, but the
governance contract itself is intended to remain narrow, deterministic, and
closed.