% ==========================================================
% 110_relationship_to_dqc_and_fpc.tex
% Forecast Governance
% ==========================================================

\section{Relationship to DQC and FPC}
\label{sec:relationship_to_dqc_and_fpc}

Forecast Governance is not a standalone diagnostic. It is a compositional
decision layer that derives its authority entirely from the disciplined use of
two upstream constructs: Demand Quantization Compatibility (\DQC{}) and Forecast
Primitive Compatibility (\FPC{}). Each addresses a distinct structural question,
and Governance exists to bind their answers into a single, enforceable policy
outcome.

This section clarifies the respective roles of \DQC{} and \FPC{}, the ordering
constraints between them, and the manner in which Governance composes—rather
than replaces—their conclusions.

\subsection{Distinct but Complementary Roles}

\DQC{} and \FPC{} operate on different objects and answer different questions:

\begin{itemize}
  \item \textbf{\DQC{}:} Determines whether realized demand admits continuous
        interpretation at the evaluation resolution, or whether evaluation and
        control must respect a discrete support.
  \item \textbf{\FPC{}:} Determines whether a given forecast primitive responds
        meaningfully to admissible readiness adjustment under a fixed
        representation.
\end{itemize}

Neither diagnostic is sufficient on its own to authorize readiness intervention.
\DQC{} constrains the admissible unit system; \FPC{} constrains the admissible
control lever. Governance is required to reconcile these constraints into a
single policy decision.

\subsection{Strict Ordering and Dependency}

The relationship between \DQC{} and \FPC{} is strictly ordered and asymmetric:

\begin{itemize}
  \item \DQC{} must be evaluated first.
  \item \FPC{} is evaluated only within the representation declared admissible
        by \DQC{}.
\end{itemize}

This ordering is non-negotiable. Evaluating \FPC{} without first resolving demand
quantization produces ambiguous or structurally invalid signals. Governance
enforces this ordering by declaring a single authoritative coordinate system
(raw or snapped) before any compatibility assessment is considered.

\subsection{Governance as the Binding Layer}

Governance does not reinterpret or override \DQC{} or \FPC{} outputs. Instead, it
consumes them as fixed inputs and produces a closed decision artifact that
specifies:

\begin{itemize}
  \item the governing representation (raw or grid-aligned),
  \item the authoritative \FPC{} assessment,
  \item the admissibility of readiness adjustment,
  \item the interpretation and enforcement of tolerance semantics.
\end{itemize}

In this sense, Governance functions as a binding contract rather than a scoring
mechanism. It closes the decision space defined by \DQC{} and \FPC{}, eliminating
degrees of freedom that would otherwise allow inconsistent or opportunistic
interpretation.

\subsection{No Substitution or Collapsing}

It is tempting to treat Governance as a higher-level diagnostic that subsumes
\DQC{} and \FPC{}. This interpretation is incorrect.

\begin{itemize}
  \item \DQC{} cannot be inferred from \FPC{} behavior.
  \item \FPC{} cannot resolve unit admissibility.
  \item Governance cannot operate without both.
\end{itemize}

Each construct is intentionally narrow, and their separation is essential to
auditability, accountability, and long-term stability. Governance exists
precisely to preserve these separations while still enabling decisive action.

\subsection{Compositional Integrity}

Together, \DQC{}, \FPC{}, and Governance form a minimal but complete structural
stack:

\begin{center}
\emph{Demand Structure} $\rightarrow$ \emph{Primitive Compatibility}
$\rightarrow$ \emph{Policy Authority}
\end{center}

By maintaining strict boundaries between these layers and composing them
deterministically, the Electric Barometer framework ensures that readiness
decisions are not merely cost-aware or service-aware, but structurally admissible
by construction.

In this architecture, Governance is not the source of truth—it is the
\emph{enforcer} of truths established upstream.
