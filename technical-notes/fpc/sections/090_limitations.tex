% ==========================================================
% 090_limitations.tex
% Forecast Primitive Compatibility (FPC)
% ==========================================================

\section{Limitations and Extensions}

While \FPC{} provides a principled mechanism for diagnosing structural mismatch between demand
processes and forecast primitives, it is subject to several limitations that should be made explicit.
These limitations do not undermine the utility of \FPC{}, but they define the conditions under which
its classifications should be interpreted and revisited.

\subsection{Dependence on Evaluation Horizon}

\FPC{} classifications are conditional on the evaluation window over which diagnostics are computed.
Short horizons may fail to expose structural intermittency or may overweight transient effects such
as promotions, outages, or atypical demand bursts. Conversely, overly long horizons may obscure
regime shifts or recent operational changes.

As with other \FRF{} diagnostics, \FPC{} should be recomputed periodically and interpreted in the
context of known temporal changes in demand behavior.

\subsection{Sensitivity to Diagnostic Resolution}

The signals informing \FPC{} depend on the resolution at which demand is observed and forecasts are
evaluated. Aggregation can mask discreteness and intermittency, potentially producing a
\Compatible{} classification for demand that would appear \Incompatible{} at finer resolution.
Similarly, very fine-grained evaluation may exaggerate intermittency for processes that are
operationally smooth at decision-relevant timescales.

Compatibility should therefore be assessed at the resolution at which decisions are made, and
re-evaluated if that resolution changes.

\subsection{Primitive-Specific Scope}

This technical note focuses on the compatibility of point forecasting primitives augmented by
scale-based readiness adjustment. \FPC{} does not directly assess compatibility with alternative
primitives such as quantile forecasts, discrete distributions, or state-based decision rules.
Demand streams classified as \Incompatible{} with point forecasts may be well served by such
alternatives, but evaluating their suitability requires separate analysis.

Extending \FPC{} to other primitives is a natural direction for future work, but doing so requires
careful articulation of the assumptions encoded by those representations.

\subsection{Thresholds and Formalization}

Although \FPC{} is deterministic, its classification relies on qualitative patterns rather than
fixed numeric thresholds. This design choice prioritizes interpretability and auditability, but it
also introduces judgment in borderline cases. Practitioners may wish to formalize internal
guidelines for classification based on their risk tolerance and operational context.

Such formalization should be treated as an implementation detail rather than a change to the core
concept.

\subsection{Extensions}

Several extensions of \FPC{} are possible. These include:
\begin{itemize}[leftmargin=1.5em]
  \item Explicit confidence or stability measures for compatibility classification.
  \item Integration with automated policy engines that enforce readiness gating.
  \item Generalization to multi-primitive environments where different representations coexist.
\end{itemize}

These extensions are outside the scope of the present note but illustrate how \FPC{} can evolve as a
governance construct within broader decision systems.

By acknowledging these limitations, \FPC{} remains a focused diagnostic tool rather than an
overgeneralized solution, preserving its role as a safeguard for readiness evaluation.
