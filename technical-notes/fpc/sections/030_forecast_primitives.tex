% ==========================================================
% 030_forecast_primitives.tex
% Forecast Primitive Compatibility (FPC)
% ==========================================================

\section{Forecast Primitives and Structural Assumptions}

A \emph{forecast primitive} refers to the fundamental representational form through which future
demand is expressed and acted upon. Common primitives in operational settings include point
forecasts, quantile forecasts, interval forecasts, discrete count distributions, and state- or
event-based signals. Each primitive encodes implicit structural assumptions about the demand
process, regardless of the model or algorithm used to produce it.

This technical note focuses on the compatibility of \emph{point forecasting primitives}, optionally
augmented by global scale or readiness adjustments, with observed demand behavior. Point forecasts
are widely used due to their simplicity, interpretability, and ease of integration into downstream
decision systems. However, their appropriateness depends critically on properties of the underlying
process.

\subsection{Implicit Assumptions of Point Forecasting}

Point forecasts implicitly assume that demand is:
\begin{itemize}[leftmargin=1.5em]
  \item \emph{Continuous or approximately continuous}, such that deviations can be meaningfully
        measured in magnitude.
  \item \emph{Smooth or weakly intermittent}, with changes occurring gradually rather than as rare,
        discrete jumps.
  \item \emph{Scale-adjustable}, meaning that systematic bias can be mitigated through multiplicative
        or additive adjustment.
\end{itemize}

Under these assumptions, global readiness interventions such as \RAL{} can improve coverage by
inflating forecasts in a controlled manner, trading surplus for reduced shortfall. Asymmetric cost
evaluation via \CWSL{} and tolerance-based interpretation via \HRtau{} remain meaningful, and
incremental tuning produces predictable changes in readiness diagnostics.

\subsection{Demand Processes that Violate Primitive Assumptions}

Many operational demand streams violate one or more of the above assumptions. In particular,
piecewise or count-driven commodities often exhibit:
\begin{itemize}[leftmargin=1.5em]
  \item \emph{Zero inflation}, with long stretches of no demand.
  \item \emph{Bursty arrivals}, where demand appears in discrete quanta rather than continuously.
  \item \emph{Discrete support}, reflecting bundle sizes, portion counts, or fixed serving units.
\end{itemize}

When such behavior is present, point forecasts tend to miss by construction: they underpredict during
bursts and overpredict during idle periods. Global scale adjustments increase average forecast
magnitude but do not align forecasts with event timing or discrete jumps. As a result, coverage
metrics such as \NSL{} may improve slowly or not at all, tolerance-based metrics such as \HRtau{}
remain flat across wide bands, and cost-weighted loss \CWSL{} escalates rapidly as asymmetry is
increased.

\subsection{Compatibility as a Structural Property}

Forecast primitive compatibility is therefore a \emph{structural property} of the interaction
between a demand process and a representational choice, not a property of a particular model
instance. Two models producing point forecasts for the same demand stream share the same primitive
and inherit its limitations, regardless of differences in training data, features, or algorithms.

The role of \FPC{} is to make this structural dependency explicit. Rather than attributing persistent
readiness failure to insufficient tuning or poor model selection, \FPC{} asks whether the forecast
primitive itself admits meaningful improvement under admissible readiness interventions. When it
does not, continued adjustment within the same primitive is unlikely to yield operational benefit,
and alternative representations should be considered.

By articulating the assumptions encoded by forecast primitives and the conditions under which they
fail, \FPC{} provides the conceptual foundation for the diagnostic and governance procedures
introduced in subsequent sections.
