% ==========================================================
% 080_interpretive_boundaries_and_misuse.tex
% Forecast Primitive Compatibility (FPC)
% ==========================================================

\section{Interpretive Boundaries and Common Misuse}

Having established the role of \FPC{} in governance and policy gating, we now clarify the
interpretive boundaries and common forms of misuse that \FPC{} is designed to prevent. Because its
output is categorical and governance-oriented, \FPC{} is susceptible to misinterpretation if
treated as a performance score, optimization target, or proxy for forecast quality. This section
therefore establishes explicit interpretive boundaries and documents common forms of misuse that
\FPC{} is designed to prevent.

\subsection{Compatibility Is Not Performance}

An \textit{FPC-Compatible} classification does not imply that a forecast is accurate, efficient, or
operationally satisfactory. It indicates only that the forecast primitive admits meaningful
responsiveness to admissible readiness adjustment under the observed demand behavior.

Conversely, an \textit{FPC-Incompatible} classification does not imply poor modeling, low data
quality, or operational irrelevance. It signals that the chosen forecast primitive is structurally
misaligned with the demand process such that scale-based readiness intervention is ineffective or
ill-posed. Compatibility is therefore orthogonal to forecast quality and must not be interpreted as
a performance judgment.

\subsection{Compatibility Is Not an Optimization Target}

\FPC{} classifications are not quantities to be maximized, minimized, or tuned. Attempting to
optimize diagnostics or adjustment policies in order to ``improve'' compatibility constitutes a
category error.

The purpose of \FPC{} is to determine whether further optimization within a given forecast
primitive is structurally defensible at all. When incompatibility is observed, continued tuning,
calibration, or adjustment within the same primitive is unlikely to yield meaningful readiness
improvement. In such cases, the appropriate response is representational change rather than
parameter refinement.

\subsection{Conditional and Context-Dependent Interpretation}

\FPC{} classifications are conditional on the evaluation horizon, resolution, and admissible unit
system. Changes in aggregation level, operational cadence, or demand regime may alter observable
diagnostic behavior and therefore warrant re-evaluation.

As a result, compatibility classifications should be treated as time- and context-specific
governance signals rather than permanent attributes of an entity. Stability over time increases
confidence in the diagnosis, but absence of stability does not invalidate the construct. Periodic
reassessment is expected in dynamic operational environments.

\subsection{Separation from Policy Authority}

As established in Section~6, \FPC{} does not authorize or prohibit readiness actions. Its role is
limited to diagnosing primitive responsiveness.

Using \FPC{} classifications directly as action triggers—without governance mediation—breaks the
separation of concerns that underpins auditability and stability in the Forecast Readiness
Framework. \FPC{} must therefore be consumed only as an input to governance, not as a standalone
policy rule.

\subsection{Intended Use}

Proper use of \FPC{} consists of:
\begin{itemize}
    \item Diagnosing whether readiness adjustment within a forecast primitive is structurally
    meaningful.
    \item Constraining optimization and calibration workflows when incompatibility is detected.
    \item Informing governance decisions regarding admissible readiness policies.
    \item Redirecting analytical effort toward alternative primitives when warranted.
\end{itemize}

By enforcing these interpretive boundaries, \FPC{} functions as a safeguard against unproductive
tuning, misapplied policy, and misleading evaluation. Its value lies not in ranking forecasts, but
in preserving the structural validity of readiness decisions.
