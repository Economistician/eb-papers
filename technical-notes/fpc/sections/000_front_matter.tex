% ==========================================================
% 000_frontmatter.tex
% FRONTMATTER (FPC Technical Note)
% ==========================================================

\begin{abstract}
Forecast Readiness Framework (FRF) metrics quantify operational readiness under
asymmetric cost and service risk, but they do not guarantee that a given forecasting
\emph{primitive} (e.g., point forecasts with global scale adjustment) is structurally appropriate
for the underlying demand process. In practice, certain commodity streams exhibit discrete,
intermittent, or piecewise behavior (e.g., bursty counts and zero-inflation) such that incremental
tuning, cost-ratio selection, or readiness adjustment policies may yield only limited coverage gains.
At the same time, these interventions can inflate surplus, destabilize tolerance metrics, or
produce governance signals that are difficult to interpret or defend.

This technical note introduces \emph{Forecast Primitive Compatibility (FPC)} as a derived,
auditable classification that diagnoses structural mismatch between demand behavior and a forecast
primitive. FPC is not a performance metric and is not intended as an optimization objective; rather,
it is a readiness governance construct that consumes existing FRF diagnostics (e.g., \CWSL, \NSL,
\UD, and hit-rate tolerance curves) to determine whether a class of readiness interventions is
valid and defensible. We define observable incompatibility signals, propose a deterministic
compatibility taxonomy, and describe governance implications for policy gating and model selection.
\end{abstract}
