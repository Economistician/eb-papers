% ==========================================================
% 060_companion_diagnostics_and_governance.tex
% Forecast Primitive Compatibility (FPC)
% ==========================================================

\section{Companion Diagnostics and Governance}

Forecast Primitive Compatibility (\FPC{}) is not a standalone decision mechanism. It is designed
to operate as an intermediate structural diagnostic within a layered governance stack that separates
measurement validity, primitive responsiveness, and policy authority. This section clarifies the
relationship between \FPC{}, Demand Quantization Compatibility (\DQC{}), and Forecast Governance,
and delineates the boundaries of responsibility among these components.

\subsection{Relationship to Demand Quantization Compatibility}

\FPC{} presumes that forecast evaluation and readiness diagnostics are interpreted in an admissible
unit system. Determination of that unit system is explicitly out of scope for \FPC{} and is delegated
to Demand Quantization Compatibility (\DQC{}).

\DQC{} characterizes the structural support of realized demand and determines whether evaluation and
control should be interpreted in raw units or in grid-aligned (snapped) units. \FPC{} does not infer
demand quantization, estimate granularity, or impose snapping behavior. Instead, \FPC{} operates
conditionally on the evaluation space determined upstream by \DQC{}.

As a consequence, \FPC{} classifications are meaningful only with respect to a fixed unit system.
A forecast primitive may appear weakly responsive or unstable if evaluated in an ill-posed
coordinate system, even if it is structurally compatible when interpreted on an appropriate grid.
Conversely, a primitive that is incompatible under a valid unit system remains incompatible
regardless of tuning or adjustment.

\subsection{Relationship to Forecast Governance}

\FPC{} does not itself authorize, prohibit, or condition readiness interventions. Its output is a
structural classification of whether a forecast primitive admits meaningful responsiveness to
admissible readiness adjustment. Final policy decisions are the responsibility of the governance
layer.

Forecast Governance composes \FPC{} with \DQC{} to produce a single, authoritative decision
artifact. In this composition, \FPC{} supplies evidence regarding primitive responsiveness, while
\DQC{} supplies evidence regarding admissible evaluation units. Governance resolves these inputs
into explicit policy outcomes, including tolerance interpretation rules, snapping requirements,
readiness-adjustment allowability, and categorical status.

A \textit{Compatible} \FPC{} classification therefore indicates that readiness adjustment is
structurally meaningful in principle, not that it is unconditionally permitted. Likewise, a
\textit{Marginal} or \textit{Incompatible} classification constrains governance decisions but does
not itself constitute a policy ruling. Governance is the only layer that binds diagnostic evidence
to enforceable decisions.

\subsection{Position in the Forecast Readiness Framework}

Within the Forecast Readiness Framework (\FRF{}), \FPC{} occupies a distinct intermediate position
between metric evaluation and policy enforcement. The conceptual flow may be summarized as:
\[
\text{Forecast Output} \;\rightarrow\; \text{\FRF{} Metrics} \;\rightarrow\; \FPC{} \;\rightarrow\;
\text{Governance Decision}.
\]

In this stack, \FPC{} serves as a governance boundary that constrains continued optimization,
calibration, or readiness adjustment within a forecast primitive when empirical diagnostics
indicate that such interventions are ineffective or ill-posed. By enforcing this boundary, \FPC{}
protects downstream decision systems from unproductive tuning and ensures that governance
decisions are grounded in structural validity rather than metric behavior alone.

By clearly separating unit-system admissibility (\DQC{}), primitive responsiveness (\FPC{}), and
policy authority (Governance), the framework preserves auditability, interpretability, and stability
across evolving operational contexts.
