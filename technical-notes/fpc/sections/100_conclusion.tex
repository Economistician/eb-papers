% ==========================================================
% 100_conclusion.tex
% Forecast Primitive Compatibility (FPC)
% ==========================================================

\section{Conclusion}

Forecast readiness evaluation presumes not only that forecast error can be measured, but that the
chosen forecast primitive is structurally appropriate for the demand process it represents. When
this presumption fails, continued tuning, asymmetric evaluation, and readiness adjustment can
produce misleading signals and unproductive governance behavior. \FPC{} addresses this gap by making
structural compatibility explicit.

By consuming observable diagnostics from the \FRF{}—including \NSL{}, \UD{}, \HRtau{}, and \CWSL{}—
\FPC{} provides a deterministic, auditable classification of whether a point forecasting primitive
admits meaningful readiness improvement under admissible interventions. In doing so, it reframes
persistent readiness failure not as a modeling deficiency, but as evidence of representational
mismatch.

The introduction of \FPC{} completes the readiness evaluation stack. It supplies a principled stop
condition that constrains when readiness policies such as global adjustment and cost-ratio
calibration are valid, and when they are not. For demand streams classified as \Incompatible{},
\FPC{} redirects attention toward alternative representations or decision rules better aligned with
structural demand behavior.

More broadly, \FPC{} reinforces the governance orientation of the Electric Barometer. Rather than
encouraging perpetual optimization within a fixed paradigm, it emphasizes diagnostic clarity,
methodological humility, and operational defensibility. In environments where demand behavior
violates the assumptions of point forecasting, the most responsible decision is not to tune harder,
but to change the primitive.
