% ==========================================================
% 070_governance_implications.tex
% Forecast Primitive Compatibility (FPC)
% ==========================================================

\section{Governance and Policy Implications}

The primary role of \FPC{} is to constrain and guide readiness policy. By diagnosing whether a
forecast primitive is structurally compatible with observed demand behavior, \FPC{} establishes
clear boundaries on which interventions are defensible and which are not. Building on the governance
structure established in the previous section, this section describes how \FPC{} classifications are
resolved into policy constraints and decision guidance, and outlines how \FPC{} should be used as a
policy gate within the \FRF{}.

\subsection{Policy Gating}

Once a compatibility classification has been established, it should act as an explicit gate on the
application of readiness interventions. In particular:
\begin{itemize}[leftmargin=1.5em]
  \item \Compatible{} entities admit the full range of readiness policies, including \RAL{},
        asymmetric cost evaluation via \CWSL{}, and cost-ratio calibration.
  \item \Marginal{} entities admit readiness intervention only with additional safeguards, such as
        capped adjustment ranges, enhanced reporting, or periodic re-evaluation.
  \item \Incompatible{} entities should not be subject to scale-based readiness adjustment or
        cost-ratio tuning, as such interventions do not address the underlying source of readiness
        failure.
\end{itemize}

This gating behavior prevents the misapplication of otherwise valid policies in regimes where they
are structurally ineffective.

\subsection{Implications for Cost-Asymmetry Governance}

Cost-weighted evaluation assumes that trade-offs between underbuild and overbuild can be meaningfully
managed through adjustment of asymmetry parameters. When a demand stream is \Compatible{}, this
assumption holds: increasing $\R{}$ amplifies readiness risk in a controlled manner, and calibration
procedures yield interpretable results.

For \Incompatible{} entities, however, cost-weighted metrics can become misleading. Rapid escalation
of \CWSL{} with increasing $\R{}$ reflects structural underbuild that cannot be mitigated by scale
adjustment. In such cases, cost-ratio calibration may converge to extreme or counterintuitive values,
and reporting cost-weighted loss without qualification obscures the true source of risk. \FPC{}
therefore provides a principled basis for disabling or annotating asymmetric cost analysis when it is
not governance-valid.

\subsection{Model Comparison and Selection}

Forecast comparison exercises implicitly assume that models share a compatible primitive. When this
assumption is violated, relative performance rankings are unstable and often driven by artifacts of
intermittency rather than substantive differences in readiness. \FPC{} identifies when such
comparisons are inappropriate.

For \Incompatible{} entities, model selection within a point-forecasting paradigm is unlikely to be
informative. Governance processes should instead redirect effort toward alternative representations
or decision rules. For \Marginal{} entities, comparisons may be reported but should be interpreted
with caution and supplemented by structural diagnostics.

\subsection{Reporting and Auditability}

Because \FPC{} is derived from observable diagnostics, its classifications can be audited and
reproduced. Governance reporting should include:
\begin{itemize}[leftmargin=1.5em]
  \item The underlying diagnostic signals informing classification.
  \item The compatibility state assigned to each entity.
  \item Any policy constraints imposed as a result of that classification.
\end{itemize}

Making these elements explicit supports transparent decision-making and prevents silent escalation
of readiness interventions in incompatible regimes.

\subsection{Role of Domain Knowledge}

While \FPC{} does not depend on domain labeling, domain knowledge remains valuable for interpretation
and validation. Observed alignment between \FPC{} classifications and known demand properties (e.g.,
piecewise or bundle-driven items) increases confidence in governance decisions. However, policy
enforcement should be based on observable behavior rather than manual flags, ensuring that
compatibility judgments remain empirical and adaptive.

By formalizing how compatibility constrains readiness policy, \FPC{} reinforces the governance
orientation of the \FRF{} and protects operational decision systems from unproductive or misleading
intervention.
