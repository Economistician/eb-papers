% ==========================================================
% 040_observable_signals.tex
% Forecast Primitive Compatibility (FPC)
% ==========================================================

\section{Observable Signals of Incompatibility}

\FPC{} is grounded entirely in observable behavior. It does not rely on demand annotations, causal
models, or assumptions about the data-generating process beyond those implied by the forecast
primitive itself. Instead, incompatibility is diagnosed through the joint behavior of existing
\FRF{} diagnostics evaluated under baseline forecasts and under admissible readiness interventions.
This section describes the primary observable signals that indicate structural mismatch between a
demand process and a point forecasting primitive.

\subsection{Persistent Low Coverage}

A necessary condition for compatibility is that baseline coverage, as measured by \NSL{}, is
meaningfully greater than zero. When \NSL{} remains persistently low across the evaluation horizon,
the forecast primitive fails to meet realized demand in most intervals. While low coverage alone does
not imply incompatibility, it establishes a baseline against which responsiveness must be assessed.

In compatible regimes, low baseline coverage typically reflects systematic bias or under-scaling and
can be mitigated through controlled adjustment. In incompatible regimes, low coverage is structural
and persists regardless of adjustment magnitude.

\subsection{Weak Responsiveness to Readiness Adjustment}

A central diagnostic for \FPC{} is the responsiveness of coverage to readiness adjustment. Let
\NSL{} denote baseline coverage and let $\NSL^{(\alpha)}$ denote coverage after applying a readiness
adjustment \RAL{} with parameter $\alpha$. The change
\[
\dNSL{} = \NSL^{(\alpha)} - \NSL{}
\]
captures the effectiveness of scale-based intervention.

When $\dNSL{}$ remains small across a wide range of $\alpha$, the demand process is effectively
insensitive to scale adjustment. This behavior indicates that coverage failures are driven by timing,
intermittency, or discreteness rather than magnitude, and that further adjustment will primarily
inflate surplus without delivering readiness gains.

\subsection{Flat Tolerance Response}

Tolerance-based hit rates provide a complementary signal. For a tolerance band $\tauval$, the metric
\HRtau{} measures the fraction of intervals in which forecast error lies within an acceptable range.
In compatible regimes, increasing $\tauval$ yields a smooth and monotone increase in \HRtau{}, and
moderate tolerance bands capture a substantial fraction of intervals.

In incompatible regimes, \HRtau{} remains low even for large $\tauval$. Flat tolerance response
indicates that errors are not small perturbations around realized demand but are instead dominated by
large, discrete misses. This behavior is particularly indicative of zero-inflated or bursty demand
processes.

\subsection{High Conditional Shortfall Severity}

Conditional severity metrics such as \UD{} reveal the depth of failures when they occur. High values
of \UD{} combined with low shortfall frequency suggest that misses are infrequent but large. In such
cases, scale-based adjustments may reduce the number of shortfalls marginally while leaving their
depth largely unchanged, resulting in poor trade-offs between coverage and surplus.

When high \UD{} coincides with weak responsiveness of \NSL{} and \HRtau{}, it provides strong evidence
that the forecast primitive is misaligned with the structure of demand.

\subsection{Rapid Escalation of Asymmetric Cost}

Finally, incompatibility is often reflected in the curvature of cost-weighted loss as asymmetry
increases. Under \CWSL{}, increasing the cost ratio $\R{}$ amplifies the penalty for underbuild. In
compatible regimes, \CWSL{} typically increases smoothly with $\R{}$, reflecting controlled trade-offs
between underbuild and overbuild.

In incompatible regimes, \CWSL{} escalates rapidly as $\R{}$ increases, often without corresponding
improvements in coverage. Such behavior indicates that underbuild dominates cost exposure in a manner
that cannot be mitigated by scale adjustment, rendering cost-ratio calibration and asymmetric
evaluation misleading from a governance perspective.

\subsection{Joint Interpretation}

No single signal is sufficient to diagnose incompatibility. \FPC{} is informed by the \emph{joint}
behavior of these observables. Persistent low coverage, weak $\dNSL{}$, flat \HRtau{} response, high
\UD{}, and rapid \CWSL{} escalation together constitute a robust empirical signature of structural
mismatch. These signals are synthesized into a compatibility classification, described next, which
provides the basis for constraining readiness policy through governance.
