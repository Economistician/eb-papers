% ==========================================================
% 050_classification.tex
% Forecast Primitive Compatibility (FPC)
% ==========================================================

\section{Compatibility Classification}

The observable signals described in the previous section motivate a qualitative classification of
forecast primitive compatibility. The purpose of this classification is not to score or rank demand
streams, but to determine whether a given forecast primitive admits meaningful readiness improvement
under admissible interventions. \FPC{} therefore produces a small set of interpretable compatibility
states rather than a continuous metric.

\subsection{Compatibility States}

We define three compatibility classes:
\begin{itemize}[leftmargin=1.5em]
  \item \Compatible{}
  \item \Marginal{}
  \item \Incompatible{}
\end{itemize}

Each class corresponds to a characteristic pattern of response across \FRF{} diagnostics. The
classification is intended to be deterministic and auditable, based on observable behavior rather
than fitted parameters or statistical inference.

\subsection{\Compatible{}}

A demand stream is classified as \Compatible{} with a point forecasting primitive when readiness
interventions produce predictable and defensible improvements. Typical characteristics include:
\begin{itemize}[leftmargin=1.5em]
  \item Baseline \NSL{} that is materially greater than zero.
  \item Positive and meaningful $\dNSL{}$ under moderate \RAL{}.
  \item A smooth and increasing \HRtau{} response as $\tauval$ grows.
  \item Moderate \UD{}, indicating that shortfalls are not dominated by extreme events.
  \item Gradual escalation of \CWSL{} as cost asymmetry $\R{}$ increases.
\end{itemize}

In this regime, coverage failures are primarily driven by scale or bias rather than timing or
discreteness. Global readiness adjustment and cost-aware governance are therefore valid tools for
improving operational readiness.

\subsection{\Marginal{}}

A demand stream is classified as \Marginal{} when some readiness improvement is achievable, but
responses are weak, unstable, or accompanied by significant trade-offs. Common patterns include:
\begin{itemize}[leftmargin=1.5em]
  \item Low to moderate baseline \NSL{}.
  \item Small or inconsistent $\dNSL{}$ under \RAL{}.
  \item Modest increases in \HRtau{} only at large tolerance levels.
  \item Elevated \UD{}, suggesting the presence of intermittent deep misses.
  \item Noticeable curvature in \CWSL{} with increasing $\R{}$.
\end{itemize}

Marginal compatibility indicates mixed failure modes. Scale-based adjustment may yield incremental
benefits, but such interventions should be applied cautiously and accompanied by explicit governance
controls. Alternative primitives or hybrid approaches may be warranted, though incompatibility is
not yet definitive.

\subsection{\Incompatible{}}

A demand stream is classified as \Incompatible{} when readiness diagnostics indicate structural
misalignment with the forecast primitive. This regime is characterized by:
\begin{itemize}[leftmargin=1.5em]
  \item Persistently low \NSL{}, often near zero.
  \item Negligible $\dNSL{}$ across a wide range of \RAL{} adjustments.
  \item Flat \HRtau{} response even for large tolerance bands.
  \item High \UD{} driven by infrequent but severe shortfalls.
  \item Rapid escalation of \CWSL{} with increasing cost asymmetry.
\end{itemize}

In this regime, further tuning within the same primitive is unlikely to yield meaningful readiness
improvement. Scale adjustment primarily inflates surplus without addressing the structural source of
misses. For such entities, readiness policies predicated on point forecasting are not defensible, and
continued optimization should be replaced by a change in representational approach.

\subsection{Interpretation and Use}

The \FPC{} classification should be interpreted as a governance signal rather than a performance
judgment. A designation of \Incompatible{} does not imply that a demand stream is unimportant or
unmanageable; rather, it indicates that the chosen forecast primitive is ill-suited to its behavior.
Conversely, a \Compatible{} classification does not guarantee high readiness, only that readiness
interventions are structurally valid.

By collapsing complex diagnostic behavior into a small number of interpretable states, \FPC{}
provides a practical mechanism for constraining readiness policy, preventing unproductive tuning,
and directing attention toward appropriate modeling or decision alternatives. These classifications
acquire operational meaning only when situated within the broader governance stack, described next.
