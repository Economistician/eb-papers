% ==========================================================
% 010_overview.tex
% Forecast Primitive Compatibility (FPC)
% ==========================================================

\section{Overview}

Forecast readiness evaluation assumes not only that forecast errors can be measured, but that the
\emph{forecast primitive} used to generate those errors is structurally appropriate for the demand
process under consideration. In many operational settings, this assumption is violated. Commodity
streams may be discrete, intermittent, or piecewise, exhibiting long runs of zero demand punctuated
by bursty arrivals at specific quanta (e.g., counts and bundles). Under such behavior, conventional
point forecasting---even when evaluated with asymmetric cost metrics---can produce persistent
under-coverage and deep shortfalls that do not respond meaningfully to incremental tuning.

The Forecast Readiness Framework (\FRF{}) provides a family of diagnostics for asymmetric operational
risk, including cost-weighted loss (\CWSL{}), coverage-oriented measures (\NSL{}), conditional shortfall
severity (\UD{}), and tolerance-based hit rates (\HRtau{}). These metrics characterize \emph{how} a
forecast fails, but they do not by themselves determine whether a class of readiness interventions is
\emph{valid}. In practice, practitioners often apply additional levers---global scale adjustments,
buffer policies, or readiness adjustment layers (\RAL{})---to improve service coverage. When the demand
process is compatible with the forecast primitive, such interventions can increase coverage and
reduce readiness loss in a controlled manner. When it is not, the same interventions may yield
negligible gains in coverage while inflating surplus, degrading tolerance metrics, and producing
governance signals that are difficult to interpret or defend.

This technical note introduces \emph{Forecast Primitive Compatibility} (\FPC{}) as a derived,
auditable classification that diagnoses structural mismatch between a demand process and a forecast
primitive. \FPC{} is not a performance metric and is not intended as an optimization objective.
Rather, it is a governance construct that consumes observable \FRF{} diagnostics to determine whether
a forecast primitive admits meaningful readiness improvement under admissible interventions. The
central premise is simple: if coverage and tolerance do not respond to readiness levers in a
predictable way, the appropriate conclusion is not ``tune harder,'' but ``change the primitive.''

We formalize the notion of forecast primitives and the structural assumptions they implicitly encode,
identify observable incompatibility signals (including weak responsiveness of \NSL{} to \RAL{}, flat
tolerance response curves under \HRtau{}, high conditional shortfall depth \UD{}, and rapid escalation of
\CWSL{} under increasing cost asymmetry), and propose a deterministic compatibility taxonomy:
\Compatible, \Marginal, and \Incompatible. We then outline the governance implications of \FPC{} for
readiness policy gating, cost-ratio calibration, and model selection.

The goal of \FPC{} is to prevent the misuse of readiness interventions in regimes where they are
structurally ineffective, to make limitations explicit and defensible, and to direct attention toward
appropriate alternative primitives (e.g., discrete, hurdle, quantile, or state-based approaches) when
necessary. In doing so, \FPC{} completes the readiness evaluation stack by introducing an explicit
governance boundary that constrains when readiness interventions are structurally admissible.
