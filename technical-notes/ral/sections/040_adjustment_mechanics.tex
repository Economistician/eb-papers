% ----------------------------------------------------------
% ADJUSTMENT MECHANICS
% ----------------------------------------------------------
\section{Adjustment Mechanics}

The Readiness Adjustment Layer (RAL) operates by evaluating a constrained set of
candidate forecast adjustments and selecting the modification that yields the
lowest decision-relevant loss under predefined readiness constraints. The
mechanism is deterministic, bounded, and applied entirely downstream of forecast
generation. This section describes the control logic governing the adjustment
process, independent of any specific implementation.

\subsection{Candidate Adjustment Set}

Given a baseline forecast \(\yhat\) and a feasible adjustment set
\(\deltaset = [\deltamin, \deltamax]\), RAL constructs a finite collection of
candidate adjusted forecasts of the form
\[
\yhat^{(\delta)} = \yhat + \delta, \quad \delta \in \deltaset.
\]
The feasible set \(\deltaset\) is externally defined and reflects operational
constraints such as buffer limits, capacity flexibility, or policy-imposed
guardrails. RAL does not expand or relax these bounds under any circumstances.

% ----------------------------------------------------------
% FIGURE: RAL CONTROL ENVELOPE
% ----------------------------------------------------------
\begin{figure}[htbp]
\centering
\begin{tikzpicture}
\begin{axis}[
    width=0.95\textwidth,
    height=0.42\textwidth,
    xlabel={Forecast index / time},
    ylabel={Quantity},
    xmin=0, xmax=10,
    ymin=0, ymax=12,
    grid=both,
    legend style={at={(0.02,0.98)},anchor=north west},
    ticklabel style={font=\small},
    label style={font=\small},
    legend cell align={left},
]

% --- Baseline forecast (stylized curve) ---
\addplot+[
    mark=none,
    thick,
    solid
]
coordinates {
    (0,3.0) (1,3.4) (2,3.8) (3,4.2) (4,4.6)
    (5,5.0) (6,5.3) (7,5.6) (8,5.9) (9,6.1) (10,6.3)
};
\addlegendentry{Baseline forecast $\yhat$}

% --- Lower bound ---
\addplot+[
    mark=none,
    thick,
    dashed,
    red
]
coordinates {
    (0,2.6) (1,3.0) (2,3.4) (3,3.8) (4,4.2)
    (5,4.6) (6,4.9) (7,5.2) (8,5.5) (9,5.7) (10,5.9)
};
\addlegendentry{Lower bound $\yhat_{\min}$}

% --- Upper bound ---
\addplot+[
    mark=none,
    thick,
    dashed,
    brown
]
coordinates {
    (0,3.6) (1,4.0) (2,4.4) (3,4.8) (4,5.2)
    (5,5.6) (6,6.0) (7,6.3) (8,6.6) (9,6.8) (10,7.0)
};
\addlegendentry{Upper bound $\yhat_{\max}$}

% --- Shaded feasibility envelope (requires \usepgfplotslibrary{fillbetween}) ---
\addplot[name path=upper, draw=none]
coordinates {
    (0,3.6) (1,4.0) (2,4.4) (3,4.8) (4,5.2)
    (5,5.6) (6,6.0) (7,6.3) (8,6.6) (9,6.8) (10,7.0)
};
\addplot[name path=lower, draw=none]
coordinates {
    (0,2.6) (1,3.0) (2,3.4) (3,3.8) (4,4.2)
    (5,4.6) (6,4.9) (7,5.2) (8,5.5) (9,5.7) (10,5.9)
};
\addplot[
    draw=none,
    fill=gray,
    fill opacity=0.12,
] fill between[of=upper and lower];

% --- Adjusted (readiness) forecast (solid black) ---
\addplot+[
    mark=none,
    very thick,
    black,
    solid
]
coordinates {
    (0,3.3) (1,3.7) (2,4.1) (3,4.5) (4,4.9)
    (5,5.3) (6,5.7) (7,6.0) (8,6.3) (9,6.5) (10,6.7)
};
\addlegendentry{Adjusted forecast $\yhatadj$}

% --- Annotation for "feasibility envelope" ---
\node[anchor=west, font=\small]
at (axis cs:6.2,7.55) {Feasibility envelope};

% --- Annotation arrow to shaded region ---
\draw[->, thick]
(axis cs:6.1,7.35) -- (axis cs:6.4,6.3);

\end{axis}
\end{tikzpicture}
\caption{
Conceptual illustration of the Readiness Adjustment Layer (RAL) as a bounded
control layer operating downstream of forecasting. RAL selects an adjusted
forecast $\yhatadj$ within a pre-defined feasibility envelope (between
$\yhat_{\min}$ and $\yhat_{\max}$) around the baseline forecast $\yhat$ by
minimizing decision-relevant loss under asymmetric operational cost.
If no improvement is available within the envelope, RAL defaults to the identity
transformation $\yhatadj = \yhat$.
}
\label{fig:ral_control_envelope}
\end{figure}

Figure~\ref{fig:ral_control_envelope} illustrates the bounded feasibility
envelope within which RAL operates. Candidate adjustments are evaluated only
within this predefined region, ensuring that readiness modifications remain
interpretable, auditable, and aligned with executional reality rather than
unconstrained post-hoc tuning.

The adjustment set may be evaluated at discrete resolution, but the choice of
resolution does not alter the conceptual behavior of the mechanism: all candidate
forecasts are bounded transformations of the original forecast.

\subsection{Structured Adjustment Policies}
\label{subsec:structured_adjustment_policies}

While the adjustment set \(\deltaset\) is often expressed as a simple bounded
interval, operational feasibility constraints are frequently encoded not as
continuous ranges but as discrete policy rules. RAL does not require candidate
adjustments to be uniformly distributed or homogeneous across the forecast space.
Instead, the feasible set may be \emph{structured} according to predefined policy
rules that partition the forecast domain into regions of differing operational
risk.

In such cases, the candidate adjustment set is implicitly defined by a policy
artifact that specifies both:
\begin{itemize}[leftmargin=*]
    \item threshold boundaries that delineate forecast risk regions, and
    \item region-specific adjustment magnitudes permitted within each region.
\end{itemize}

For example, a piecewise policy may define distinct adjustment behavior for
baseline forecasts in low-, medium-, and high-risk regimes, such that
\[
\yhatadj
=
\begin{cases}
\yhat, & \yhat < \tau_{\text{mid}} \\
\yhat + \delta_{\text{mid}}, & \tau_{\text{mid}} \le \yhat < \tau_{\text{high}} \\
\yhat + \delta_{\text{high}}, & \yhat \ge \tau_{\text{high}},
\end{cases}
\]
where \(\tau_{\text{mid}}\) and \(\tau_{\text{high}}\) are externally specified
thresholds and \(\delta_{\text{mid}}, \delta_{\text{high}} \in \deltaset\).

Importantly, these structured policies do not alter the fundamental decision
logic of RAL. They simply define the candidate set over which loss is evaluated.
Thresholds and region-specific adjustments are treated as fixed policy inputs,
not learned parameters, preserving determinism and auditability.

This formulation allows RAL to express operationally meaningful control behavior
(e.g., heightened conservatism near capacity limits) while maintaining the core
principle that \emph{all adjustments remain bounded, explicit, and loss-governed}.
The selection rule, fallback behavior, and loss evaluation process remain
unchanged.

\subsection{Loss Evaluation}

Each candidate adjusted forecast is evaluated using a decision-oriented loss
function, typically the cost-weighted service loss \(\CWSL(\cdot)\). This loss
function encodes asymmetric penalties associated with under- and over-forecasting
and reflects the relative severity of readiness failures.

Treating cost asymmetry as an explicit input to the decision rule—rather than as a
model tuning artifact—aligns with established principles of cost-sensitive
decision theory, in which costs are first-class objects governing optimal action
selection rather than parameters to be optimized away \citep{elkan2001}.

Loss evaluation is performed consistently across all candidates. RAL does not
apply stochastic weighting, adaptive learning, or context-specific overrides
during evaluation. As a result, identical inputs produce identical loss values and
identical adjustment decisions.

\subsection{Selection Rule}

RAL selects the adjustment \(\delta^{*}\) that minimizes decision-relevant loss
over the feasible set:
\[
\delta^{*}
=
\argmin_{\delta \in \deltaset}
\CWSL(\yhat + \delta).
\]
The adjusted forecast is then defined as
\[
\yhatadj = \yhat + \delta^{*}.
\]
When multiple candidate adjustments yield equivalent loss values, RAL applies a
consistent tie-breaking rule that favors minimal deviation from the baseline
forecast. This preference preserves interpretability and limits unnecessary
forecast distortion.

\subsection{Control and Fallback Behavior}

RAL enforces two explicit control conditions:
\begin{itemize}[leftmargin=*]
    \item \textbf{Boundedness:} All adjustments remain within the predefined
    feasible set \(\deltaset\).
    \item \textbf{Non-degradation:} If no candidate adjustment produces a loss
    improvement relative to the baseline forecast, RAL defaults to the identity
    transformation \(\yhatadj = \yhat\).
\end{itemize}

These controls ensure that RAL cannot introduce unbounded behavior, amplify model
errors, or degrade performance under the specified loss function. The mechanism
is therefore corrective rather than speculative, prioritizing robustness and
operational safety over aggressive optimization.

\subsection{Determinism and Auditability}

The adjustment process is fully deterministic. Given identical forecasts,
adjustment bounds, and loss parameters, RAL will always produce the same adjusted
forecast and diagnostic outputs. This property supports reproducibility, audit
replay, and post-hoc inspection of adjustment decisions.