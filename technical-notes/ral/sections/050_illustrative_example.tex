% ----------------------------------------------------------
% ILLUSTRATIVE EXAMPLE
% ----------------------------------------------------------
\section{Illustrative Example}

This section provides a simplified example to illustrate how the Readiness
Adjustment Layer (RAL) operates in practice. The example is intentionally
synthetic and is not intended to represent any specific operational system.

\subsection{Baseline Forecast and Constraints}

Consider a single-period demand forecast with baseline value \(\yhat = 100\).
Suppose the operational context permits limited upward or downward adjustment,
with a feasible adjustment set defined as
\[
\deltaset = [-5, +10].
\]
Negative adjustments correspond to reducing preparedness, while positive
adjustments increase readiness through additional capacity or buffer allocation.

Assume realized demand is \(y = 108\), and that under-forecasting incurs a higher
penalty than over-forecasting due to lost demand or service degradation.

\subsection{Loss Evaluation}

Let the decision-oriented loss function \(\CWSL(\cdot)\) impose asymmetric cost,
with under-forecasting penalized more heavily than over-forecasting. Under this
loss structure:
\begin{itemize}[leftmargin=*]
    \item The baseline forecast \(\yhat = 100\) results in under-forecasting and
    incurs a relatively high loss.
    \item Modest upward adjustments reduce under-readiness and lower the
    associated penalty.
    \item Excessive upward adjustments increase over-preparation cost without
    additional benefit.
\end{itemize}

RAL evaluates candidate adjusted forecasts \(\yhat + \delta\) for
\(\delta \in \deltaset\) and computes the corresponding loss for each candidate.

\subsection{Selected Adjustment}

Suppose the loss-minimizing adjustment is \(\delta^{*} = +6\), yielding an adjusted
forecast \(\yhatadj = 106\). This adjustment reduces under-forecasting while
remaining within the allowable bounds.

If no adjustment within \(\deltaset\) produced a loss improvement relative to the
baseline, RAL would instead select the identity transformation
\(\yhatadj = \yhat\).

\subsection{Interpretation}

This example highlights several key properties of RAL:
\begin{itemize}[leftmargin=*]
    \item The adjustment is driven by asymmetric cost rather than point accuracy.
    \item The selected adjustment is bounded and operationally interpretable.
    \item The mechanism prioritizes readiness improvement while avoiding
    unnecessary forecast distortion.
\end{itemize}

The example demonstrates how RAL modifies forecasts only when doing so improves
decision-relevant outcomes under the specified loss and constraints, reinforcing
its role as a controlled post-forecast adjustment layer.