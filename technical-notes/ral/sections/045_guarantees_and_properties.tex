% ----------------------------------------------------------
% GUARANTEES AND BEHAVIORAL PROPERTIES
% ----------------------------------------------------------
\section{Guarantees and Behavioral Properties}

The Readiness Adjustment Layer (RAL) is designed as a bounded, deterministic
control mechanism rather than an unconstrained optimization procedure. This
section describes the behavioral properties that RAL guarantees by construction,
as well as properties that hold only under stated conditions. These distinctions
are critical for correct interpretation and auditability.

\subsection{Determinism}

RAL is fully deterministic. Given identical inputs—including the baseline forecast
\(\yhat\), feasible adjustment set \(\deltaset\), and loss function parameters—the
adjusted forecast \(\yhatadj\) is uniquely determined. The adjustment process
contains no stochastic components, adaptive learning, or stateful dependencies.

This property ensures reproducibility, supports audit replay, and allows
post-hoc inspection of adjustment decisions without ambiguity.

\subsection{Boundedness}

All adjustments produced by RAL are guaranteed to lie within the predefined
feasible set \(\deltaset = [\deltamin, \deltamax]\). RAL does not extrapolate beyond
these bounds, relax constraints, or introduce nonlinear transformations of the
forecast.

Boundedness ensures that adjusted forecasts remain operationally interpretable
and consistent with externally imposed readiness constraints.

\subsection{Non-Degradation Under the Specified Loss}

RAL guarantees non-degradation with respect to the specified decision-oriented
loss function. Specifically, the adjusted forecast \(\yhatadj\) satisfies
\[
\CWSL(\yhatadj) \le \CWSL(\yhat),
\]
with equality holding when no feasible adjustment yields an improvement.

This guarantee applies only to the loss function used during evaluation and does
not imply improvement under alternative metrics or symmetric accuracy measures.

\subsection{Identity Condition}

If the loss-minimizing adjustment corresponds to \(\delta = 0\), or if no
candidate adjustment within \(\deltaset\) reduces loss, RAL defaults to the
identity transformation:
\[
\yhatadj = \yhat.
\]
This behavior prevents unnecessary forecast distortion and ensures that RAL
cannot introduce changes absent a clear decision-relevant benefit.

\subsection{Monotonicity Preservation}

When applied uniformly across forecast intervals, RAL preserves the relative
ordering of forecasts. That is, if \(\yhat_i \le \yhat_j\) prior to adjustment, the
same ordering holds for the adjusted forecasts \(\hat{y}_{\text{adj}, i} \le \hat{y}_{\text{adj}, j}\).

This property supports interpretability and prevents rank-order inversions that
could complicate downstream decision logic.

\subsection{Segment-Conditioned Policies and Fallback Guarantees}

RAL may be deployed using segment-conditioned adjustment policies, in which
distinct adjustment parameters are specified for predefined segments (e.g.,
interfaces, regions, or asset classes), typically subject to minimum data volume
or tail-risk criteria defined during offline policy construction. Segment
conditioning allows RAL to reflect localized operational risk profiles while
preserving global behavioral guarantees.

Segment-conditioned behavior is governed by three design constraints.

First, segment overrides are explicitly sparse. Only a limited subset of segments
are permitted to deviate from global adjustment parameters, typically subject to
minimum data volume or tail-risk criteria defined during offline policy
construction. This constraint prevents uncontrolled policy fragmentation and
limits the surface area of operational complexity.

Second, deterministic fallback behavior is guaranteed. For any segment lacking a
specific override, or for any previously unseen segment, RAL defaults
deterministically to the global adjustment policy. As a result, the adjusted
forecast remains well-defined for all inputs, independent of segmentation
coverage.

Third, segment-conditioned policies remain bounded and reviewable. All segment-
level parameters are drawn from the same feasible adjustment space as the global
policy, and no segment may introduce adjustments outside approved bounds. The
full policy—including global parameters and segment overrides—can therefore be
represented as a finite, auditable artifact.

Importantly, segment-conditioned RAL policies are \emph{artifacts}, not models.
They encode decisions produced by an offline tuning process but do not learn,
adapt, or evolve during application. This distinction preserves the determinism,
auditability, and governance properties of RAL while allowing controlled
heterogeneity where operationally justified.

\subsection{Conditional Properties}

Certain properties of RAL hold only under specific assumptions:
\begin{itemize}[leftmargin=*]
    \item \textbf{Global optimality} is not guaranteed unless the loss function is
    convex over the feasible set and the candidate adjustment set is evaluated
    exhaustively.
    \item \textbf{Separability across intervals} holds only when the loss function
    decomposes additively across time or units.
    \item \textbf{Stability across loss specifications} is not guaranteed; changes
    in cost calibration may alter the selected adjustment.
\end{itemize}

These conditions are not enforced by RAL and must be validated externally where
relevant.

\subsection{What RAL Does Not Guarantee}

RAL does not guarantee improvements in traditional forecast accuracy metrics,
nor does it correct structural deficiencies in the underlying forecasting model.
It does not adapt dynamically to unobserved regime changes, nor does it infer or
learn optimal adjustment bounds.

These limitations reflect the intended role of RAL as a controlled, auditable
adjustment layer rather than a substitute for model development or evaluation.
