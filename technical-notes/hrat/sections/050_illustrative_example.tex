% ----------------------------------------------------------
% ILLUSTRATIVE EXAMPLE
% ----------------------------------------------------------
\section{Illustrative Example}

To demonstrate how HR@\(\tau\) operates at the interval level, this section
presents a twelve-interval example with a tolerance of \(\tau = 3\). For each
interval, realized demand, forecasted demand, and the resulting absolute error are
evaluated against the tolerance band to classify each interval as a hit or miss.

Figure~\ref{fig:hrat_tolerance_plot} visualizes the realized series, forecast
values, and the \(\pm 3\) tolerance band around realized demand. Intervals in
which the forecast lies within the tolerance band are shown as hits, while those
outside the band are shown as misses.

\input{figures/hrat_tolerance_plot}

Table~\ref{tab:hrat_example} summarizes the numerical values corresponding to the
visual example, including the realized demand \(y_t\), forecast \(\hat{y}_t\),
absolute error \(|e_t|\), and an indicator of whether the interval satisfies the
tolerance condition.

\input{tables/hrat_example_table}

Out of the twelve evaluation intervals, ten satisfy the tolerance condition,
yielding
\[
\mathrm{HR@}_\tau = \frac{10}{12} \approx 0.833.
\]
This value indicates that the forecast is operationally acceptable in roughly
83\% of intervals given a tolerance threshold of three units. The example
illustrates how HR@\(\tau\) provides a transparent and interpretable assessment of
practical forecast performance.