% ----------------------------------------------------------
% LIMITATIONS
% ----------------------------------------------------------
\section{Limitations and Appropriate Use}

While HR@\(\tau\) provides a clear and interpretable measure of how often forecast
errors remain within an operationally acceptable range, the metric also has
limitations that should be considered when applying it in practice or comparing
forecasting systems. In particular, HR@\(\tau\) is intended as an evaluative
diagnostic of tolerance compliance and should not be optimized in isolation or
treated as a prescriptive execution target.

\subsection{Dependence on the Tolerance Parameter}
The informativeness of HR@\(\tau\) depends critically on the value of the
tolerance parameter. If \(\tau\) is set too high, the metric may overstate
forecast quality by classifying large deviations as acceptable. If set too low,
it may understate practical performance by treating small, inconsequential
differences as failures. Careful domain-specific calibration is therefore
required.

\subsection{No Information on Error Magnitude or Direction}
HR@\(\tau\) is a binary classification metric: it records only whether an
interval's error is within tolerance, not how far out of tolerance a miss is or
whether the deviation reflects overforecasting or underforecasting. The metric
should therefore be interpreted alongside magnitude-based and directional
diagnostics when error severity or asymmetry is operationally meaningful.

\subsection{Symmetric Treatment of Over- and Underestimation}
Because HR@\(\tau\) uses absolute error, it assumes that positive and negative
deviations have similar operational consequences. In environments where one form
of deviation is more costly---for example, where underbuilding leads to lost
demand or service degradation---HR@\(\tau\) should be supplemented with metrics
that explicitly account for directional risk.

\subsection{Scale Sensitivity Across Entities}
A fixed tolerance level may not be universally meaningful across entities with
different demand magnitudes. A tolerance of three units may be negligible for a
high-volume item but significant for a low-volume one. When comparing forecasts
across heterogeneous units, normalization or entity-specific tolerance levels may
be necessary to maintain interpretability.

\subsection{Loss of Distributional Detail}
By reducing each interval to a binary hit/miss indicator, HR@\(\tau\) discards
information about the underlying error distribution. Two forecasting systems may
achieve similar hit rates while exhibiting very different patterns of error
severity. HR@\(\tau\) is therefore most effective when used in conjunction with
metrics that capture variance, tail behavior, or systematic bias.

\subsection{Sampling Variability in Short Horizons}
As an empirical proportion computed over a finite number of intervals, HR@\(\tau\)
is subject to sampling variability. In short evaluation windows, small changes in
performance can lead to large variations in the estimated hit rate. Confidence
intervals or repeated-sample evaluation may be needed for reliable inference.
