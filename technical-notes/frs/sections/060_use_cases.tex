% ----------------------------------------------------------
% USE CASES
% ----------------------------------------------------------
\section{Use Cases Across Operational Domains}

The Forecast Readiness Score (\FRS) supports decision-making in environments
where service reliability, asymmetric cost considerations, and execution
feasibility are more important than purely statistical accuracy. Because \FRS{}
summarizes both shortfall avoidance and cost-efficient performance, it is
well-suited for operational forecasting workflows that require interpretable
readiness diagnostics. Importantly, \FRS{} is intended as an evaluative signal of
forecast readiness rather than a direct optimization target or execution rule.

\subsection{Short-Horizon Demand Planning}

In short-horizon production, retail replenishment, and perishable inventory
settings, operational disruption is driven primarily by \emph{shortfalls}.
Small overbuilds are often absorbable, but even a modest number of shortfall
intervals can degrade service or throughput. \FRS{} is valuable here because:

\begin{itemize}
    \item \NSL{} captures how frequently demand is fully covered,
    \item \CWSLscaled{} penalizes shortfalls more heavily than excess,
    \item the composite score reflects readiness to execute without service failure.
\end{itemize}

These properties make \FRS{} a natural KPI for short-horizon service reliability.

\subsection{Labor and Capacity Planning}

In staffing, labor scheduling, and capacity planning environments (e.g.,
restaurants, logistics centers, call centers), underforecasts degrade service
quality by generating long queues, slow response times, or labor shortages.
Overstaffing, while inefficient, is typically less harmful than understaffing.

\FRS{} provides a principled framework to:

\begin{itemize}
    \item incorporate asymmetric cost assumptions through the \((\cu, \co)\) parameters,
    \item quantify readiness to meet workload requirements,
    \item compare alternative forecast models under realistic operational criteria.
\end{itemize}

\subsection{Model Selection and Deployment Decisions}

\FRS{} can serve as a primary metric for choosing among competing forecasting
models. Because it encodes operational feasibility rather than purely numerical
accuracy, it helps prevent situations in which a statistically optimal model
performs poorly once deployed.

Applications include:

\begin{itemize}
    \item validation-set model comparison,
    \item production monitoring of forecast degradation,
    \item regression testing after model retraining.
\end{itemize}

\subsection{Segmented and Entity-Level Evaluation}

Operational environments often exhibit heterogeneous behavior across products,
locations, time blocks, or customer segments. \FRS{} can be computed:

\begin{itemize}
    \item globally (system-wide readiness),
    \item per entity (e.g., per SKU or store),
    \item per segment (e.g., store clusters, product families).
\end{itemize}

This enables targeted diagnosis of readiness gaps and supports interventions
tailored to specific segments.

\subsection{Operational Monitoring and Governance}

Because \FRS{} is bounded and interpretable, it can function as an operational
KPI within forecasting governance frameworks. Declines in \FRS{} over time signal:

\begin{itemize}
    \item increasing shortfall frequency,
    \item rising asymmetric error costs,
    \item erosion of deployment readiness.
\end{itemize}

This makes \FRS{} useful for ongoing monitoring, alerting, and readiness audits
in production forecasting systems.
