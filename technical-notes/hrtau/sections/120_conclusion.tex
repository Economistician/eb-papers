% ----------------------------------------------------------
% CONCLUSION
% ----------------------------------------------------------
\section{Conclusion}
\label{sec:conclusion}

HR@$\tau$ provides a direct and interpretable measure of forecast readiness by
explicitly encoding tolerance for error.
However, without principled selection of the tolerance parameter, readiness
assessment risks becoming arbitrary, unstable, or vulnerable to post hoc
adjustment.
This technical note reframes tolerance selection as a governed calibration
problem rather than a modeling choice.

By treating HR@$\tau$ as a response surface over tolerance, we introduce a
systematic framework for sensitivity analysis and deterministic calibration.
Quantile-based targets, knee detection, and utility-based rules provide
complementary mechanisms for selecting $\tau$ based on observed forecast
behavior, operational preferences, and governance constraints.
Entity-level extensions and safeguards further support responsible application
in heterogeneous settings.

Together with cost-ratio calibration for \CWSL{}, tolerance calibration defines
the evaluation envelope within which readiness is assessed.
These calibration primitives operate upstream of metric computation and ensure
that readiness metrics are applied within a consistent, auditable, and
operationally meaningful context.

The result is a readiness evaluation framework that separates measurement from
assumption, supports transparent governance, and aligns forecast assessment with
real operational tolerance.
As part of the broader Forecast Readiness Framework, calibrated HR@$\tau$ enables
forecast evaluation that is not only accurate, but decision-aligned and fit for
deployment.