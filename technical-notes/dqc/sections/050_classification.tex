% ==========================================================
% 050_classification.tex
% Demand Quantization Compatibility (DQC)
% ==========================================================

\section{DQC Classification}
\label{sec:dqc_classification}

This section formalizes the Demand Quantization Compatibility (\DQC{}) classification.
\DQC{} is a structural diagnostic that determines how realized demand should be
interpreted for the purposes of tolerance evaluation and readiness governance.
It does not estimate a latent data-generating process; rather, it assigns each
evaluated series to a discrete structural regime based on observable behavior.

\subsection{Classification Objective}

The objective of \DQC{} classification is to determine whether tolerance-based
measures and readiness interventions may be interpreted directly in raw units, or
whether they must instead respect an underlying demand grid or set of attainable
values. This determination is a prerequisite for valid governance and cannot be
inferred from forecast performance alone.

\subsection{Classification Taxonomy}

Each demand series is assigned to exactly one of the following mutually exclusive
classes:
\begin{itemize}[leftmargin=1.5em]
  \item \textbf{Continuous-Like}
  \item \textbf{Quantized}
  \item \textbf{Piecewise-Packed}
\end{itemize}

These classes are not ordered by quality or desirability. Each represents a
distinct structural regime with different implications for tolerance interpretation,
forecast snapping, and readiness policy.

\subsection{Continuous-Like}

If neither quantized nor piecewise-packed criteria are met, demand is classified as
\emph{continuous-like} at the evaluation resolution.

This classification does not assert true continuity of the underlying process.
Rather, it asserts that discretization effects are negligible for the purposes of
readiness interpretation and governance. In this regime, tolerance-based diagnostics
and readiness adjustments may be interpreted directly in raw units without
structural risk.

\subsection{Quantized}

A series is classified as \emph{quantized} when a single candidate grid unit
$\Delta^{\ast}$ explains the majority of nonzero observations and off-grid dispersion
is sufficiently small.

Formally, quantized behavior is declared when:
\begin{itemize}[leftmargin=1.5em]
  \item The best multiple-rate alignment $\rho(\Delta^{\ast})$ exceeds a prescribed
        threshold, and
  \item Normalized off-grid dispersion falls below its corresponding tolerance.
\end{itemize}

This pattern indicates that demand increments occur in fixed quanta, even if
occasional deviations arise from noise or aggregation. In this regime, tolerance
and readiness must be interpreted in grid units rather than raw units, and forecast
snapping is required for governance validity.

\subsection{Piecewise-Packed}

Some operational processes do not adhere to a single minimal unit but instead
exhibit concentration on a small set of discrete attainable values (e.g., pack
sizes, batch outputs, or menu bundles).

A series is classified as \emph{piecewise-packed} when:
\begin{itemize}[leftmargin=1.5em]
  \item Alignment to any single grid unit is weak or ambiguous, but
  \item A small number of distinct values explain a large share of total demand mass.
\end{itemize}

Piecewise-packed demand is structurally discrete but not reducible to a single
quantum. In this regime, snapping to a single grid is insufficient. Governance must
treat tolerance and readiness in terms of discrete attainable values rather than
continuous proximity.

\subsection{Determinism and Tie-Breaking}

\DQC{} classification is fully deterministic and rule-based. Given identical inputs
and thresholds, the same demand series will always receive the same classification.

When multiple candidate grid units satisfy alignment criteria, \DQC{} applies a
stable tie-breaking rule that favors:
\begin{enumerate}
  \item Smaller grid units (finer resolution), and
  \item Lower off-grid dispersion.
\end{enumerate}

This ensures reproducibility, auditability, and stability under repeated evaluation,
and prevents threshold jitter from inducing policy instability.

\subsection{Classification as Governance Input}

The output of \DQC{} classification consists of:
\begin{itemize}[leftmargin=1.5em]
  \item The assigned structural class,
  \item The detected grid unit (when applicable),
  \item Supporting evidence metrics.
\end{itemize}

These outputs are consumed directly by the governance layer to determine:
\begin{itemize}[leftmargin=1.5em]
  \item Whether forecast snapping is required,
  \item How tolerance $\tau$ should be interpreted,
  \item Which Forecast Primitive Compatibility (FPC) evaluation
        (raw or snapped) is authoritative.
\end{itemize}

\DQC{} classifications are not performance metrics and are not subject to
optimization. They exist solely to constrain downstream evaluation and policy,
ensuring that readiness decisions are grounded in an admissible structural
interpretation of demand.
