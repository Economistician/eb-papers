% ==========================================================
% 010_overview.tex
% Demand Quantization Compatibility (DQC)
% ==========================================================

\section{Overview}
\label{sec:dqc_overview}

Operational demand streams frequently violate the implicit continuity
assumptions embedded in common forecasting and evaluation pipelines. In many
settings, realized demand arrives in \emph{discrete}, \emph{quantized}, or
\emph{piecewise-packed} units driven by inventory constraints, packaging
conventions, batching, or executional indivisibilities (e.g., items sold in
fixed counts, cases, or bundles). When such structure is ignored, downstream
metrics may exhibit pathological behavior: tolerance bands become unstable,
hit-rate measures oscillate with minor threshold changes, and scale-based
readiness adjustments appear ineffective or misleading despite apparently
reasonable forecasts.

\medskip
\begin{framed}
\noindent\textbf{Core Framing Question.}\;
\emph{At the chosen evaluation resolution, does realized demand behave
continuously, or does it exhibit a quantized or packed structure that requires
grid-aware interpretation?}
\end{framed}
\medskip

\emph{Demand Quantization Compatibility (DQC)} is introduced as a diagnostic and
governance construct that explicitly addresses this question.

DQC is not a performance metric and is not an optimization objective. It does not
evaluate forecast accuracy, cost efficiency, or service quality. Instead, it
classifies the \emph{structural form} of the realized demand process using
observable properties of the demand series itself. Its sole purpose is to
determine whether downstream evaluation and control mechanisms—particularly
tolerance-based service metrics and readiness adjustments—must be interpreted in
raw units or in snapped, grid-aligned units.

Within the Forecast Readiness Framework (FRF), DQC plays a foundational role.
Where Forecast Primitive Compatibility (FPC) diagnoses whether a given forecast
\emph{primitive} can respond meaningfully to readiness intervention, DQC
diagnoses whether the \emph{measurement space} in which that intervention is
evaluated is structurally valid. A forecast primitive may be compatible in
principle yet appear ineffective if evaluated against a demand process whose
support is discretized relative to the tolerance or adjustment scale.

Accordingly, DQC provides an upstream governance signal that answers three
practical questions:
\begin{enumerate}[label=(\roman*)]
  \item Is demand sufficiently continuous-like to justify raw-unit evaluation?
  \item If not, what grid (granularity) best characterizes realized demand?
  \item Should tolerances and readiness adjustments be interpreted in grid units
        rather than raw units?
\end{enumerate}

The remainder of this technical note formalizes these ideas. We define observable
quantization signals, propose a deterministic compatibility taxonomy, and
describe how DQC integrates with FRF diagnostics and governance without
introducing new optimization targets or subjective tuning.
