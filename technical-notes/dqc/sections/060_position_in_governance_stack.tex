% ==========================================================
% 060_position_in_governance_stack.tex
% Demand Quantization Compatibility (DQC)
% ==========================================================

\section{Position in the Governance Stack}
\label{sec:dqc_position_governance_stack}

Demand Quantization Compatibility (\DQC{}), Forecast Primitive Compatibility
(\FPC{}), and Governance form a deliberately ordered decision stack. Each
construct addresses a distinct question, and none is a substitute for another.

This section clarifies their respective roles and the logic of their
composition within the Electric Barometer Governance framework.

\subsection{Distinct Questions, Distinct Layers}

The three constructs answer fundamentally different questions:
\begin{itemize}
  \item \textbf{\DQC{}:} \emph{What is the structural support of realized demand?}
  \item \textbf{\FPC{}:} \emph{Is a given forecast primitive structurally compatible
        with that support under admissible readiness adjustment?}
  \item \textbf{Governance:} \emph{Given both, what actions are permitted?}
\end{itemize}

\DQC{} is concerned exclusively with demand realization. \FPC{} is concerned with
the behavior of a forecasting primitive under controlled perturbation.
Governance is the policy layer that binds the two into enforceable decisions.

\subsection{Why \DQC{} Precedes \FPC{}}

\FPC{} diagnostics rely on tolerance-based signals (e.g.\ $\HRtau$, $\Delta \NSL$,
$\Delta \CWSL$) whose interpretation depends critically on the unit system in
which they are evaluated.

Without first resolving whether demand is continuous-like or quantized, \FPC{}
signals are ambiguous:
\begin{itemize}
  \item Apparent insensitivity may reflect discrete demand support rather than
        primitive failure.
  \item Apparent gains may be artifacts of tolerance choices smaller than the
        realizable demand increment.
\end{itemize}

\DQC{} therefore precedes \FPC{} and determines the admissible space in which
\FPC{} is evaluated. This ordering is non-negotiable.

\subsection{Raw vs.\ Snapped \FPC{} as Separate Diagnostics}

When demand is quantized or piecewise-packed, two distinct \FPC{} views may
exist:
\begin{itemize}
  \item \emph{Raw \FPC{}}: evaluated in continuous units,
  \item \emph{Snapped \FPC{}}: evaluated after snapping forecasts to the detected
        demand support.
\end{itemize}

These diagnostics are not interchangeable. Governance treats them
asymmetrically:
\begin{itemize}
  \item Raw \FPC{} is informational only when snapping is required.
  \item Snapped \FPC{} is authoritative for readiness allowability.
\end{itemize}

This separation prevents Governance from being influenced by diagnostics that
violate structural constraints.

\subsection{Governance as the Binding Contract}

Governance consumes \DQC{} and \FPC{} outputs and emits a single decision
artifact. This artifact encodes:
\begin{itemize}
  \item Whether snapping is required,
  \item How tolerances $\tau$ must be interpreted,
  \item Whether readiness adjustment is allowed, constrained, or forbidden.
\end{itemize}

Neither \DQC{} nor \FPC{} makes policy decisions independently. Governance is the
only layer permitted to do so.

\subsection{Avoiding Metric Entanglement}

A core design principle of Electric Barometer is the avoidance of metric
entanglement. \DQC{} does not depend on cost, service targets, or optimization
objectives. \FPC{} does not depend on assumptions about demand discretization.
Governance is the only layer in which these concerns are reconciled.

This separation ensures that:
\begin{itemize}
  \item Structural validity is assessed before performance,
  \item Policy is enforced before optimization,
  \item Metrics remain interpretable within their intended domain.
\end{itemize}

\subsection{Conceptual Summary}

\DQC{} defines the admissible representation of demand.
\FPC{} determines whether a readiness-adjusted forecast primitive behaves
coherently within that representation.
Governance binds these diagnoses into enforceable operational policy.

Together, they transform readiness evaluation from an accuracy-centric exercise
into a structurally grounded decision system.
