% ==========================================================
% 040_observable_signals.tex
% Demand Quantization Compatibility (DQC)
% ==========================================================

\section{Observable Signals}
\label{sec:dqc_observable_signals}

Demand Quantization Compatibility (DQC) is diagnosed entirely from observable
properties of the realized demand series. No forecasting outputs, tuning
parameters, or optimization artifacts are required. This section defines the
signals used to detect quantized and packed support and explains their
interpretation.

\subsection{Candidate-Grid Alignment Rate}

For a given candidate grid unit $\Delta$, define the alignment indicator
\[
  a_{it}(\Delta) = \Ind\!\left( \frac{\yit}{\Delta} \in \mathbb{Z}_{\ge 0} \right),
\]
where $\Ind(\cdot)$ is the indicator function.

The \emph{multiple-rate alignment} for unit $\Delta$ is then
\[
  \rho(\Delta) = \frac{1}{|\mathcal{T}^{+}|}
  \sum_{t \in \mathcal{T}^{+}} a_{it}(\Delta),
\]
where $\mathcal{T}^{+}$ denotes the set of time indices with nonzero demand.

High values of $\rho(\Delta)$ indicate that demand lies predominantly on integer
multiples of $\Delta$. DQC evaluates $\rho(\Delta)$ across a fixed set of
candidate grid units and records the best-aligned unit and its associated rate.

\subsection{Packed-Unit Detection}

Quantized demand may arise from multiple operational pack sizes rather than a
single minimal unit. To detect this structure, DQC evaluates whether realized
demand concentrates on a small set of discrete values that collectively explain
most of the mass.

Operationally, this is captured by:
\begin{itemize}
  \item Counting distinct nonzero demand values.
  \item Measuring the cumulative mass explained by the most frequent values.
\end{itemize}

If a small number of values account for a high fraction of demand mass and
alignment with a single grid unit is weak, DQC classifies the series as
\emph{piecewise-packed} rather than purely quantized.

\subsection{Off-Grid Dispersion}

Perfect alignment is rarely observed in practice due to measurement noise,
aggregation artifacts, or operational exceptions. DQC therefore measures the
dispersion of off-grid observations.

Let $\tilde{\yit}$ denote the nearest grid-aligned value under the best-fitting
$\Delta$, selected under a fixed rounding rule defined by governance. The
off-grid deviation is
\[
  d_{it} = \yit - \tilde{\yit}.
\]

DQC summarizes dispersion using a robust statistic, such as the median absolute
deviation (MAD), normalized by the grid unit:
\[
  \text{OffGridRatio} = \frac{\mathrm{MAD}(d_{it})}{\Delta}.
\]

Low values indicate tight clustering around grid points, reinforcing a
quantization diagnosis even when exact alignment is imperfect.

\subsection{Minimum Support Guardrails}

To avoid pathological classifications driven by sparse data, DQC enforces a
minimum number of nonzero observations before quantization signals are
considered. If this requirement is not met, the series is treated as
\emph{structurally indeterminate} and governed as continuous-like by default
as a conservative placeholder, rather than as a positive declaration of
continuous behavior.

This guardrail ensures that DQC diagnoses meaningful support structure rather
than artifacts of short or degenerate samples.

\subsection{Signal Summary}

Collectively, DQC relies on the following observable \emph{diagnostic signals},
not performance metrics:
\begin{itemize}
  \item Best-aligned grid unit $\Delta^{\ast}$.
  \item Multiple-rate alignment $\rho(\Delta^{\ast})$.
  \item Evidence of multi-pack concentration.
  \item Off-grid dispersion relative to $\Delta^{\ast}$.
  \item Count of nonzero observations.
\end{itemize}

These signals are combined deterministically to produce a compatibility class,
as described in the next section, and are not intended to be optimized,
ranked, or interpreted as objective functions.
