% ==========================================================
% 000_frontmatter.tex
% FRONTMATTER (DQC Technical Note)
% ==========================================================

\begin{abstract}
Operational evaluation frameworks commonly assume that realized demand behaves as
continuous-like at the chosen resolution, such that tolerance-based diagnostics (e.g., hit rates
at tolerance $\tau$) and scale-based readiness adjustments (e.g., multiplicative uplift policies)
admit stable interpretation in raw units. In practice, many commodity streams are inherently
\emph{quantized}: demand arrives in discrete units, pack sizes, or piecewise-constant regimes,
often producing heavy mass at a small set of attainable values. In such regimes, continuous-valued
evaluation and control can become ill-posed. Small forecast perturbations may be observationally
indistinguishable, tolerance thresholds can be miscalibrated, and readiness policies can appear
unstable or misleading despite correct structural behavior in the appropriate units.

This technical note introduces \emph{Demand Quantization Compatibility (DQC)} as a derived,
auditable diagnostic that assesses whether a realized demand process is compatible with
continuous-like interpretation at the evaluation resolution or instead requires a grid-aware
unit system and representation (e.g., governance-controlled snapping). DQC is not a performance
metric and is not an optimization objective. Rather, it is a governance construct that classifies
demand structure, detects dominant grid granularity when present, and produces a stable basis for
interpreting tolerances, service diagnostics, and readiness policies downstream.
\end{abstract}
