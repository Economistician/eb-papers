% ==========================================================
% 030_quantization_model.tex
% Demand Quantization Compatibility (DQC)
% ==========================================================

\section{Quantization Model}
\label{sec:dqc_quantization_model}

Demand Quantization Compatibility (DQC) is grounded in a simple but critical
observation: many operational demand streams do not occupy a continuous support
at the evaluation resolution. Instead, realized demand often concentrates on a
small number of discrete values induced by physical, commercial, or procedural
constraints. This section formalizes the quantization model assumed by DQC and
clarifies how it differs from generic discreteness or count-data assumptions.

\subsection{Underlying Assumption}

Let $\yit$ denote realized demand for entity $i$ at time $t$, evaluated at a
fixed resolution (e.g., hourly, daily). DQC assumes that $\yit$ is drawn from
one of the following structural regimes:

\begin{enumerate}
  \item \textbf{Continuous-like support}  
  Demand occupies a dense subset of $\mathbb{R}_{\ge 0}$ at the evaluation
  resolution. Small perturbations in forecasts can meaningfully change coverage
  and service metrics.

  \item \textbf{Quantized support}  
  Demand lies predominantly on integer multiples of a single grid unit
  $\Delta^{\ast} > 0$, selected from a finite candidate set, i.e.
  \[
    \yit \in \{0, \Delta^{\ast}, 2\Delta^{\ast}, 3\Delta^{\ast}, \dots\}
  \]
  with high empirical mass.

  \item \textbf{Piecewise-packed support}  
  Demand concentrates on a small set of discrete values that may correspond to
  multiple pack sizes or composite units (e.g., bundles, case packs), not all of
  which are integer multiples of a single minimal unit.
\end{enumerate}

DQC does not assume that demand is integer-valued, Poisson, or count-distributed.
Instead, it models \emph{support geometry}: where realized values actually land
relative to the evaluation resolution.

\subsection{Grid-Induced Semantics}

Quantization matters because many evaluation constructs implicitly assume local
continuity. For example, tolerance-based metrics such as $\HRtau$ interpret
$\tau$ as a small perturbation radius around forecasts. When demand is quantized,
this interpretation breaks down:

\begin{itemize}
  \item Small changes in $\tau$ may have no effect until a grid boundary is
  crossed.
  \item Readiness adjustments may inflate forecasts without increasing hit-rate
  or coverage.
  \item Apparent metric smoothness may be an artifact of aggregation rather than
  genuine responsiveness.
\end{itemize}

DQC therefore treats quantization as a \emph{semantic constraint} on evaluation,
not merely a descriptive property of the data.

\subsection{Candidate Grid Units}

Rather than estimating an arbitrary real-valued grid, DQC evaluates demand
support against a finite, interpretable set of candidate grid units:
\[
  \Delta \in \{\Delta_1, \Delta_2, \dots, \Delta_K\}
\]
where candidates are chosen to reflect plausible operational units (e.g.,
1, 2, 4, 6, 8, 12, 16). These candidates are governance-defined parameters,
not learned quantities.

This design choice is deliberate:

\begin{itemize}
  \item It avoids overfitting granularity to noise.
  \item It produces audit-friendly results that align with known business
  constraints.
  \item It ensures stability across evaluation windows.
\end{itemize}

For each candidate $\Delta$, DQC evaluates how much of the realized demand mass
aligns with integer multiples of that unit.

\subsection{Quantization vs.\ Intermittency}

It is important to distinguish quantization from intermittency or sparsity.
Zero-heavy demand may still be continuous-like once nonzero values are observed,
and highly intermittent series may nevertheless be quantized when demand occurs.

DQC explicitly conditions on a minimum number of nonzero observations to avoid
classifying trivially sparse series as quantized by construction. This reinforces
that DQC diagnoses \emph{support structure}, not frequency of demand.

\subsection{Operational Interpretation}

Under this model, quantization is not a flaw in the data—it is a property of the
system being measured. DQC makes this property explicit so that downstream
evaluation and governance logic can operate in a space that respects the true
geometry of demand.

\emph{The observable signals introduced next are intentionally chosen for
determinism, auditability, and governance stability rather than statistical
optimality. They are designed to support reproducible classification and
policy enforcement under repeated evaluation.}

In the next section, we describe the observable signals used to detect these
structures in practice.
