% ==========================================================
% 090_limitations.tex
% Demand Quantization Compatibility (DQC)
% ==========================================================

\section{Limitations}
\label{sec:dqc_limitations}

Demand Quantization Compatibility (DQC) is intentionally designed as a
\emph{structural diagnostic}, not as a complete characterization of demand
behavior. As such, it carries several limitations that should be understood
when interpreting its outputs or integrating it into broader decision systems.

\subsection{Heuristic Nature of Grid Detection}

DQC infers demand granularity from observed realizations using empirical
signals such as multiple-rate concentration and off-grid dispersion. While
these diagnostics are robust in common operational settings, they are not
guaranteed to recover the true generative process.

In particular:
\begin{itemize}
  \item Finite samples may obscure the true grid,
  \item Mixed demand regimes may produce ambiguous signals,
  \item Rare but valid off-grid observations may distort dispersion metrics.
\end{itemize}

DQC therefore prioritizes conservative classification over exact grid recovery.

\subsection{Sensitivity to Sample Size and Horizon}

Quantization diagnostics require sufficient nonzero observations to be
meaningful. Very short evaluation windows, sparse demand streams, or extreme
intermittency can reduce the reliability of granularity inference.

When sample support is weak, DQC may:
\begin{itemize}
  \item Default to continuous-like classifications,
  \item Flag marginal states that defer policy decisions,
  \item Require downstream governance safeguards.
\end{itemize}

These outcomes are intentional and reflect epistemic uncertainty rather than
model failure.

\subsection{Stationarity Assumptions}

DQC implicitly assumes that demand quantization structure is stable over the
evaluation window. Temporal regime shifts—such as changes in packaging, order
policy, or measurement units—can violate this assumption.

DQC does not currently:
\begin{itemize}
  \item Detect structural breaks in granularity,
  \item Model time-varying grids,
  \item Attribute changes to operational causes.
\end{itemize}

Such extensions are explicitly out of scope for the present formulation.

\subsection{No Performance or Cost Awareness}

DQC deliberately ignores cost asymmetry, service targets, and forecast
performance. A quantized demand process may still be well served by certain
forecasting approaches or readiness strategies under specific objectives.

DQC answers only:
\begin{quote}
\emph{Is continuous interpretation structurally admissible here?}
\end{quote}

It does not answer whether a particular decision is economically optimal or
desirable.

\subsection{Not a Replacement for Domain Knowledge}

While DQC provides an auditable, data-driven signal, it cannot substitute for
operational context. Known packaging rules, contractual constraints, or system
design decisions may define granularity more reliably than empirical inference.

In such cases, DQC should be treated as:
\begin{itemize}
  \item A consistency check,
  \item A guardrail against misuse of continuous diagnostics,
  \item A supplement to documented process knowledge.
\end{itemize}

\subsection{Role Within the Electric Barometer Stack}

These limitations are mitigated by design through composition:
\begin{itemize}
  \item Governance prevents inappropriate use of DQC outputs,
  \item FPC contextualizes quantization effects within forecast behavior,
  \item Threshold tuning allows conservative or permissive stances as needed.
\end{itemize}

DQC is most effective when used as intended: a narrow, structural diagnostic
within a layered readiness evaluation framework, not as a standalone decision
engine.
