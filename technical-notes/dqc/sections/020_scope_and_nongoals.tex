% ==========================================================
% 020_scope_and_nongoals.tex
% Demand Quantization Compatibility (DQC)
% ==========================================================

\section{Scope and Non-Goals}
\label{sec:dqc_scope_nongoals}

Demand Quantization Compatibility (DQC) is intentionally narrow in scope. It is
designed to answer a specific structural question about realized demand and to
support governance decisions downstream. To avoid misuse and over-extension, we
explicitly state what DQC \emph{does} and \emph{does not} attempt to do.

\subsection{Scope}

DQC operates strictly on a realized demand series at a fixed evaluation
resolution. Within this setting, its responsibilities are limited to the
following:

\begin{itemize}
  \item \textbf{Structural classification of demand support.}  
  DQC determines whether realized demand behaves as continuous-like, strongly
  quantized, or piecewise-packed at the resolution under consideration.

  \item \textbf{Granularity detection.}  
  When demand is classified as quantized or packed, DQC detects a
  representative grid unit (granularity) that characterizes the dominant
  support of the demand distribution.

  \item \textbf{Governance signaling.}  
  DQC produces an interpretable diagnostic artifact that downstream systems can
  use to:
  \begin{itemize}
    \item require or waive snap-to-grid behavior,
    \item interpret tolerance parameters in raw units or grid units,
    \item ensure consistent evaluation semantics across entities and time.
  \end{itemize}

  \item \textbf{Auditability and stability.}  
  DQC outputs are deterministic, threshold-driven, and auditable. Thresholds may
  be tuned explicitly, but the diagnostic logic itself remains stable.
\end{itemize}

DQC is designed to be evaluated independently per entity, per hierarchy level,
and per evaluation window, enabling heterogeneous demand structures to coexist
within a single operational system.

\subsection{Non-Goals}

Equally important are the capabilities DQC \emph{explicitly excludes}. DQC does
\textbf{not} attempt to:

\begin{itemize}
  \item \textbf{Measure forecast quality.}  
  DQC does not evaluate forecast accuracy, bias, cost, or service outcomes. It
  operates solely on realized demand and is agnostic to forecast values.

  \item \textbf{Optimize thresholds or grids.}  
  DQC does not search for an optimal granularity or tune itself to improve
  downstream metrics. Any thresholds used are governance parameters, not learned
  quantities.

  \item \textbf{Replace modeling or transformation choices.}  
  DQC does not recommend specific forecasting models, transformations, or data
  preprocessing steps. It merely constrains how evaluation results should be
  interpreted.

  \item \textbf{Guarantee downstream performance.}  
  A continuous-like DQC classification does not imply that readiness adjustment
  will be effective, nor does a quantized classification imply that snapping will
  improve outcomes. These judgments belong to FPC and subsequent governance
  layers.

  \item \textbf{Act as a universal statistical test.}  
  DQC is not a general-purpose test for discreteness, intermittency, or
  zero-inflation. It is a pragmatic, operational diagnostic tailored to readiness
  evaluation semantics.
\end{itemize}

By maintaining this narrow scope, DQC avoids conflating structural diagnosis with
performance assessment. Its role is to ensure that downstream metrics and
policies are applied in a space that is consistent with the observed demand
process, nothing more and nothing less.
