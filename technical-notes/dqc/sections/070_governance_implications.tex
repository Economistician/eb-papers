% ==========================================================
% 070_governance_implications.tex
% Demand Quantization Compatibility (DQC)
% ==========================================================

\section{Governance and Policy Implications}
\label{sec:dqc_governance_implications}

Demand Quantization Compatibility (\DQC{}) is not a descriptive statistic or a data
quality check; it is a \emph{governance primitive}. Its purpose is to determine which
downstream evaluation, control, and adjustment actions are structurally admissible.

This section outlines the concrete governance implications of \DQC{} classification
and how it constrains permissible system behavior.

\subsection{Authoritative Control of Representation}

\DQC{} determines the authoritative representation in which demand, forecasts, and
tolerances must be expressed.

Once \DQC{} classifies a demand series as quantized or piecewise-packed:
\begin{itemize}[leftmargin=1.5em]
  \item Snapping becomes mandatory,
  \item Raw-unit interpretation is no longer admissible,
  \item Downstream diagnostics and policy logic must operate in a demand-congruent space.
\end{itemize}

This prevents representational drift in which different system components operate
under incompatible assumptions about the attainable support of demand.

\subsection{Snapping and Unit Interpretation}

Snapping is a governance-controlled transformation that enforces structural coherence
between forecasts, diagnostics, and realized demand. It is not an optional preprocessing
step.

\paragraph{What Snapping Is.}
Snapping is the deterministic projection of forecasts (and, where relevant, tolerance
bands) onto a demand-congruent support. Given a detected grid unit $\Delta^{\ast}$,
a raw forecast $\yhatit$ is transformed into a snapped forecast $\tilde{y}_{it}$ such that:
\[
\tilde{y}_{it} \in \{ k \Delta^{\ast} : k \in \mathbb{Z}_{\ge 0} \},
\]
with $\tilde{y}_{it}$ chosen by a fixed rounding rule (e.g., nearest admissible value).
Snapping alters \emph{representation}, not information: it enforces structural alignment.

\paragraph{When Snapping Is Required.}
Snapping requirements are determined exclusively by \DQC{} classification:
\begin{itemize}[leftmargin=1.5em]
  \item \textbf{Continuous-like}: snapping is not required.
  \item \textbf{Quantized}: snapping is required to the detected grid unit $\Delta^{\ast}$.
  \item \textbf{Piecewise-packed}: snapping is required to the admissible value set implied
        by the detected packing structure.
\end{itemize}
No downstream component may override this requirement. Applying readiness or tolerance
logic in raw units when snapping is required constitutes a governance violation.

\paragraph{Tolerance Semantics.}
The authoritative interpretation and enforcement of tolerance semantics under DQC
classification are defined in Section~\ref{sec:dqc_governance_implications:ToleranceCompliance}.

\paragraph{Diagnostics Under Snapping.}
When snapping is required, diagnostics must be recomputed in snapped space:
\begin{itemize}[leftmargin=1.5em]
  \item Raw-unit \FPC{} signals may be logged for exploratory purposes only.
  \item Snapped-unit \FPC{} signals are authoritative for governance decisions.
\end{itemize}
Mixing raw and snapped diagnostics within a single governance decision is explicitly prohibited.

\subsection{Interaction with Forecast Primitive Compatibility (FPC)}

\DQC{} governs \emph{where} Forecast Primitive Compatibility (\FPC{}) is evaluated:
\begin{itemize}[leftmargin=1.5em]
  \item For continuous-like demand, \FPC{} is evaluated in raw units.
  \item For quantized or piecewise-packed demand, \FPC{} must be evaluated in snapped space.
\end{itemize}

In the latter case, raw-space \FPC{} signals are explicitly subordinated. They may be inspected
for exploratory analysis but cannot govern readiness policy. This separation ensures that readiness
allowability is assessed against the true structural degrees of freedom of the demand process.

\subsection{Readiness Adjustment Policy}

Readiness adjustments such as scale-based policies (\eg \RAL{}) are sensitive to the unit system in
which they are applied. \DQC{} constrains readiness policy as follows:
\begin{itemize}[leftmargin=1.5em]
  \item Adjustments applied in raw units are forbidden when demand is quantized or piecewise-packed.
  \item Adjustments applied in snapped units are permitted only if snapped-space \FPC{} indicates
        compatibility.
\end{itemize}

Thus, \DQC{} is a necessary (but not sufficient) condition for readiness intervention. It filters out
structurally invalid adjustment pathways before performance metrics or cost tradeoffs are considered.

\subsection{Tolerance Semantics and Compliance}
\label{sec:dqc_governance_implications:ToleranceCompliance}

Tolerance-based service metrics are often treated as tunable parameters. \DQC{} converts tolerance
interpretation into a compliance obligation.

When snapping is required:
\begin{itemize}[leftmargin=1.5em]
  \item $\tau$ must correspond to integer multiples of the detected grid unit (or admissible values),
  \item Tolerances smaller than the demand increment are invalid by construction,
  \item Reported hit rates must reflect realizable demand outcomes.
\end{itemize}

This prevents governance loopholes in which apparent service quality is achieved through structurally
impossible tolerance settings.

\subsection{Auditability and Policy Traceability}

Because \DQC{} decisions are deterministic and threshold-driven, they provide a stable audit trail for
governance outcomes. Each governance decision can be traced to:
\begin{itemize}[leftmargin=1.5em]
  \item Observed demand structure,
  \item Explicit quantization thresholds,
  \item A declared \DQC{} class.
\end{itemize}

This enables post hoc review of readiness decisions without reliance on opaque model internals or
subjective judgment.

\subsection{Failure Modes Without DQC}

Absent \DQC{}, systems commonly exhibit the following pathologies:
\begin{itemize}[leftmargin=1.5em]
  \item Penalizing forecasts for unavoidable discretization effects,
  \item Declaring readiness gains that are artifacts of unit mismatch,
  \item Applying readiness adjustments that cannot be operationally realized.
\end{itemize}

\DQC{} eliminates these failure modes by enforcing structural coherence as a first-class governance
constraint.

\subsection{Summary}

\DQC{} converts an often-implicit assumption---that demand behaves continuously---into an explicit,
testable, and enforceable contract. In doing so, it anchors the governance layer of Electric Barometer,
ensuring that all downstream decisions operate on structurally valid representations before any notion
of accuracy, cost, or optimization is considered.
