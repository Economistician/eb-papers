% ==========================================================
% 080_interpretive_boundaries_and_misuse.tex
% Demand Quantization Compatibility (DQC)
% ==========================================================

\section{Interpretive Boundaries and Common Misuse}
\label{sec:interpretive_boundaries_and_misuse}

Having established the role of \DQC{} in governance and policy gating, we now
clarify the interpretive boundaries and common forms of misuse that \DQC{} is
designed to prevent. Because its output is categorical and governance-oriented,
\DQC{} is susceptible to misinterpretation if treated as a performance signal,
data quality metric, or modeling recommendation. This section delineates the
intended scope of \DQC{} and documents failure modes that arise when those
boundaries are violated.

\subsection{DQC Is Not a Performance Metric}

A \DQC{} classification does not measure forecast accuracy, service quality, or
operational efficiency. It reflects only the structural support of realized
demand at the evaluation resolution.

A demand series classified as \emph{quantized} or \emph{piecewise-packed} is not
inherently problematic, difficult, or undesirable. Likewise, a
\emph{continuous-like} classification does not imply favorable readiness
outcomes. \DQC{} is orthogonal to performance and must not be interpreted as a
judgment on forecasting quality or operational success.

\subsection{DQC Is Not a Data Quality Diagnostic}

\DQC{} does not assess data cleanliness, measurement error, or reporting
fidelity. Discrete or packed demand structures often arise from legitimate
operational constraints such as batching, menu design, capacity limits, or
ordering rules.

Treating quantization signals as evidence of data defects or preprocessing
errors constitutes misuse. \DQC{} assumes that observed demand reflects the
operational reality faced by decision systems and evaluates only whether that
reality admits continuous interpretation.

\subsection{DQC Is Not a Modeling Prescription}

\DQC{} does not recommend specific forecasting models or primitives. While a
quantized or packed classification may motivate the exploration of discrete,
hurdle, or state-based approaches, model selection and specification remain
outside the scope of this diagnostic.

The role of \DQC{} is to determine whether downstream evaluation and readiness
mechanisms are structurally admissible under a given representation. It signals
when a representational change may be necessary, but it does not dictate how
that change should be implemented.

\subsection{Snapping Is Not an Accuracy Improvement}

Because snapping enforces representational coherence between forecasts and
realized demand, it is sometimes misinterpreted as an accuracy-improving
transformation. This interpretation is incorrect.

Snapping does not improve predictive accuracy, reduce error, or enhance model
skill. It may increase apparent hit rates or reduce reported error solely by
eliminating structurally impossible values. These effects reflect compliance
with demand support, not improved forecasting performance.

\subsection{Conditional and Context-Dependent Interpretation}

\DQC{} classifications are conditional on evaluation resolution, aggregation
level, and operational context. Changes in cadence, batching policy, menu
structure, or demand aggregation may alter observable quantization behavior and
warrant re-evaluation.

As a result, \DQC{} classifications should be treated as context-specific
governance signals rather than permanent attributes of an entity. Stability over
time increases confidence in the diagnosis, but absence of stability does not
invalidate the construct.

\subsection{Separation from Policy Authority}

\DQC{} does not itself authorize or prohibit readiness actions. Its role is
limited to determining admissible representations and unit systems.

Policy decisions are made only by the governance layer, which composes \DQC{}
with Forecast Primitive Compatibility (\FPC{}) and applies explicit decision
rules. Using \DQC{} classifications directly as action triggers—without
governance mediation—breaks the separation of concerns that underpins
auditability, accountability, and stability in the Electric Barometer framework.

\subsection{Intended Use}

Proper use of \DQC{} consists of:
\begin{itemize}
  \item Determining whether demand may be interpreted in raw units or must respect
        a discrete support.
  \item Enforcing snapping and tolerance semantics as governance obligations.
  \item Constraining downstream diagnostics and readiness evaluation to
        structurally admissible spaces.
  \item Preventing misinterpretation of discretization effects as forecast error.
\end{itemize}

By enforcing these interpretive boundaries, \DQC{} protects readiness evaluation
from representational misuse and ensures that downstream governance decisions
operate on structurally valid foundations. Its value lies not in evaluating
performance, but in preserving the admissibility of evaluation itself.
