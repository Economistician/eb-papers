% ==========================================================
% 100_conclusion.tex
% Demand Quantization Compatibility (DQC)
% ==========================================================

\section{Conclusion}
\label{sec:dqc_conclusion}

Demand Quantization Compatibility (DQC) formalizes a structural question that is
often left implicit in operational analytics: \emph{whether demand can be
meaningfully interpreted as continuous at the resolution used for evaluation
and control}. By elevating this question to a first-class diagnostic, DQC
provides a principled safeguard against the silent misuse of tolerance-based
metrics, readiness adjustments, and governance signals.

DQC does not seek to optimize forecasts, improve accuracy, or recover latent
generative mechanisms. Instead, it delivers a deterministic, auditable
classification of demand structure based on observable behavior. This
classification enables downstream systems to:
\begin{itemize}
  \item Enforce unit-consistent interpretation of tolerances,
  \item Require snapping to discrete grids when warranted,
  \item Prevent structurally invalid readiness adjustments,
  \item Align evaluation semantics with operational reality.
\end{itemize}

Within the Electric Barometer ecosystem, DQC occupies a complementary role to
Forecast Primitive Compatibility (FPC). Where FPC diagnoses whether a forecast
primitive can respond meaningfully to adjustment, DQC diagnoses whether the
units of interpretation themselves are admissible. Governance composes these
diagnostics into a single, authoritative policy decision that is explicit,
stable, and reproducible.

More broadly, DQC illustrates a central design principle of the Forecast
Readiness Framework: \emph{structural validity precedes optimization}. By
making hidden assumptions about continuity and granularity explicit, DQC
reduces the risk of false confidence, brittle policies, and misleading
performance narratives.

As operational forecasting systems increasingly integrate automated control
levers, asymmetric cost models, and fine-grained diagnostics, such structural
guardrails become essential. DQC provides one such governance safeguard—narrow
in scope, deliberate in intent, and designed to be composed, not overextended.

In this sense, DQC is not merely a diagnostic, but a reminder: before asking
whether a decision is good, we must first ensure that it is \emph{well-defined}.
