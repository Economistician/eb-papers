% ----------------------------------------------------------
% OVERVIEW
% ----------------------------------------------------------
\section{Overview}

Operational environments that rely on short-horizon forecasting often require
demand to be met within discrete intervals in order to maintain throughput,
service quality, and system stability. In such settings, even small shortfalls
can introduce delays, trigger bottlenecks, or disrupt workflow. As a result,
evaluating forecasts solely on average accuracy may be insufficient; it is
also necessary to assess how reliably forecasts provide adequate coverage
within each interval.

The No-Shortfall Level (\NSL) provides a concise measure of this reliability.
Defined as the proportion of intervals in which forecasted demand meets or
exceeds realized demand, \NSL captures a binary notion of success that aligns
directly with the operational requirement to avoid shortfall events. Its
structure makes the metric easy to interpret, invariant to demand scaling,
and broadly applicable across domains characterized by interval-based
decision-making. Importantly, \NSL is an evaluative reliability diagnostic: it
assesses the frequency of shortfall avoidance under observed forecasts, but
does not prescribe execution policies or operational control decisions.

While magnitude-based accuracy metrics summarize average deviation, they do
not describe how frequently operational requirements are satisfied within
individual intervals. \NSL complements these metrics by focusing on the
occurrence, rather than the size, of shortfalls. This technical note presents
a formal definition of \NSL, examines its key properties, and illustrates its
behavior under different forecasting patterns.
