% ----------------------------------------------------------
% OPERATIONAL MOTIVATION
% ----------------------------------------------------------
\section{Operational Motivation}

Many operational workflows exhibit strong directional asymmetry in the
consequences of forecast error. A modest underforecast during a peak period may
lead to lost throughput, service degradation, queueing instability, or cascading
delays. By contrast, a modest overforecast is often absorbed through buffers,
inventory, slack labor capacity, or unused production bandwidth. This directional
imbalance in the operational impact of forecasting error reflects a longstanding
theme in decision-oriented forecasting research, where asymmetric loss functions
are used to represent unequal consequences of over- and underprediction
\citep{zellner1986}. As a result, operators are typically more sensitive to the
\emph{cost consequences} of error than to its numerical magnitude alone.

Cost-Weighted Service Loss (\CWSL) encodes these consequences directly. By
assigning explicit penalty weights to shortfall and overbuild magnitudes, the
metric enables evaluation in environments where:

\begin{itemize}
    \item underbuilding leads to operational or financial loss,
    \item overbuilding introduces secondary inefficiencies but is less harmful,
    \item the relative severity of these effects is understood or can be estimated,
    \item comparisons must remain meaningful across heterogeneous items, locations,
          or dayparts.
\end{itemize}

Traditional symmetric error metrics cannot express these realities. They treat
positive and negative deviations equivalently and therefore obscure whether a
forecast poses operational risk. In contrast, \CWSL{} provides a cost-aligned
view of error by quantifying the \emph{effective fraction of demand lost} under
the chosen penalty structure. This makes the metric particularly well suited to
readiness-oriented forecasting contexts—those in which the consequences of being
short or long differ materially and must be reflected in model evaluation.