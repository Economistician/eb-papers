% ----------------------------------------------------------
% OVERVIEW
% ----------------------------------------------------------
\section{Overview}

Many operational forecasting environments—such as production, staffing,
fulfillment, logistics, energy management, and short-horizon retail demand—
exhibit strong asymmetry in the consequences of forecast error. Traditional
symmetric metrics (e.g., RMSE, MAE, MAPE) treat positive and negative deviations
equivalently, but in practice the cost of underforecasting often far exceeds the
cost of surplus. A small shortfall during a peak period may cause service
failures, throughput loss, or cascading delays, while a small overforecast is
frequently absorbed through buffers, slack capacity, or inventory. This reflects
the broader principle that forecast performance must be evaluated in terms of
decision or cost impact rather than symmetric numerical accuracy alone
\citep{armstrong2001}.

Cost-Weighted Service Loss (\CWSL) is designed for these asymmetric environments.
The metric applies explicit penalty weights to the magnitudes of shortfall and
overbuild, capturing not just the size of forecast error but its \emph{operational
consequences}. By normalizing by total realized demand, \CWSL{} expresses the
effective fraction of throughput “lost’’ due to misalignment between the forecast
and demand, enabling consistent comparison across items, locations, dayparts,
and operational settings.

\CWSL{} serves as a directionally aware, operationally aligned complement to
traditional statistical accuracy metrics. It provides insight into whether a
forecasting system meets the cost and service expectations of real-world
execution, rather than simply minimizing numerical error. This technical note
formalizes the \CWSL{} metric, illustrates its mathematical and behavioral
properties, and motivates its use in environments where asymmetric error
penalties are essential for readiness-oriented evaluation.