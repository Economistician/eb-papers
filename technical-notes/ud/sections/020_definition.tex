% ----------------------------------------------------------
% DEFINITION
% ----------------------------------------------------------
\section{Definition and Mathematical Formulation}

Let $e$ denote the entity being forecasted (item, store, workload type, capacity
stream, or similar), and let $T$ represent the set of evaluation intervals. For each
interval $t \in T$, let $y_{et}$ denote realized demand (or load) and let
$\yhat_{et}$ denote the corresponding forecast. A shortfall occurs in interval $t$
when $\yhat_{et} < y_{et}$.

Define the shortfall magnitude as
\[
s_{et}
=
\pospart{y_{et} - \yhat_{et}}
=
\max\!\bigl(y_{et} - \yhat_{et},\, 0\bigr).
\]

The set of shortfall intervals for entity $e$ is
\[
T^{SF}_e
=
\{\, t \in T : s_{et} > 0 \,\}.
\]

Underbuild Depth measures the average shortfall magnitude over these intervals:
\[
\UD_e
=
\frac{\sum_{t \in T^{SF}_e} s_{et}}{|T^{SF}_e|}
\quad \text{for } |T^{SF}_e| > 0.
\]

Intervals in which $\yhat_{et} \ge y_{et}$ do not contribute to the metric. When no
shortfalls occur, $\UD$ is undefined; practical implementations may return zero or a
missing value depending on whether the distinction between ``no shortfalls'' and
``no depth to average'' is operationally meaningful.