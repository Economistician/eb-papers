% ----------------------------------------------------------
% OVERVIEW
% ----------------------------------------------------------
\section{Overview}

Underbuild Depth (UD) measures the average magnitude of forecast shortfalls,
conditional on a shortfall occurring. While coverage-oriented metrics describe
whether a forecast meets realized demand, UD isolates the severity of the misses
that do occur. This severity dimension is particularly relevant in high-frequency
operational environments—such as QSR production systems, retail replenishment
cycles, manufacturing lines, staffing operations, and logistics networks—where
deep underbuilding can produce substantial disruption even when shortfalls are
relatively infrequent.

In these settings, the depth of a shortfall often determines the extent of
operational impact, including delayed recovery, throughput degradation, and
service instability. UD provides a concise measure of this effect by averaging the
positive part of underbuilding across shortfall intervals. By focusing exclusively
on intervals in which demand exceeds the forecast, the metric reveals whether a
forecasting system tends to miss narrowly or by margins large enough to trigger
material performance degradation. Importantly, UD is an evaluative severity
diagnostic: it characterizes the depth of underbuild events under observed
forecasts, but does not prescribe buffer sizes, staffing levels, or operational
control actions.

This technical note formalizes the Underbuild Depth metric, outlines its
mathematical and operational properties, and illustrates how the measure captures
the magnitude dimension of underbuilding in interval-based forecasting contexts.
