% ----------------------------------------------------------
% LIMITATIONS
% ----------------------------------------------------------
\section{Limitations and Appropriate Use}

Although \UD\ provides a focused measure of shortfall severity, several limitations
should guide its interpretation and application. In particular, \UD\ is intended
as an evaluative diagnostic of underbuild depth and should not be optimized in
isolation or treated as a prescriptive control target; it is most informative
when interpreted alongside complementary frequency- and balance-based metrics.

\subsection{Undefined When No Shortfalls Occur}

When no shortfall intervals are observed, \UD\ is undefined because there is no
depth to average. Implementations may return zero or a missing value depending on
whether the distinction between ``no shortfalls'' and ``no depth'' is
operationally meaningful.

\subsection{No Frequency Information}

\UD\ measures severity rather than incidence. Forecasting systems with identical
\UD\ values may exhibit very different frequencies of shortfalls. For a complete
assessment of forecast readiness, \UD\ should be paired with coverage-based
metrics.

\subsection{Scale Dependence}

Because \UD\ reflects the magnitude of unmet demand, it is sensitive to the scale
of the underlying process. Comparisons across entities with different demand
levels require normalization or supplemental diagnostics.

\subsection{Sensitivity to Rare Extremes}

A small number of deep shortfalls can disproportionately elevate \UD. While this
behavior is often desirable for risk identification, it should be interpreted
carefully in noisy or highly volatile environments.

\subsection{No Information on Overbuilding}

\UD\ does not capture surplus capacity or overforecasting behavior. Evaluations
concerned with the balance between underbuilding and overbuilding require
additional metrics.

\subsection{Dependence on Interval Resolution}

The choice of interval length affects the metric. Shorter intervals may reveal
localized deep shortfalls, while longer intervals may obscure them. \UD\ should be
computed at the operational decision cadence relevant to the system being
evaluated.
