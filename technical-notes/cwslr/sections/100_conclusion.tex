% ----------------------------------------------------------
% 010 CONCLUSION
% ----------------------------------------------------------
\section{Conclusion}
\label{sec:conclusion}

Asymmetric cost assumptions are unavoidable in operational forecasting.
The question is not whether asymmetry exists, but how it is specified,
justified, and governed.
This technical note has introduced a principled, data-driven approach to
calibrating asymmetric cost ratios for use with Cost-Weighted Service Loss
(\CWSL{}).

By evaluating \CWSL{} over a discrete grid of candidate cost ratios and selecting
the point of cost balance, the proposed method anchors asymmetry in observed
forecast behavior rather than subjective preference.
The resulting calibration is deterministic, transparent, and reproducible,
making it suitable for production evaluation pipelines and governance contexts.

Crucially, calibration is positioned as a preprocessing step rather than an
optimization shortcut.
It defines the evaluation environment in which forecasts are judged, preserving
the integrity of model comparison and preventing post hoc metric manipulation.
This separation ensures that readiness scores remain interpretable across models,
entities, and time periods.

At finer levels of granularity, entity-specific calibration reveals systematic
differences in forecast behavior that would otherwise be obscured by global cost
assumptions.
These differences are not statistical noise, but actionable signals that support
segmentation, targeted intervention, and readiness-aware deployment strategies.

Within the Forecast Readiness Framework, grid-based cost calibration functions as
a readiness primitive.
Together with tolerance calibration (HR@\(\tau\)), shortfall avoidance (NSL),
loss depth (UD), and readiness adjustment layers, it forms a coherent evaluation
stack that aligns statistical forecasts with real operational consequences.

The intent of this work is not to prescribe a single ``correct'' cost ratio, but
to establish a defensible baseline from which informed risk preferences can be
applied. In doing so, it advances forecast evaluation from abstract accuracy toward
decision-aligned readiness, while remaining explicitly evaluative rather than
prescriptive with respect to downstream operational decisions.
