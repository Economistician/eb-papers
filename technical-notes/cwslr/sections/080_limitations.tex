% ----------------------------------------------------------
% LIMITATIONS
% ----------------------------------------------------------
\section{Limitations}
\label{sec:limitations}

While balance-based calibration provides a principled and transparent method for
selecting asymmetric cost ratios in \CWSL{}, several limitations should be
acknowledged.

First, the calibration procedure is inherently retrospective.
It relies on historical forecast errors and assumes that the relative frequency
and magnitude of underbuild and overbuild observed during calibration are
representative of future operating conditions.
Structural changes in demand patterns, capacity constraints, or operational
policies may therefore require periodic recalibration.

Second, balance-based calibration equalizes realized underbuild and overbuild
\emph{cost mass}, not necessarily true economic cost.
In environments where the actual cost functions are highly nonlinear,
time-varying, or state-dependent, the inferred ratio should be interpreted as an
\emph{effective asymmetry} rather than a literal estimate of financial loss.

Third, the approach depends on the choice of the candidate grid of ratios.
Although coarse grids are often sufficient to reveal stable regions, excessively
narrow or sparse grids may obscure meaningful structure.
Conversely, very fine grids may introduce apparent precision that exceeds the
information content of the data.

At the entity level, small sample sizes pose an additional challenge.
Entities with limited history may produce unstable or extreme ratio estimates.
For this reason, practical implementations should enforce minimum sample sizes,
apply optional global caps, or pool information hierarchically when appropriate.

Finally, balance-based calibration does not encode organizational risk appetite
or strategic priorities directly.
In some contexts, decision-makers may intentionally favor persistent
shortfall-avoidance or surplus-avoidance even when historical costs appear
balanced.
In such cases, calibrated ratios should be viewed as informative diagnostics
rather than binding prescriptions.

These limitations do not diminish the utility of the method, but they emphasize
the importance of treating cost-ratio calibration as a governed modeling
component—subject to review, validation, and revision—rather than a one-time
optimization.