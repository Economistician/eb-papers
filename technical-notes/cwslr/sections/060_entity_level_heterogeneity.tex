% ----------------------------------------------------------
% ENTITY-LEVEL HETEROGENEITY
% ----------------------------------------------------------
\section{Entity-Level Cost Asymmetry}
\label{sec:entity_level}

Operational systems rarely exhibit homogeneous cost asymmetry across all
entities.
Products, locations, services, or demand segments may differ substantially in
their tolerance for shortfall versus surplus.
A single global cost ratio can therefore obscure systematic heterogeneity and
mask localized readiness risks.
This section extends balance-based calibration to the entity level while
introducing safeguards to preserve interpretability and stability.

\subsection{Motivation for entity-specific calibration}
\label{subsec:entity_motivation}

Let $\entity \in I$ index entities such as items, stores, or service classes.
Historical forecast errors often reveal persistent differences in how entities
experience shortfall and overbuild.
For example, a high-volume staple may incur severe penalties when underbuilt,
while a low-volume or highly substitutable item may tolerate occasional shortage
with limited impact.
Applying a single global $\R{}$ in such settings conflates these regimes and can
produce misleading evaluations.

Allowing entity-specific ratios $\Ri{}$ enables evaluation to reflect these
systematic differences.
However, unconstrained per-entity calibration risks overfitting to noise,
particularly for sparse or volatile entities.
The objective is therefore not maximal granularity, but controlled heterogeneity.

\subsection{Entity-level calibration rule}
\label{subsec:entity_rule}

For each entity $\entity$, we apply the balance-based calibration rule to the
subset of historical observations associated with that entity.
Given a minimum sample requirement $n_{\min}$, we compute
\[
    \Rist \in
    \arg\min_{\R \in \Rgrid}
    \left|
        \UnderCost_{\entity}(\R{}) - \OverCost_{\entity}(\R{})
    \right|,
\]
where $\UnderCost_{\entity}(\R{})$ and $\OverCost_{\entity}(\R{})$ denote realized
underbuild and overbuild costs aggregated over entity $\entity$.
The resulting $\Rist$ reflects historical cost balance for entity $\entity$
under the observed error distribution and should be interpreted as a reference
evaluation assumption rather than an estimate of true or future optimal cost
asymmetry.

Entities with fewer than $n_{\min}$ finite observations are excluded from
calibration and assigned no entity-specific ratio.
This constraint prevents unstable estimates driven by small samples and makes
the calibration behavior explicit.

\subsection{Global caps and safeguards}
\label{subsec:entity_safeguards}

Even with sufficient data, entity-level calibration can produce extreme ratios
when historical errors are highly unbalanced.
To prevent tolerance inflation or pathological asymmetry, we introduce optional
global safeguards.

A common approach is to cap entity-level ratios by a global reference value,
derived from the full dataset.
Let $\Rstar$ denote the globally calibrated ratio from
Section~\ref{sec:calibration}.
We enforce
\[
    \Rist \leftarrow \min(\Rist, \Rstar),
\]
or, more generally, cap $\Rist$ by a high quantile of the empirical distribution
of entity-level ratios.
Such caps ensure that local calibration does not override system-wide governance
principles.

\subsection{Interpretation and reporting}
\label{subsec:entity_interpretation}

Entity-level ratios should be reported alongside diagnostics including sample
size, achieved balance, and any applied caps.
These diagnostics are essential for distinguishing meaningful heterogeneity from
noise-driven variation.

Entity-level calibration should be treated as a \emph{diagnostic and analytical
tool first}, used to surface structural differences and potential readiness
risks, rather than as a default mechanism to be deployed uniformly across all
entities.

Importantly, entity-level calibration is not intended to replace global
calibration.
Rather, it provides a refinement layer that can be applied selectively when
operational context and data availability justify additional complexity.
In many deployments, a hybrid approach—global calibration with entity-level
overrides for well-supported cases—offers the best balance between fidelity and
governance.

The next section consolidates these ideas by discussing practical diagnostics,
reporting conventions, and governance workflows that support responsible use of
cost asymmetry in production environments.
