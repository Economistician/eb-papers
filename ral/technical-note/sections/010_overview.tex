% ----------------------------------------------------------
% OVERVIEW
% ----------------------------------------------------------
\section{Overview}

The Readiness Adjustment Layer (RAL) is a post-forecast control mechanism within
the Forecast Readiness Framework (FRF) designed to improve operational outcomes
under asymmetric cost and executional risk. Rather than modifying forecasting
models or retraining predictive systems, RAL applies bounded, deterministic
adjustments to forecast outputs in order to reduce decision-relevant loss and
mitigate systemic under-readiness. This distinction reflects the practical reality
that forecast generation and operational execution are often decoupled, with
limited opportunity to revise upstream models once decisions must be made.

Importantly, RAL is not an optimizer over forecasts, nor a post-hoc tuning device
intended to improve apparent accuracy. Instead, it operates as a
\emph{readiness-preserving control layer}, applying constrained adjustments within
a predefined feasibility envelope to align forecasts with operational cost and
risk considerations at decision time. This framing distinguishes RAL from both
model training and evaluation-stage optimization, and prevents the adjustment
process from being interpreted as forecast manipulation.

In many operational environments, forecast accuracy alone is insufficient to
ensure readiness. Under asymmetric cost structures, under-forecasting may incur
irreversible penalties—such as unmet demand, service degradation, or capacity
shortfalls—while over-forecasting carries comparatively lower or more recoverable
costs. Even forecasts that are unbiased or optimal under symmetric error measures
can therefore lead to systematically fragile outcomes. RAL addresses this gap by
explicitly incorporating cost asymmetry and readiness constraints into a controlled
adjustment process applied after forecasting but before execution.

The adjustment performed by RAL is intentionally constrained. Adjustments are
bounded within predefined feasibility limits and evaluated against a
decision-oriented loss function, ensuring that forecast modifications remain
interpretable, auditable, and operationally defensible. When no improvement is
possible within the allowable bounds, RAL defaults to the identity transformation,
preserving the original forecast. This design emphasizes risk mitigation and
robustness rather than aggressive optimization.

This technical note formalizes the Readiness Adjustment Layer, defines its inputs
and outputs, and describes the control logic governing forecast adjustments. It
also outlines the behavioral properties, limitations, and intended use cases of
RAL, situating it as a complementary mechanism to forecast evaluation metrics
rather than a replacement for predictive modeling.