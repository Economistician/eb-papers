% ----------------------------------------------------------
% CONCLUSION
% ----------------------------------------------------------
\section{Conclusion}

The Readiness Adjustment Layer (RAL) provides a structured mechanism for improving
operational readiness in environments where forecast errors carry asymmetric and
potentially irreversible costs. By applying bounded, deterministic adjustments to
forecast outputs, RAL addresses a gap between predictive accuracy and executional
sufficiency that is not resolved by traditional evaluation metrics alone.

RAL is intentionally positioned downstream of forecasting and upstream of
execution. It does not modify or replace forecasting models, nor does it attempt
to optimize predictive accuracy. Instead, it formalizes the discretionary
adjustments that operators often apply informally, grounding them in explicit
constraints and decision-oriented loss considerations.

Crucially, RAL is not an optimizer over forecasts, but a readiness-preserving
control layer operating within a pre-defined feasibility envelope, ensuring that
adjustments remain interpretable, auditable, and operationally defensible.

When applied in appropriate contexts, RAL can reduce under-readiness risk while
preserving organizational control and decision transparency. Its design
emphasizes robustness over aggressiveness and governance over opportunistic
performance gains, making it suitable for operational environments where
reproducibility and accountability are as important as loss reduction.

As part of the Forecast Readiness Framework, the Readiness Adjustment Layer
complements forecast evaluation metrics and readiness diagnostics by translating
assessment into controlled action. Its effectiveness ultimately depends on
accurate loss specification, well-chosen constraints, and disciplined governance,
underscoring the importance of integrating adjustment mechanisms within a broader
readiness-oriented decision framework.