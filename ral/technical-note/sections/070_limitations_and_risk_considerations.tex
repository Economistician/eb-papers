% ----------------------------------------------------------
% LIMITATIONS AND RISK CONSIDERATIONS
% ----------------------------------------------------------
\section{Limitations and Risk Considerations}

The Readiness Adjustment Layer (RAL) is designed as a controlled mechanism for
post-forecast adjustment under asymmetric cost and readiness constraints.
However, its effectiveness and appropriateness depend on the quality of its
inputs and the context in which it is applied. This section outlines key
limitations and associated risks to inform responsible use.

\subsection{Dependence on Loss Specification}

RAL relies on a decision-oriented loss function to evaluate candidate adjustments.
If the loss function does not accurately reflect operational priorities or cost
asymmetry, the resulting adjustments may be misaligned with actual risk. In
particular, miscalibration of under-forecasting penalties can lead to excessive
over-preparation or insufficient mitigation of readiness failures.

Loss specification is treated as an external responsibility; RAL does not infer,
validate, or correct cost parameters.

\subsection{Constraint Misconfiguration}

Adjustment bounds play a central role in governing RAL behavior. Bounds that are
too restrictive may prevent meaningful improvement, while overly permissive
bounds may allow adjustments that exceed practical or organizational limits.
Because RAL enforces but does not derive feasibility constraints, improper bound
selection can undermine the intended safeguards of the mechanism.

\subsection{Masking of Upstream Model Issues}

As a post-forecast adjustment layer, RAL does not correct structural deficiencies
in the underlying forecasting model. Persistent reliance on large or frequent
adjustments may indicate model bias, regime change, or degraded predictive
performance. Without appropriate monitoring, RAL could obscure these issues by
compensating for upstream errors rather than prompting corrective action.

\subsection{Context-Specific Validity}

The benefits of RAL depend on the presence of asymmetric cost and limited
executional flexibility. In environments where errors are symmetrically penalized
or where corrective actions can be taken dynamically and costlessly, RAL may
provide little or no benefit. Applying RAL outside its intended context risks
introducing unnecessary complexity without corresponding operational value.

\subsection{Interpretation and Governance Risk}

Adjusted forecasts produced by RAL may be misinterpreted as improvements in
predictive accuracy rather than as readiness-oriented modifications. Clear
governance, documentation, and communication are required to ensure that
stakeholders understand the purpose and limitations of the adjustment layer.
Failure to maintain this distinction could lead to inappropriate reliance on RAL
outputs or misaligned performance evaluation.

\subsection{Audit and Oversight Considerations}

Although RAL is deterministic and bounded, effective oversight depends on the
availability of diagnostic outputs and adjustment rationale. Organizations
deploying RAL should ensure that adjustment parameters, loss specifications, and
before-and-after metrics are retained for audit and review. Absent such controls,
the transparency benefits of the mechanism may be diminished.