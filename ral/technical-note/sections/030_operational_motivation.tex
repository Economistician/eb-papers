% ----------------------------------------------------------
% OPERATIONAL MOTIVATION
% ----------------------------------------------------------
\section{Operational Motivation}

Forecasting systems are typically evaluated using accuracy-oriented metrics that
treat overestimation and underestimation symmetrically. While such measures are
useful for model comparison, they often fail to reflect the operational reality
in which forecast errors translate into uneven and irreversible consequences.
This distinction between predictive accuracy and decision relevance is well
established in the forecasting literature \citep{gneiting2011}.
In many executional environments, under-forecasting leads to lost demand, service
degradation, or capacity shortfalls that cannot be fully recovered, whereas
over-forecasting may result in softer penalties such as idle capacity or
temporary inefficiency. As a result, forecasts that appear adequate under
symmetric error criteria may nonetheless expose the system to persistent
readiness risk.

This asymmetry creates a structural disconnect between predictive accuracy and
operational sufficiency. Even well-calibrated or unbiased forecasts can produce
systematically fragile outcomes when decision thresholds are tight or when
response times are long relative to demand variability. In such settings, the
cost of being unprepared often exceeds the cost of modest over-preparation,
suggesting that post-forecast adjustments may be necessary to align forecasts
with executional priorities.

From a decision-science perspective, this fragility reflects sensitivity to
implicit assumptions rather than model failure. Robust decision-making therefore
requires explicit examination of how outcomes change under bounded variations
in controllable assumptions, a principle emphasized in the sensitivity analysis
literature \citep{saltelli2008}. In operational contexts, these assumptions often
take the form of cost asymmetry, tolerance thresholds, or readiness constraints
that are external to the forecasting model itself.

Operational constraints further complicate the use of traditional forecasting
approaches. In many organizations, forecasting models are developed, validated,
and deployed on cadences that do not align with day-to-day operational decisions.
Once a forecast has been generated, there may be limited opportunity to retrain
models, incorporate new information, or alter upstream assumptions. However,
decision-makers may still retain limited flexibility to adjust commitments,
buffers, or resource allocations within predefined bounds. These realities
motivate the need for a mechanism that operates downstream of forecasting while
remaining upstream of execution.

The Readiness Adjustment Layer addresses this gap by providing a controlled means
of modifying forecast outputs in response to asymmetric cost and readiness
considerations. By constraining adjustments within operationally feasible limits
and evaluating them against a decision-oriented loss function, RAL enables
forecasts to better support execution without undermining the integrity of the
underlying predictive models. This approach emphasizes robustness and risk
mitigation rather than maximal accuracy, reflecting the practical demands of
operational decision-making.