% ----------------------------------------------------------
% USE CASES
% ----------------------------------------------------------
\section{Use Cases}

The Readiness Adjustment Layer (RAL) is intended for operational environments
where forecasts directly inform decisions with asymmetric cost and limited
opportunity for upstream model revision. This section outlines representative
use cases where a post-forecast adjustment mechanism provides practical value.

\subsection{Capacity and Resource Planning}

In capacity-constrained systems—such as staffing, production, or logistics—under-
forecasting demand can result in service failures or unmet demand that cannot be
recovered after the fact. Over-forecasting, by contrast, may lead to idle
resources or temporary inefficiency. RAL enables modest, bounded adjustments to
baseline forecasts that reduce under-readiness risk while respecting operational
constraints on over-preparation.

\subsection{Inventory and Buffer Management}

Inventory planning often involves explicit buffer limits and asymmetric holding
versus stockout costs. When forecasting models are calibrated for long-term
accuracy, short-term readiness may still be insufficient under volatile demand.
RAL allows forecasts to be adjusted within predefined buffer ranges to better
align with stockout risk tolerance without altering the underlying demand model.

\subsection{Service-Level–Driven Operations}

In environments governed by service-level agreements or internal reliability
targets, the cost of falling below a readiness threshold can far exceed the cost
of exceeding it. RAL provides a structured mechanism for adjusting forecasts to
prioritize service reliability while maintaining transparency around the
magnitude and rationale of adjustments.

\subsection{Operational Decision Pipelines}

Many organizations operate decision pipelines in which forecasting, planning, and
execution are handled by distinct systems or teams. Once forecasts are published,
downstream operators may have limited authority to alter inputs but retain
discretion over buffers, contingencies, or safety margins. RAL formalizes this
discretion into a consistent, auditable process that bridges forecasting outputs
and executional requirements.

\subsection{When RAL Is Not Appropriate}

RAL is not intended for environments where forecast errors are symmetrically
penalized, where adjustments can be made continuously and costlessly at execution
time, or where upstream models can be readily retrained to incorporate changing
conditions. In such cases, traditional forecasting or adaptive modeling
approaches may be more appropriate.